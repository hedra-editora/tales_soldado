\chapter*{}

\vspace*{\fill}

\thispagestyle{empty}

\begin{flushright}
para o amigo R. S., \emph{weitermachen!}
\end{flushright}

\chapter*{Prólogo\\
Somos contemporâneos de nossa escravidão}
\addcontentsline{toc}{chapter}{Prólogo}

Chico Buarque de Holanda conta uma história de quando menino, no final
dos anos 1950, em que ele reclamou com seu pai de como, ao pegar um taxi
dirigido por um motorista negro, Sérgio Buarque iria se sentar ao lado
do motorista. Então Sérgio o fez descer do carro e, na rua, lhe disse
com dureza que ele nunca mais repetisse aquilo, que ele nunca mais
desprezasse e menosprezasse de modo racista, e falsamente superior, o
povo brasileiro\ldots{} No entanto, como se sabe, nem todos os brasileiros
tiveram como pai Sérgio Buarque de Holanda. E é certo que aquele menino
inteligente tirara aquelas ideias torpes sobre os códigos a serem
seguidos entre as classes sociais, e a expressão das raças, de algum
lugar. Um lugar ainda mais amplo e forte, que produzia mesmo aquele gozo
sádico de \emph{não pensamento}, chamado Brasil, que não era legitimado
por seu próprio pai, de fato uma outra entidade, representante de outro
Brasil.

Por essas e por outras o próprio Sérgio Buarque de Holanda orientou a
sua pesquisa original sobre as origens do Brasil para algum horizonte de
transformação efetiva das coisas, uma proposição concreta de avanço de
racionalidade institucional, de lei acordada ordenadora de instituições
e dos direitos públicos, na busca de uma real democracia formalmente
garantida, para dar conta das formas de mentalidade escravocratas que
ainda circulavam, não apenas na relação da elite com os pobres e negros,
mas também nos vínculos dependentes, sem mediação e \emph{cordiais} da
relação dos homens livres com o poder e das próprias elites entre si. A
esperança de substituir o afeto calculado e dependente e o mundo do
favor pela lei e pelo reconhecimento de direitos configuraria a nossa
revolução modernizante, segundo o projeto afirmativo de \emph{Raízes do
Brasil}.

No entanto, a corrosão do caráter do fundo de violência não superado
brasileiro, que muitas vezes quando se trata do direito à violência
direta em relação aos pobres é ainda sustentada contra qualquer ordem de
direito existente entre nós, nunca permitiu que a profecia de fé
racionalizante de Sérgio Buarque se realizasse plenamente por aqui. Se
não, como explicar a recente e absolutamente escandalosa absolvição, em
instância superior após a condenação em um primeiro tempo, dos policiais
assassinos de 111 presos, quase todos negros, no Massacre do Carandiru
de 1992, ou o assassinato de 493 jovens, também na sua maioria negros,
pela polícia paulista, nos dias seguintes à revolta do \versal{PCC}, em maio de
2006, que jamais foram esclarecidos, julgados ou punidos? Como explicar
que a elite paulistana faça questão de tirar \emph{selfies} com esta
mesma polícia, afirmando assim o vínculo de proteção de classe, quase
particular, da polícia que se privatiza, e legitimando todas as suas
ações de extermínio contra jovens pobres e negros no Brasil? Como
explicar o acomodamento geral frente os atuais nove assassinatos por dia
cometidos pelas polícias do Brasil?

E mesmo o banal esforço político de um estado de direito minimamente
democrático foi prática e amplamente um projeto fracassado no tempo em
que Sérgio Buarque viveu, se considerarmos os reais 60 anos de regimes
antisociais, anti"-democráticos autoritários ou ditatoriais brasileiros,
incluindo aí a famigerada República Velha, o Estado Novo getulista e a
violenta ditadura civil"-militar de 1964-1984, conjunto de obra de elites
que deformou completamente, em termos humanos, sociais e espirituais, o
terrível século \versal{XX} brasileiro. Apesar do esforço generoso, produtivo e
esteticamente brilhante de nosso modernismo, artístico e intelectual,
uma corrente do espírito que forçou e alterou muitos aspectos do Brasil
moderno arcaico, o que significam de fato \emph{estes sessenta anos em
cem} de regimes autoritários, antissociais, discricionários e de
altíssima concentração de poder em pleno século \versal{XX} brasileiro?

Com algum ajuste de lentes, e com olhos livres, podemos ver em tal
dificuldade democrática limite da elite brasileira em promover acesso e
integração de massas no espaço da riqueza nacional, tão indecentemente
orientada para o alto, uma verdadeira repetição da estrutura
concentracionária, de exclusão absoluta do mundo do trabalho do mundo do
direito, própria de nosso século \versal{XIX} escravista. Nossos surtos
autoritários de poder concentrado, tão próprios do século da
modernização acelerada nacional, e hoje de caráter neo"-liberal, não
estariam apenas repondo a ordem mais séria e constante, a ordem
ocidental capitalista escravista periférica, cuja fronteira terrível
atravessou a nossa própria nação como uma verdade de força irrecusável?
E como ficam os sujeitos \emph{não modernos brasileiros}, que apesar de
negociarem a produção e a riqueza nacional no mercado mundial, e sempre
participarem das benesses da internacionalização do capital, são em
geral incapazes de produzir integração social suficiente e em níveis de
desenvolvimento contemporâneos, isto de modo constante, em todos os
tempos do processo histórico brasileiro?

O desprezo e o ódio destrutivo contra o esforço lulo"-petista de alterar
o grau de integração social brasileira, o que ocorreu, como se sabe,
durante dez anos, é um sinal forte de um ato social que visa abertamente
riqueza, privilégios e poder e que também recusa abertamente a
responsabilidade social e o desejo de caráter coletivo, o mais pleno
direito à vida cidadã brasileira. O que sabemos bem do Brasil é que o
pais produz poder, formas sociais e muita riqueza, e o faz geralmente
sobre os escombros da reprodução histórica da miséria, da pobreza e da
exclusão; e para tanto, sendo assim, em grande parte a nação militariza
fortemente a sua vida pública, transformando a ação social necessária em
caso de polícia e exclusão sempre tolerada, que produz o verdadeiro
veneno subjetivante político que embala nossos extremamente medíocres
conservadores.

A escravidão brasileira de ao menos trezentos anos é um fato histórico
assinalado com alguma insistência -- uma vez que irrecusável -- em nossa
vida social, cultural e política de ainda hoje; ao mesmo tempo, é um
universo recusado em profundidade do ponto de vista da verdade de seus
próprios efeitos e razões, muito provavelmente, e sobre o corpo de
muitos, \emph{ainda} \emph{simplesmente} \emph{presentes} na real
\emph{geopolítica} \emph{psíquica} do Brasil de hoje. O Brasil sempre
foi contemporâneo de sua escravidão.

Desprezo social radical bem afirmado, concentração de poder tolerada,
direito à exploração máxima da vida dos pobres sem a necessidade de
conceder direito algum ao seu outro de classe entendido como pleno
objeto, servilismo global e minoridade conceitual frente às ordens de
poder são pontos constantes da vida simbólica de nosso conservadorismo
social mais comum. E também, é importante lembrarmos, são dimensões
político subjetivas coincidentes, ponto a ponto, com a estrutura
simbólica de nossa vida escravocrata nacional original.

No seu maravilhoso e explosivo, radical e exigente libelo
antiescravista, \emph{O abolicionismo}, Joaquim Nabuco já anotara com
grande clarividência que o problema da escravidão continuada brasileira
não estava no projeto imediato de sua extinção, já então, em 1883,
inevitável e sensível no horizonte, mas no quanto e por quanto tempo ela
perduraria como estrutura social de mentalidades, com seus efeitos
práticos e subjetivantes deletérios, a travar e a arrastar o
desenvolvimento social e humano do país periférico, desviando"-o para uma
região correspondente às suas próprias formas fortes e violentas.

Daquele ponto de vista, que é também o nosso, a escravidão não era um
mal que cicatrizasse com o ato histórico simples de sua suspensão -- por
mais que o Império escravocrata brasileiro do século \versal{XIX} tenha resistido
com brios à ideia da liberdade, e também, muito por causa disso. Ao
contrário, a escravidão brasileira era um mal continuado, pervasivo e
geral, que atravessaria tempos e normas de muitas distâncias históricas
diferentes, que manteria estruturas de poder e de desrespeito humano
ainda por dezenas, \emph{e} \emph{talvez} \emph{centenas}, de anos e que
produziria novas e fantásticas práticas de exceção ao cânone político
ocidental dos direitos humanos fundamentais. Nossa escravidão nacional,
desenvolvida durante a mais plena modernidade, sobretudo, marcaria a
formação do caráter de muitos como uma \emph{neurose} \emph{principal}
\emph{instalada}, gozosa perversa, que continuaria a produzir as formas
brasileiras de violência em relação aos negros e aos pobres através dos
tempos -- e, contemporaneamente, penso que em relação aos gays e trans,
noias e mendigos, imigrantes e exilados, desadaptados, ou que cultivam
religiões afro brasileiras, e por aí afora\ldots{} -- em um movimento que
percorre um tempo histórico de longa duração, indeterminado em seus
limites.

Assim, nossa formação transnacional, atlântica, luso africana, católica,
de força concentracionária e de povo desprovido de direitos, de colônia
de exploração mercantil, orientada desde sempre para o mercado global e
para os poderes centrais, teria produzido uma forma intensa de
articulação interior do \emph{sujeito do poder} por estas paragens -- e
aqui, por um segundo, eu quase disse \emph{pastagens\ldots{} --} entre nós e
neste projeto singular de nação. De fato, no Brasil, o sujeito do poder
colonial, escravocrata, está na base do sujeito do poder nacional,
originalmente escravocrata também. Tais sujeitos eram o mesmo e, para
compreendê"-lo, em conjunto com os mecanismos absurdos de concentração da
renda precisamos conceber \emph{uma genealogia da crueldade} particular
brasileira. E, desde já podemos afirmar que, no Brasil, sempre, toda
constante e absurda sinalização \emph{contra o valor universal dos
direitos humanos} é de fato \emph{um ato derivado diretamente de nossa
origem e de nosso constante desejo escravista,} que assim é reposto, não
repousa e nem se desmobiliza.

Nossos grosseiros cidadãos que atacam a base acordada fundamental da
cidadania ocidental, que recusam felizes os direitos humanos universais,
sempre com evidente prazer sádico do direito local de contradizer a lei
geral mais ampla, deveriam ser abertamente chamados de
\emph{neo"-escravistas} brasileiros -- e não de genericamente
\emph{fascistas} como costumamos fazer -- se não forem os constantes
\emph{escravistas reais} brasileiros, fundo autoritário extremo que de
fato os sustenta e que eles representam.

A nação consciente, desejada e produtora de lutas democratizantes,
realizada em formas públicas possíveis de maior ou de menor integração
democrática, teria sempre que conviver com o sujeito originário do ato
de violência escravocrata, de desrespeito absoluto pelo corpo e pelo
direito do outro, de máxima exploração e de felicidade de disposição
sádica dos senhores, do gesto de força garantido sobre a vida dos
trabalhadores, estrangeiros africanos, alucinadamente forçados para os
limites da humanidade dita brasileira. Uma estruturação sádica subjetiva
forte, e preguiçosa quanto ao valor do próprio trabalho, que atravessa
tempos históricos heterogêneos, que resiste -- como simplesmente
resistem os desejos infantis no modelo psicanalítico dos sonhos -- e que
insiste em uma nação e cultura que, devendo avançar como
\emph{progresso}, não deve superar a sua violência originária como
\emph{ordem}.

A escravidão é o sonho limite de nossos autoritários, tão constantemente
presentes na história brasileira e o pesadelo real de todos os que a
sentem e combatem. Ela é a \emph{ordem} prévia, desejada e imposta na
articulação de sadismo e classismo entre nós, nossa infeliz
\emph{origem} \emph{imanente}, muito anterior a toda vida pública
democrática e includente.

Mais ou menos, os traços luso ibéricos católicos de nossa vida de baixa
produtividade colonial, escravocrata e de tipo \emph{ancién regime},
além de concreta e pouco imaginativa a respeito do tipo de ordem que
representava, deixaram marcas profundas em nossa estrutura social aberta
ao desenvolvimento moderno, de nosso século \versal{XX} ao menos. Como Caio Prado
Jr. não se cansou de assinalar, aquele mundo era de baixa produção de
capital e de alguma sociedade complexa de classes e de consumo, de
alguma ordem de contrato e de reconhecimento e de alguma ciência
possível, que pudesse ser por fim convertida em fator real de capital e
chegar a completar a estrutura econômica da nação moderna. E é
impressionante mesmo reconhecermos como algo destas condições gerais
originais nos afeta ainda agora, com a nova ordem autoritária grave da
política brasileira de nosso próprio tempo presente, que nos ameaça
invadir com suas marcas também o século \versal{XXI} brasileiro, que já se perde.

Estudando os raros nomes e conceitos dados à nossa escravidão na origem
nacional no século \versal{XIX} brasileiro, eu pude verificar seis ações de
sentido presentes na cultura a respeito de nossa forma de produção de
riqueza e sociedade que cindia a vida entre escravos e senhores.

Seis principais modos de conceber nossa escravidão e nossa vida popular
e cultura que tinha contato com o povo trabalhador, escravo, modos que,
de um jeito ou de outro tem vida longa na história das ideias e da
subjetivação no Brasil:

O modo do silêncio, ou \emph{\textbf{o} \textbf{poder} \textbf{como}
\textbf{ato} \textbf{e} \textbf{recusa} \textbf{simbólica}}: este foi o
modo principal do poder e das elites\emph{, todos liberais quando
conservadores escravocratas, e conservadores escravocratas quando
liberais,} operar a cultura da escravidão, ao menos até a década de 1850
no Brasil. Não falar da coisa, ato correlato de se impedir de pensá"-la,
fazia com que se produzisse uma cultura satisfeita no puro ato, do
privilégio do gozo da situação como ela era, do gozo do sadismo e da
preguiça senhoril. Não falar, não imaginar, não escrever sobre a
escravidão e a estrutura social nacional, estabeleceram as condições
arcaicas de nosso profundo e lamentável, generalizado,
\emph{anti"-intelectualismo} brasileiro. A escravidão brasileira recusava
o pensamento a seu respeito. Este mundo é o mundo da ausência de textos,
da existência do insubsistente em nossa cultura. Na ausência do
compromisso simbólico na linguagem ele vai emergir e se produzir como
\emph{coisa}, como traços materiais concretos, atos, e alucinose, o
delírio da cultura que é, mas que não se representa. Esta recusa revela
um plano nacional amplo de \emph{má fé}, frente à assunção possível de
um discurso autoritário e contra"-iluminista brasileiro, que de fato não
podia confrontar de nenhum modo os resultados sociais que as revoluções
liberais modernizantes, cientifico racionalistas europeias e americana,
produziam de modo expansivo, tendente a tomar o mundo, na época. O modo
do silêncio constrito, \emph{casmurro}, a autocensura dos senhores
brasileiros, e sua morte da imaginação estética e crítica, também se
articulava à linguagem da eficácia de governo dos fazendeiros
\emph{saquarema}, esta sim uma linguagem inevitavelmente existente, sob
pena de não haver país algum caso faltasse. Uma linguagem seca e
desencantada, sobre a escravidão como ação nacional de Estado, puro
Estado, que também está na origem de nosso pobre discurso burocrático
competente e autorizado, que parece nunca precisar responder de fato ao
Brasil \emph{como ele é}.

O modo de Vilhena, ou \emph{\textbf{o poder contra o erotismo}}: recusa
afirmativa do valor da vida popular e da cultura que dela emana,
controle estrito policial da vida dos pobres, controle político policial
dos espaços públicos das cidades. Modo de explicitude e afirmação do
poder direto e da cidade como espaço de intervenção e de desenho, morto
culturalmente, do poder normativo. Poder positivo antipopular, poder que
emana do poder, essencialmente incapaz de uma imaginação qualquer, de
algum contato onírico erótico com a vida popular brasileira. Racismo
conservador eroticamente empobrecido. Origem de nosso espírito
\emph{sério}, de repressão naturalizada e cisão constante e agressiva
com a vida popular e a cidade, modalidade subjetiva autoritária
nacional, que tem longa vida em São Paulo.

O modo de Gonçalves Dias, ou \emph{\textbf{a} \textbf{denúncia}
\textbf{do} \textbf{não} \textbf{dito}}: em alguns poemas, e
principalmente no diálogo \emph{Meditação}, de 1846, não publicado em
vida, o poeta denunciou a degradação ética, a perversão disseminada da
violência liberada e direta dos senhores, a baixíssima produtividade,
bem como a degradação da cultura material e estética da nação, causadas
pela escravidão. Ele antecipou abertamente o discurso abolicionista,
Joaquim Nabuco e até mesmo Caio Prado Jr.. Não por acaso foi o primeiro
artista importante do país. Mas Gonçalves Dias também nunca publicou o
diálogo antiescravista, confirmando, a contrapelo, o silêncio geral que
congelaria o país, e a mentalidade de sua elite, por quase meio século.

O modo de Schlichthorst, ou \emph{\textbf{gozo}, \textbf{erotismo}
\textbf{e} \textbf{cultura}}: o estrangeiro que reconhece a vida
pulsante da cidade, a metrópole escravista, produzindo formas de
civilização. A ambivalência de um olhar que vê a exceção brasileira, do
ponto de vista da onipresença da violência, mas também da necessidade da
vida viver e criar modos eficazes de existência material e simbólica.
Pela primeira vez se entende o espaço popular brasileiro como origem de
cultura moderna, e os negros já são criadores, poetas, músicos,
dançarinos. Além de vir daí a nossa pedra filosofal, muiraquitã, da
nossa chave vanguardista hiper moderna, de uma cultura constituída como
um gesto de arranjo \emph{no precário}. Atenção interessada às emanações
culturais vindas do corpo, a dança, o canto, a música de dança,
correlatos imediatos e reflexos de um erotismo presente e negociado na
cultura, principalmente no corpo da mulher escrava. Onirismo desejante,
criador de sonhos, da vida dos escravos; mais erotismo negociado, em
troca de tostões, como fonte de vida e de cultura. Pela primeira vez as
soluções de compromisso, entre corpo, erotismo e cultura, modernas ao
modo brasileiras, síntese produtiva de nossa escravidão, origem e
amalgama de fundo de um filão importante de nosso modernismo, se
enunciava, \emph{no olhar do estrangeiro}, que devorava e era devorado
pelo Brasil. Origem da nossa ambivalência erótico cultural, positiva
como lugar de reconhecimento do ato popular, mas controladora dos
limites da ação política dos pobres \emph{neste continente erótico
simbólico interessado} e, sendo assim, portanto, sexista, machista e
perversa em outro grau de compromisso. Campo cultural gozoso, a partir
de formas estéticas ligadas ao corpo e resultado do grande movimento da
mestiçagem brasileira. Ou, noutras palavras, \emph{relação}
\emph{sexual} \emph{cultural}, produtiva mas politicamente controlada e
orientada.

O modo de Gonzaga, \emph{\textbf{o} \textbf{reconhecimento} \textbf{à}
\textbf{distância}}: o poeta inconfidente mineiro, de modo simétrico ao
do mercenário alemão Schlichthorst, mas com sinal de distanciamento e
não de interesse, observou pela primeira vez a incorporação dos batuques
e danças populares na vida do palácio do poder e da elite colonial
controladora da riqueza e da vida social nas Minas Gerais. A
incorporação antropofágica interna, entre níveis de cultura e história
diferentes, ficava sinalizada, mas o ponto de vista era irônico e
negativo, e sempre do alto. O poeta parece dizer que a cultura popular,
de origem africana, poderá circular pelos circuitos híbridos da elite e
do poder, desde que os pobres continuem sempre bem postos em seu lugar
original final. O pacto cultural, conservador irônico brasileiro,
distanciado em relação ao destino real daqueles atores populares que
animavam a vida, era o que se anunciava naquele olhar distanciado,
irônico negativo, de participante ausente do destino do parceiro
cultural. E é assim que vamos, nossa classe média culta e elite, ao
carnaval e à Bahia. E é assim que o grande samba impacta nossa Bossa
Nova, e retorna aos nossos salões culturais, onde os pobres e negros
continuam nos servindo.

O modo de Alencar, \emph{\textbf{conservadorismo} \textbf{patriarcal}
\textbf{e} \textbf{atraso}}: diante da crise final da escravidão
brasileira, com a ascensão de nosso \emph{abolicionismo forte} nos anos
de 1870 e 1880, os conservadores enunciaram a sua teoria da relação
social e cultural eterna entre senhores e escravos. Com risco do fim do
sistema é necessária a emergência da voz reativa, que no pleno poder de
controle da escravidão rural e urbana brasileira, até então
absolutamente não falava. Deste ponto de vista, do poder, a escravidão
foi um bem civilizatório, que constituiu e tornou viável o novo mundo
americano, dizia o romancista nacional, porta"-voz conservador. Ela devia
ser vista com benevolência e de modo favorável -- à autoimagem dos
senhores. Para aquele discurso, a escravidão foi um processo e um bem
histórico necessário à criação de um novo mundo, e seu custo moral e
humano está pago por princípio. Também, o escravo é um
\emph{meio}-\emph{sujeito}, que pode ser feliz, pois recebe as benesses
afetivas e pessoais do seu senhor, que lhe tem interesse humano,
cristão. Sua condição social deveria evoluir, sempre muito segura e
gradualmente, para a de uma \emph{servidão tutelada}, baseada nos afetos
e direitos civilizatórios dos senhores sobre ele, e não para a plena
cidadania, estranha ao país. Muito de nossa ideologia paternalista, de
beneméritos de pobres explorados, que devem ser para sempre pobres
explorados brasileiros, se enuncia com antecedência, desfaçatez e força
estilística própria, naquele indigitado discurso pró escravista, a voz
da vida intelectual conservadora brasileira na segunda metade do século
\versal{XIX}. Horrorosa, mas realista, ideologia positiva do poder e sua
subjetivação, brasileiro.

Desprezo social radical bem afirmado, concentração máxima de poder
tolerada, direito à máxima exploração da vida dos pobres sem a
necessidade de conceder direito algum ao outro de classe, servilismo
global e minoridade conceitual diante de qualquer ordem de poder são
traços constantes de nosso conservadorismo mais comum, coincidentes
ponto a ponto com a estrutura simbólica de nossa vida escravocrata
original.

\chapter*{O Soldado Antropófago\\escravidão e \emph{não-pensamento} no Brasil}
\addcontentsline{toc}{chapter}{O Soldado Antropófago}

\section{I}

Durante muito tempo, quando nos olhávamos no espelho desde um certo modo
de ser brasileiros, por vezes víamos, alguém via, coisas que pareciam
muito belas. Daquele ponto de vista, aquela perspectiva que nos era
cara, tínhamos a famosa visão do paraíso, \emph{uma imagem utópica}
\emph{que emanava de nós mesmos}, com a qual desejávamos sonhar e viver
em ilusão partilhada: uma terra onde o homem sempre foi cordial, e a
mulher, especialmente sensual. Tal imagem íntima e coletiva, que se
ofereceu com facilidade à indústria da imaginação entre nós, falava
muito de uma terra boa, de uma terra gostosa, onde se canta o samba que
dá e a morena é sestrosa. Uma terra onde a baiana tem, e entra na roda,
deixando a moçada com água na boca. Um nosso mundo bem conhecido, entre
o desejo e a imaginação, em que nossas garotas lindas andavam por
Ipanema, malemolentes e leves, cantando canções infantis ou sambas
enredo para o Pão de Açúcar, enquanto os homens mais velhos pensando na
beleza que existe, sonhavam em levar o seu samba moderno ao Carneggie
Hall de Nova York. Tal mundo expandia de muitos modos o seu afeto
imaginativo sobre nós, desde a terra até suas mulheres, homens e
projeto, muito mais anseio do que lei reconhecida.

Assim \emph{um mundo bom} se configurou no Brasil, no qual os pobres
tinham um violão e também um fusca e uma gravata florida. Onde eles
amavam a seleção e o flamengo e tinham uma nega linda, chamada Tereza.
Talvez ela fosse até mesmo uma mulata exuberante, de Di Cavalcanti, ou
uma baiana que não era falsa, mas, ao contrário, frajola, com um
verdadeiro requebrado do lado que mexia com o juízo dos trabalhadores.
Ela era mesmo, como ouvimos tantas vezes, beleza pura, de pele escura,
de carne dura. Aqui era o lugar em que, na casa dos nossos pobres, se
comia o famoso feijão da Vicentina, todo mundo era bamba e a menina
dança, quando se dizia \emph{oba!, salve a Bahia senhor!} Um país
abençoado por Deus, onde os mais antigos nobres e pastores de nossa
inexistente, mas sonhada, Arcádia da Colônia Ultramarina de Vila Rica
\emph{não eram mesmo um qualquer}, e onde, mais de uma vez, chegou a
hora dessa gente bronzeada mostrar o seu valor. Bonito por natureza,
parecia ser um fato de origem que no Brasil até os sabiás e as palmeiras
eram mais belos do que em qualquer outro lugar da terra. E, para lá, ou
para outra \emph{Passárgada} ainda melhor, um passeio por Copacabana,
gostosa quentinha, cheia de bugigangas e suas meninas, onde sonhavam os
inocentes do Leblon, ainda voltaríamos todos, um dia. Talvez não todos,
mas cada um de nós.\footnote{A respeito da comovente e patética história
  imaginária da colônia da Arcádia Romana Ultramarina, inventada sonhada
  por Claudio Manuel da Costa, o seu ``vice"-custode'', ver Sérgio
  Buarque de Holanda, ``Glauceste acadêmico'', em \emph{O espírito e a
  letra \versal{II}}, São Paulo Companhia das Letras, 1996, e, principalmente, o
  estudo aprofundado do sentido político cultural da arcádia em Portugal
  e na sua Colônia Ultramarina de Vila Rica, ``Gosto arcádico'', em
  \emph{Esboço de figura, homenagem a Antonio Candido}, São Paulo,
  Livraria Duas Cidades, 1979.}

Um mundo, aquele nosso, em que os heróis calçavam chuteiras imortais, os
nobres e os reis eram Didis e Pelés e, quando não se tinha onde morar,
simplesmente se morava na areia. Um país que certamente tinha ouvido
musical, em que o Tio Sam queria conhecer a nossa batucada, onde o
artista mais moderno dançava na praia, com as capas da Mangueira, era
marginal e também era herói e iluminava os terreiros, os terrenos
baldios, porque nós queremos sambar e que, na Casa Branca, já dançou a
batucada de ioiô e de iaiá.

Um país no qual, quando o lindo poeta baiano era preso por algum claro
enigma, também chamado Brasil, ele saia da prisão direto para Londres,
cantando suas memórias do cárcere com infinita alegria utópica:
\emph{alô, alô, realengo, aquele abraço}. Um mundo desejado, assim,
vivido, alucinado, em que os patriarcas, os saudosos pais fundadores
dessa espécie de pátria, de origem ibérica portuguesa, católicos,
fidalgos e tardo liberais, sonhavam de fato com o futuro mais digno, da
nação como ``Tristeza do império'', nas palavras do poeta nacional em
\emph{Sentimento do mundo}:

\begin{verse}
Os conselheiros angustiados\\
ante o colo ebúrneo\\
das donzelas opulentas\\
que ao piano abemolavam\\
``bus-co a cam-pi-na se-re-na\\
pa-ra-li-vre sus-pi-rar'',\\
esqueciam a guerra do Paraguai,\\
o enfado bolorento de São Cristóvão,\\
\hspace{15pt}a dor cada vez mais forte dos negros\\
e sorvendo mecânicos uma pitada de rapé,\\
sonhavam a futura libertação dos instintos\\
e ninhos de amor a serem instalados nos \qb{}arranha-céus de\\
Copacabana,\\
{[}com rádio e telefone automático.
\end{verse}

Um mundo projetado assim, bom e violento, onde um dia todos deitaremos à
sombra de uma palmeira, que já não há.

\section{II}

Além da grande alegria e beleza, também somos um verdadeiro país jovem
-- mas que envelhece rápido -- que fêz"-se independente por ato e por
desejo da mesma monarquia portuguesa que bem nos constituiu como sua
colônia de exploração comercial exclusiva, de modo radical, ao longo de
trezentos anos. Uma comunidade imaginada cujo primeiro Rei local, Rei de
Portugal, abriu nossos portos para o mundo no tardar da hora moderna de
1808. Mais particularmente, os portos do Brasil foram abertos
precisamente para aqueles mesmos navios ingleses que escoltaram nosso
Rei imperial de Portugal em sua fuga intempestiva, com sua corte e sua
biblioteca, frente ao avanço das forças de Napoleão, então
incontroláveis, que ainda representavam a Revolução modernizante geral
francesa de 1789, que somente deste modo se aproximou de Lisboa. No
mesmo movimento do tempo em que Lisboa fugia estrategicamente da
modernidade europeia avançada, para o Brasil.

Assim teve origem o novo espaço político e social do que viria a ser um
país: \emph{em um mais puro e legítimo movimento} \emph{da} \emph{reação
europeia}, um efetivo ato de conservação, com todas as forças
disponíveis e alguma estratégia, de uma Corte absolutista católica em
risco. Um efetivo país reacionário em relação aos tempos modernos que
sopravam com as novas revoluções liberais, finalmente constituído,
quatorze anos depois da chegada do Rei, pela mão do jovem Príncipe
herdeiro português em pessoa -- que segundo diziam as ruas da época, já
havia sido enfeitiçado por uma \emph{michela} de São Paulo -- de olho e
tentando intervir em uma crise política que acontecia na cidade
portuguesa do Porto, onde se encontravam os seus reais interesses
europeus, que agora, no segundo quartil do século \versal{XIX}, se casavam com os
seus reais interesses brasileiros. Desde sempre a política no Brasil foi
confusa, e bastante descentrada em relação à vida interior da nação.
Desde sempre ela foi comprometida pelo forte marco colonial, que
constituiu o espaço social do país muito antes dele chegar a ser
concebido, de qualquer modo que fosse.

Fruto de uma fuga, e de um gesto, muito próprios do \emph{ancien régime}
europeu, aberto em plena modernidade do século \versal{XIX}, o Brasil só poderia
constituir"-se como novo e estranho Império tropical, muito instável pela
natureza da aceleração da modernização do próprio século que não
correspondia minimamente às coisas do atraso social local, com seus
homens livres dependentes, em geral improdutivos, em constante busca de
nobilitação de tipo antigo e com a sua vida popular mesclada com a vida
muito mais forte daquilo que era a nossa verdadeira e fixada casta
trabalhadora: os escravos negros brasileiros, sequestrados e importados
aos milhões da África.\\

\section{III}

Mas, também, é preciso lembrar, este é um país que adentrou a
modernidade mundial muito tempo antes de sua marca de fundação como
alguma espécie de nação híbrida, tentativa de invenção de Império
constitucional tropical, escravista desde sempre, em pleno século da
emergência universal da indústria, dos grandes mercados de trabalho e
consumidores cosmopolitas e da forte luta de classes central que então
emergia.

O espaço do que viria a ser o Brasil foi por muito tempo um verdadeiro
espaço de trocas econômicas globais, um espaço sobretudo comercial, de
fato o espaço \emph{transicional} de um mercado produtor de excedentes
coloniais, modalidade originária de capital mercantil, grande fornecedor
de açúcar, de ouro, de diamantes, de algodão, de cachaça, de tabaco, de
café e também, por outro lado, ainda mais espetacular mercado comprador
de uma massa de 5 ou 6 milhões de escravos africanos, fora os mortos, ao
longo de 350 anos -- 42\% dos quais foram caçados, vencidos em guerras e
trazidos ao então novo país na primeira metade do século \versal{XIX}, contra
todos os tratados de suspensão do tráfico até então assinados. Em outra
opinião possível, bem própria do século \versal{XIX}, José de Alencar escreveu um
dia sobre os números gerais da ampla escravidão americana: ``calcula"-se
em cerca de quarenta milhões o algarismo desta vasta importação''\ldots{}

Tratava"-se assim de um verdadeiro e maravilhoso espaço de transações
mundiais, para quem ganhava, em que uma massa de produtos colônias era
trocada por uma massa de homens transformados em mercadoria e em
propriedade, em negociações que se davam em um universo geográfico
transcontinental expandido, tão fortemente atlântico e africano quanto
interior -- demonstraram bem Luiz Felipe de Alencastro e Alberto da
Costa e Silva -- e que se tornou unidade geopolítica e comunidade
institucional e imaginada autônoma tardiamente, em 1822\footnote{``A
  primeira escravidão teve caráter colonial, com fundamentos legais e
  socioeconômicos derivados do Velho Mundo, principalmente do
  Mediterrâneo. (\ldots{}) A primeira escravidão envolveu duas novas
  instituições: o tráfico oceânico de escravos e a \emph{plantation}
  escravista americana. (\ldots{}) Nunca antes na história houvera um império
  marítimo como esse, que comprava trabalhadores forçados em um
  continente para organizá"-los e explorá"-los em outro, com o objetivo de
  produzir artigos de consumo popular em um terceiro.'' Robin Blackburn,
  ``Por que a segunda escravidão?'', em \emph{Escravidão e capitalismo
  histórico no século \versal{XIX},} Rafael Marquese e Ricardo Salles, orgs., Rio
  de Janeiro: Civilização Brasileira, 2016, p.14.}. Tardiamente, ou
\emph{atrasado}, como se convencionou dizer, se pensarmos como baliza a
emergência e a existência de sociedades modernas integradas e autônomas,
de direitos civis universais garantidos e de dinâmica democrática geral.
Um espaço geopolítico bem fundado na força da mão de obra internacional
escrava, de fato africana -- África que, segundo um prócer das origens
da nação, sonhando com o mundo que discutiremos mesmo aqui, \emph{um}
\emph{dia civilizaria o Brasil} -- e que realizava seus valores de troca
finais em Portugal e Inglaterra.

Um país constituído assim, deste modo sério e sobre este leito real,
que, cumprindo tratados internacionais assinados em 1826 com a
Inglaterra, declarou a ilegalidade definitiva do tráfico atlântico de
escravos em 1831 e que, ainda uma vez, declarou, novamente, a mesma
ilegalidade definitiva do tráfico atlântico de escravos em 1850\ldots{}

Todavia, com característicos traços de compaixão, o Brasil imperial
chegou a promulgar uma lei compensatória, que dizia que crianças
nascidas de mulheres escravas brasileiras poderiam ser livres após os 8
anos de vida, se os senhores assim o desejassem. Isto em 1871.
Evidentemente, a liberação do cativeiro ocorreria se os donos daquelas
crianças fossem corretamente ressarcidos pelo \emph{capital} perdido no
ato legal humanitário; do contrário, previdentemente, a libertação
ocorreria \emph{aos 21 anos de vida do escravo nascido livre}, dando a
chance justa dos senhores serem pagos com o próprio trabalho expropriado
do \emph{virtual} homem livre brasileiro. A este ato de civilização,
muito significativo, chamou"-se de modo inventivo, poético e amplo, como
também nos é próprio em casos assim, de \emph{lei do ventre livre}. Bem
como, com ainda mais acentuado amor pela humanidade, outra lei, a
\emph{do sexagenário}, liberava de trabalhos forçados e da propriedade
senhoril escravos \emph{maiores de 65 anos}, no avançado da hora do ano
de 1885. Não sem antes, como sempre, discutir a indenização dos
senhores, injustamente expropriados pela imprevista nova ordem
filantrópica que \emph{chegava ao país desde fora}.

Todo este movimento positivo, claramente afirmativo de direitos humanos
e populares e de desenvolvimento social brasileiro, se deu antes de --
por respeito acumulado e aprendido frente às canhoneiras inglesas que
andavam à solta pela costa do país e que, em busca de navios negreiros
há muito ilegais, botaram a baixo a fortaleza de Paranaguá em 1850 -- o
país finalmente chegar a abolir a sua escravidão formal, por fim, no
adiantado da hora do ano de 1888. O verdadeiro ano do nascimento de
Fernando Pessoa, da fundação da \versal{IBM} e do rebaixamento do preço
internacional do lingote de alumínio de \versal{US}\$ 4,96 para \versal{US}\$ 0,78, pelo
novo método químico inventado pelo americano Charles Martin Hall.

\section{IV}

Deste modo, tentando dar conta da cultura de tal delírio histórico
concreto, dito um país, e retornando ainda ao ano de 1822, que, dizem,
foi o da marca de nossas origens políticas e institucionais nacionais,
podemos recordar um pouco como um certo escritor que então já se sonhava
brasileiro, Januário da Cunha Barbosa, \emph{presbítero nascido na
cidade do Rio de Janeiro em 1780} -- ou, ainda, se quisermos, e para
cada hora histórica, poeta, jornalista, historiador, orador sacro,
cônego da Capela Real de Pedro \versal{I}, político e fundador do Instituto
Histórico e Geográfico Brasileiro -- nos legou a profecia de nossas
maiores grandezas e de nossas glórias especiais em seu justamente
esquecido pequeno poema áulico \emph{Nicteroy}, mais uma última
tentativa retardatária de épico brasileiro, para que não esqueçamos e
nem duvidemos de algumas das dimensões mais originais de nós
mesmos.\footnote{Sobre a ideia de uma \emph{história delírio} do Brasil
  -- a contrapelo de nossa história com parâmetros oficiais ocidentais
  que nunca ganham integridade no mundo da vida brasileira --, e uma
  intuição estético política de nossos modernistas -- de Oswald de
  Andrade e o \emph{Macunaíma} a \emph{Terra em transe} e o momento
  alegórico do tropicalismo de 1968 -- ver Jean"-Christophe Goddard,
  \emph{Brazuca, negão e sebento}, São Paulo: \versal{N}"-1, 2017.}

Trata"-se de um raro testemunho literário do que mais ou menos ia na
cabeça dos novíssimos brasileiros emancipados, portugueses até ontem, ao
menos dos livres e com grandes aspirações e algum futuro no novo espaço
político nacional que se instituía. O poema, se podemos chegar a dizer
assim, dá corpo curioso à ordem da cultura oficial que já celebrava
todas as convenções do \emph{novíssimo}, mas igualmente \emph{já bem
velho}, poder nacional, em conjunto com uma acumulação fantástica de
referências mitológicas, que, do ponto de vista do todo, pouco ou nada
falavam do tempo histórico existente, mas que, ainda assim, ainda
falavam algo para um certo tipo de sujeito em um pais novo/velho como
era o Brasil. Antonio Candido, com algum esforço e mínima gota de
condescendência, julgou o poema como sendo apenas \emph{péssimo},
\emph{um esforço ao mesmo tempo ridículo e comovedor}.

Seu ``Argumento'' de abertura, ao meu ver o que havia de melhor no
escrito, dizia o seguinte:

``Niteroy, filho do Gigante Minas e de Atlântida, era nascido de poucos
dias, quando seu pai foi morto por Marte na guerra dos Gigantes. Netuno,
tocado das lágrimas de Atlântida, o fez criar em terras desconhecidas,
que depois se chamarão Brasil. Niteroy, crescendo, tentou vingar a morte
de seu pai, renovando a guerra. Com este fim, com muita antecipação e
segredo, juntou pedras sobre pedras, que ainda formam a serra chamada
dos Órgãos. Júpiter, conhecendo seus intentos, o matou com um raio,
quando ele estava sob aquele cúmulo de penedos, meditando na empresa. O
seu corpo tombou sobre um vale, que hoje é a baía de seu nome, porque
Netuno o converteu em mar, cedendo às suplicas de Atlântida, e marcando
a sua separação de Oceano, com o grande rochedo, que fora arrancado por
Niteroy para ser arremessado à Marte, e que com ele desabara da serra.
Glauco, para consolar Atlântida, profetiza a glória do Brasil, e com
especialidade a do lugar, em que seu filho fora convertido em mar,
principiando pela descoberta de Pedro Álvares Cabral, até o nascimento
da sereníssima Senhora Princesa da Beira, enlaçados os Troncos de
Bragança e d'Austria. Finda a Profecia, Atlântida é reconhecida ninfa
marítima.''

Belas origens, afirmativas de nossa veneranda tradição desmiolada. De
uma perspectiva que exigisse algum vínculo, algum traço de
sensibilidade, para a pulsação da história, e da história na vida local,
o texto era nitidamente feito de passado que não passava e de
convencionalismo pesado, de imaginação codificada. Beija mão satisfeito
e autoconsciente do poder que importa, meio samba do crioulo doido
classicizante de uma história que não podia ter referência na vida
social que de fato a constituia, condescendência com formas mortas, sem
mínimo ou nenhum sentido de realidade contingente, congelado mesmo nas
convenções, hipérbole ilimitada e satisfeita, ridículo inconsciente do
papelão à luz de alguma consciência moderna possível, mas inexistente no
caso, que deveria se sustentar exclusivamente em uma real cultura de
Corte, existente ao redor, de fato laudatória. Na consciência portuguesa
brasileira conservadora de início dos anos 1800, no Brasil, o movimento
já datado de 300 anos de repactuação da cultura europeia cristã com
elementos alegóricos metafóricos clássicos greco"-romanos, que deu o tom
da modernidade renovadora \emph{dos anos de 1500}, pré contrarreforma
ibérica, ainda era o modo de explicitar o vínculo e a natureza de uma
realidade outra, uma mímese arruinada, alegoria dependente, da própria
ordem de deformação da consciência histórica.\footnote{Ver Malcolm Bull,
  \emph{The mirror of the gods, classical mythology in renaissance art},
  Londres: Peguin Books, 2005.}

Algo do forte poder conservador das ideias brasileiras, quando políticas
e estéticas, e seu \emph{kitsch} especial, atraso de mentalidade que
denúncia o postiço da modernização, determinado por esta própria posição
no sentido das coisas, pacto de arcaísmo e adulação, já estava expresso
com precisão na anti"-forma daquele mínimo poema pátrio original,
\emph{ridículo e comovedor}.

De fato, o recurso alegórico ao passado, historicamente anódino,
evitativo de toda tenção e tentação histórica e social ao redor, seria
uma tópica constante, mesmo que melhorada imaginativamente, do
conservadorismo brasileiro. Ele poderá ser visto por ainda muito tempo
na vida cultural do país. Como estava presente, por exemplo, na
estratégia pacificadora e excludente da realidade social negra
brasileira do romance fabular, mas alegoria nacional de origem, \emph{O
guarani}, o primeiro fato literário social notável do país -- cujo
comentário de Angela Alonso a respeito do dia -- político -- da estreia
da ópera de Carlos Gomes sobre o livro de Alencar, em 1870, é perfeito:
``Em cena, o mito de origem da nacionalidade, o enlace da moça de
ascendência europeia com um indígena tão nobre quando um aristocrata. No
palco, aventureiros espanhóis e portugueses, fidalgos e caciques,
Cecília e Peri, ninguém da cor de Rebouças {[}\emph{o engenheiro
abolicionista presente na plateia}{]}.// O tempo da ópera era o passado
remoto, o início da colonização, antes de o país se partir em senhores e
escravos, antes de se generalizar a instituição mãe de todas as outras,
a engrenagem do Império -- a escravidão, fundamental e tácita. Os
negros, ausentes do enredo, estavam no camarim, montaram o cenário,
dirigiram cabriolés, engomaram as roupas, lustraram os sapatos,
alimentaram cada uma das bocas da plateia com seus quitutes e seu leite.
\emph{O guarani} nada dizia deles.''\footnote{E conclui, com a busca
  político simbólica da integridade da cultura conservadora do Brasil
  Império da exclusão civil de massas de trabalhadores negros, e também
  da exclusão de sua representação imaginativa fundamental, do qual José
  de Alencar foi o principal artista: ``Naquele 1870, maestro,
  imperador, partidos, cortesãos celebravam a monarquia, sua obra de
  civilização, sua nação inventada, mas logo encenariam um enredo mais
  incerto.'' Angela Alonso, \emph{Flores, votos e balas -- o movimento
  abolicionista brasileiro ((1868-88)}, São Paulo: Companhia das Letras,
  2015, p. 48, 49.}

Do mesmo modo que o tempo das representações do novo país era o passado,
envolto em matéria fetichisante de uma nobreza em dissolução, mas sempre
reafirmada -- daí a \emph{alucinose} muito típica destas obras -- também
a própria subjetivação, a formação e o compromisso dos homens que o
constituíam estavam concretamente enraizadas em muitas dimensões da vida
colonial portuguesa brasileira -- o seu jeito de comer, de dormir e de
ganhar dinheiro, dizia Mário de Andrade -- de modo que a marca política
abstrata da origem nacional, a nossa \emph{independência,} relativamente
pouco significava como \emph{diferença} frente ao sistema de repetições
da verdadeira formação do \emph{sujeito colonial brasileiro}. Ainda mais
considerando"-se a plena manutenção da sociedade escravista como base
viva, legítima para aqueles homens, do novo país. Januário da Cunha
Barbosa era um daqueles novos velhos sujeitos, portugueses brasileiros
ou vice"-versa, portadores em si mesmos de um contínuo histórico de
mentalidade de longa duração, muito estudado por Sérgio Buarque de
Holanda, sujeitos constituídos na ordem do poder concentracionário, do
absolutismo precoce português, como dizia Raimundo Faoro -- um mundo
católico, imperialista colonial e mercantil escravista, sem traço de
democracia ou dialética de poder público. O constrangimento do ato
simbólico plenamente moderno devia ser grande. Daí o congelamento geral
das formas, rigoroso e pueril a um tempo, quando ele tenta dizer do novo
país, que para ele é mais ou menos o mesmo de seu passado. Cunha Barbosa
era também mais ou menos o mesmo tipo humano que podemos intuir por um
segundo na biografia política de um homem do poder do primeiro reinado,
como aquele que, em 1831, reapresenta o projeto de lei de proibição do
tráfico de escravos no Brasil, já acertado em 1826 pela forte pressão
inglesa -- e só tornado efetivo depois da terceira tentativa, com a lei
Eusébio de Queiróz, em 1850\ldots{} --:

``O projeto de lei foi apresentado no Senado pelo marquês de Barbacena.
Filho de Coronel e neto de contratador de diamantes, Felisberto Caldeira
Brant Pontes (1772-1841) recebeu as primeiras instruções em Mariana, fez
estudos complementares no Colégio dos Nobres, em Lisboa, e frequentou a
Academia de Marinha, também em Portugal. Foi então nomeado major do
Estado"-Maior e ajudante de ordens do governador de Angola, onde
acompanhou de perto o tráfico negreiro por alguns anos. Depois seguiu
para Salvador e aí se casou com a filha de um comerciante local de
grosso trato. Caldeira Brant passou a entestar as transações do sogro,
relacionando"-se com `diferentes praças do mundo' -- não é possível
asseverar que ele mercava em escravos africanos, mas a hipótese soa
plausível. Declarada a independência, recebeu de D. Pedro \versal{I} proposta
para ocupar a pasta da Guerra e da Marinha e, suposto declinasse o
convite, continuou próximo do imperador, apoiando a constituição
outorgada em 1824. Em viagem à Europa, cumpriu a tarefa de negociar a
Independência brasileira com portugueses e com o todo"-poderoso do
Foreign Office, George Canning. O tratado daí resultante foi trazido ao
Rio de Janeiro por Charles Stuart, com outro sobre o fim do tráfico
negreiro. Proprietário de escravos, experiente na África, negociante
avultado e envolvido no reconhecimento diplomático do Império (com todas
as implicações para as permutas afro"-brasileiras), Barbacena era um dos
políticos mais abalizados para estimar os efeitos do tratado de 1826,
bem como os de uma lei complementar em 1831. E, ainda assim, ele não
hesitou.''

De fato, no Brasil, o sujeito do poder colonial, escravocrata, está na
base do sujeito do poder nacional. Tais sujeitos, distantes na aparência
e na marca significante mínima, mas que não fazia registro de diferença
-- como mais tarde Machado de Assis demonstrou com rigor e ironia sem
fim, nas continuidades psíquicas, ideológicas e de sociabilidade que
simplesmente duravam, na passagem do Império para a República, em
\emph{Esaú e Jacó} -- eram o mesmo, em tudo aquilo que de fato importa
para a vida popular e a virtualidade democrática de um país
moderno.\footnote{Tâmis Parron, \emph{A política da escravidão no
  império do Brasil, 1826,1865}, , Rio de Janeiro: Civilização
  Brasileira, 2011, p.85.}

\section{V}

É um certo anacronismo exigirmos de Januário da Cunha Barbosa um
entendimento da modernidade que não era o seu, nem o dos homens de sua
classe e condição em seu novo país. No entanto um anacronismo, um desejo
de luz moderna sobre a experiência e a dinâmica brasileira que um outro
autor, contemporâneo de Cunha Barbosa, não por acaso, como veremos, foi
perfeitamente capaz de enunciar.

O viajante alemão Carl Schlichthorst, ao pensar pela primeira vez os
horizontes possíveis para a literatura no Brasil das origens e com a
experiência de ter vivido aqui por dois anos, nos mesmos anos de 1820 do
\emph{Nicteroy} de Barbosa, simplesmente escreveu, a respeito da
possível representação literária do país, fazendo precoce tábula rasa da
consciência convencional dos brasileiros, com convicção e facilidade:
``A mitologia grega, na maior parte baseada em fenômenos da natureza,
faria triste papel sobre o céu tropical. Poderá a Aurora servir para
abrir com seus dedos cor de rosa o reposteiro de um dia, cujo
esplendoroso colorido faria empalidecer o próprio Apolo? Ninfas e faunos
serão por acaso habitantes adequados às florestas virgens e eternamente
verdes, em cujo seio inviolado se escondem mais maravilhas do que as
poderia criar a mais viva fantasia? As primeiras tentativas de musa
brasileira fazem já supor que tomará um voo mais original e que o Brasil
conservará sua independência, quer poética, quer polítca.''\footnote{Capítulo
  ensaio ``Literatura Brasileira'', de \emph{O Rio de Janeiro como é},
  \emph{1824-1826 --} \emph{uma vez e nunca mais,} de C. Schlichthorst,
  Rio de Janeiro: Editora Getulio Costa, p. 157.} O espírito de
independência e liberdade de movimentos do viajante, e seu fundo
iluminista popular, projetava na riqueza natural, na novidade e na
modernidade compulsória da nova nação um princípio de inquietação e
invenção que, durante muito tempo, nosso real conservadorismo, que tinha
outro fundamento, não permitiria confirmar.

O que importa é que, de algum modo, no épicozinho retardatário de 1822,
uma quase alegoria carnavalesca superficial a contrapelo do desejo
elevado de seu autor, sem referências na modernidade, já estava bem
indicada, e talvez até mesmo já claramente \emph{formada}, com grande
antecipação histórica mas com estrutura social de razões clara que
lesava toda possibilidade de forma, a equação simbólica difícil daquela
nossa famosa e real \emph{aristocracia do nada} brasileira, a qual Paulo
Emílio Salles Gomes, discípulo sério de Oswald de Andrade, combateu a
grande irresponsabilidade nos anos duros de 1930 a 1970. Lá também já
estava, lépida e pronta para produzir o próprio espaço nacional que lhe
correspondia, a consciência de \emph{ocupante}, também nos termos
críticos de Paulo Emílio na década de ditadura de 1970, estrangeirada no
nada de um passado ou de um futuro apenas inexistentes naqueles termos
do que seria o espaço nacional real, estrutura de sentido e subjetivação
muito própria de parte significativa das elites nacionais, as mesmas que
Oswald de Andrade ridicularizaria sem piedade, pelos mesmos motivos, em
suas obras em prosa de vanguarda dos anos de 1920.\footnote{``Rejeitando
  uma mediocridade, com a qual possui vínculos profundos, em favor de
  uma qualidade importada das metrópoles com as quais tem pouco o que
  ver, esse público exala uma passividade que é a própria negação da
  independência a que aspira. (\ldots{}) A esterilidade do conforto
  intelectual e artístico que o filme estrangeiro prodiga faz da parcela
  de público que nos interessa uma aristocracia do nada, uma entidade em
  suma muito mais subdesenvolvida do que o cinema brasileiro que
  deserdou.'' Paulo Emílio Salles Gomes, \emph{Cinema: trajetória no
  subdesenvolvimento}, Rio de Janeiro: Paz e Terra, 1980, pág. 101.}

Creio tratar"-se de fato, por exemplo, também da mesma posição simbólica
bastante questionável, de longa duração, a respeito da qual Antonio
Candido, amigo íntimo de Paulo Emílio e crítico original de Oswald,
comentando romance com pendor \emph{filosofante, intelectual e
universal} de José Geraldo Vieira, \emph{A quadragésima porta}, em um
momento de grande força crítica e consciência, ainda na década de 1940,
escreveu, batendo com justeza e com urgência histórica: ``que nunca mais
sejam possíveis no Brasil obras semelhantes e classes que as tornem
viáveis e significativas'', anotando ser um espírito de classe
brasileira que passou simplesmente incólume aos efeitos da Revolução de
1930.\footnote{Antonio Candido, ``O Romance da nostalgia burguesa'', em
  \emph{Brigada ligeira}, São Paulo: Livraria Martins Editora, 1945.}

Era a posição da dura consciência crítica moderna nacional a respeito do
extrato social, com sua simbólica irresponsável e não íntegra, na qual
simplesmente ``o que impressiona é o seu desligamento total do Brasil,
de nossos problemas, de nossa maneira de ver os problemas'', nos termos
de Candido, quando expressam o seu mundo imaginário meio mágico e meio
morto, simplesmente deslocado da vida produtiva e histórica e de
qualquer ordem de compromisso social, popular, crítico ou mesmo
livremente estético. Um famoso espaço elevado, mortificado e
distanciado, de tipo \emph{flor de estufa}, nos termos de Sérgio Buarque
de Holanda, dos \emph{ricos entre si} do Brasil, nos termos de Machado
de Assis, o mesmo espaço de razões que, em uma batalha titânica e
secular, nosso modernismo e nossa cultura crítica moderna local
confrontou sistematicamente ao longo de todo o século \versal{XX}. ``Ainda quando
se punham a legiferar ou a cuidar de organização e coisas práticas, os
nossos homens de ideias eram, em geral, puros homens de palavras e
livros; não saiam de si mesmos, de seus sonhos e imaginação. Tudo assim
conspirava para a fabricação de uma realidade artificiosa e livresca,
onde nossa vida verdadeira morria asfixiada. (\ldots{}) Era o modo de não nos
rebaixarmos, de não sacrificarmos nossa personalidade no contato de
coisas mesquinhas e desprezíveis.''\footnote{Sérgio Buarque de Holanda,
  no capítulo ``Novos tempos'', de \emph{Raízes do Brasil}, São Paulo:
  Companhia das Letras, 2006, pág. 179.}

Ao mesmo tempo, espaço social e simbólico positivamente alinhado com o
poder, afirmativo satisfeito da própria submissão integral a uma
soberania qualquer, como também anotou Sérgio Buarque, um espaço
cultural ideológico que é uma das nossas formações de longa duração, do
\emph{continuum criado por força do escravismo,} como escreveu Alberto
da Costa e Silva. A cultura de \emph{medalhão}, inconsequente,
adulatória, praticamente demenciada do ponto de vista da vida do
conceito, como, de modo satírico e condensado, Machado de Assis a
enunciou e pesquisou muito, em detalhes espantosos, em seus romances de
segunda e de \emph{terceira} fases.

Que formas semelhantes a aquela fossem encontradas pouco mais tarde em
José de Alencar -- no \emph{Guarani}, por exemplo -- e, já no século \versal{XX},
no tradicional pensamento fetichista à direita brasileiro,
particularmente no mórbido \emph{quatrocentrismo} paulista -- claramente
expressas no cinema mofado das elites jogando o gamão de suas próprias
existências anti"-modernas, da Vera Cruz -- e fossem abertamente
denunciadas em uma espécie moderna de construção em abismo do passadismo
e do kitsch nacional por Glauber Rocha, ainda em 1967, em \emph{Terra em
transe --} e também, como efeito de superfície e festivo, em nosso
animado tropicalismo musical -- e que, ainda hoje se manifestem traços
reais desta barbárie arcaizante cultural nas ruas e na nova tentativa de
política de direita no Brasil pósmoderno, agora já misturados a
subjetivação do novo sujeito universal do consumo globalizado, talvez
ainda mais rebaixado, esse grande circuito de repetições do \emph{kitsch
mortificado, e hoje maníaco e determinado industrialmente, brasileiro}
revela a verdadeira força de uma íntegra convicção nacional, de
fundamento.

Desconexão da história como programa e sistema delirante de ideias para
a aproximação do poder, com produção de \emph{non sense} aplicado e como
cultura, autoritária e satisfeita, é uma prática ideológica vernácula
brasileira. É a persistente dimensão simbólica nacional, que revemos
hoje como catástrofe cultural, de \emph{boçalidade do mal.} Pode"-se
reconhece o valor contemporâneo do programa?

Foi exatamente este tipo de mentalidade e de homem que Machado de Assis,
mais uma vez, foi o primeiro a formular e apontar virtualmente o
fundamento kitsch real, o seu mal gosto anti"-moderno, morto vivo, o
compromisso subjetivo com o que não existe, a falsa cultura clássica, a
irresponsabilidade das ideias vazias e a impotência garantida em
sociedade de escravos, na cena \emph{estranha} de abertura de \emph{Dom
Casmurro}, em que Bento Santiago, o Bentinho, nos conta como,
envelhecido e bem congelado no ressentimento da própria vida, buscava
mais ou menos \emph{reconstruir a casa de sua infância} -- a casa da
mamãe Dona Glória que o prometera ao Seminário católico, do qual ele
fora salvo pela consciência aguda das coisas da menina Capitu\footnote{Ver
  a respeito do valor da consciência crítica da criança e da menina na
  literatura realista do século \versal{XIX} brasileiro, Roberto Schwarz,
  \emph{Duas meninas,} São Paulo: Companhia das Letras, 1997.} -- com a
reprodução fantasmática das efígies pintadas no teto da casa original de
figuras da história da Roma clássica, Cezar, Augusto, Nero (!) e
Massinissa (!!). Fetichismo arcaizante, que congela o infantil no velho,
anti"-moderno por excelência, de um passado que já passara, mas que, para
esse modo de ver as coisas, nunca deveria passar.

Passadismo, e cultura morta, impotência social e autoritarismo, são os
\emph{valores} de tal constituição. Verdadeiros elementos da cultura
original de nosso \emph{ancien régime escravocrata} \emph{brasileiro}.
Talvez origem de algo que é o real proto"-pós"-modernismo, autoritário,
brasileiro, hoje em voga.

Também, o ódio histérico atual, que imanta fortemente a capacidade de
pensar de setores da classe média brasileira que adoraram se converter
em massas nas ruas, no sentido estritamente freudiano da ideia, e a sua
destruição política interessada e calculada da história e do sentido
daquilo que foi o governo Lula no país, talvez ainda tenha contato com
aquelas formas não integradas, nada modernas nem responsáveis, que
formavam a origem e o fundo autoritário, automaticamente agregado ao
poder, que antecedeu a tudo no Brasil.\footnote{Sigmund Freud,
  \emph{Psicologia das massas e análise do eu e outros textos}, São
  Paulo: Companhia das Letras, 2012.}

Ainda temos Cunhas Barbosas em meio aos nossos tão radicais quanto
violentos supermodernos \emph{teapartistas intervencionistas} de
manifestações de domingo à tarde na Avenida Paulista, capazes
posteriormente de sustentar tranquilamente o arcaísmo resoluto, mas
muito esperto, de um governo Temer, e uma regressão antimoderna
definitiva como o virtual neofascismo assumido de um Bolsonaro, sempre
beirando o ridículo, super kitsch, mas a favor de todo interesse
particular e força qualquer de capital? Ou, de fato temos coisa nova, e
ainda pior, no que com muita condescendência tem se chamado hoje de nova
direita do Brasil? Qual é a nova ordem de poder e de cultura para a
velha pulsão da agregação automática brasileira, e seu mais tradicional,
profundo, verdadeiro besteirol cultural nacional?

Um dos pontos importantes da vida simbólica brasileira contemporânea --
e seu último \emph{transe}, em uma sucessão secular de crises -- é saber
o quanto a nova ideologia avançada da modernidade ultraliberal do
capitalismo globalizado, de grandes fundos internacionais que reduzem em
muito, ou totalmente, a margem de manobra das nações para a própria vida
social local, simplesmente reencontra a nossa velha disposição para a
auto"-elevação descomprometida da elite, produzida no mesmo movimento do
desprezo pela vida pobre e popular.

Elite brasileira que simplesmente se desresponsabiliza, e isto faz parte
do processo ideológico, dos próprios resultados de sua política, como de
fato se desresponsabilizou da parte violenta da crise econômica de hoje
que lhe coube -- economia parada, 12,5 milhões de desempregados durante
todo período Temer -- criada pela própria política extremada da
construção de um \emph{impeachment} de mais de ano e meio, artificial e
violento, em um país já em crise.

Precisamos saber se, mais uma vez, como em toda nossa tradição moderna
marcada com este explícito travo de atraso antimoderno, vamos avançar na
progressão da riqueza sem avançarmos nos direitos à integração social,
como simplesmente o fizemos no século do Império escravocrata, durante
toda a República Velha oligárquica e na brutal ditadura civil"-militar de
capitalismo selvagem brasileiro de 1964-84.

E o dito de Antonio Candido, de 1945, continua melancolicamente suspenso
no ar da história dos movimentos de grande violência, para tráz, que o
Brasil sempre conheceu: ``que nunca mais sejam possíveis no Brasil obras
semelhantes e classes que as tornem viáveis e significativas.''

\section{VI}

Talvez as contradições \emph{sem sujeito}, desta célebre terra dourada,
onde por amor e por esperança a terra desce, expliquem algumas poderosas
\emph{visões do paraíso}, em outra direção, de Carl Schlichthorst,
apresentadas naquele que possivelmente seja o melhor relato da vida
social brasileira existente ao tempo da nossa \emph{independência}.
Visões que serão evocadas aqui como uma metáfora fundamental; a da
\emph{fundação dos vértices simbólicos}, com base na realidade social,
em uma \emph{arqueologia de nossas possibilidades}, dos horizontes
culturais e críticos da nação, por dentro da própria experiência do
Brasil.

\emph{O Rio de Janeiro como ele é} foi escrito por um militar alemão
popular cheio de espírito, mesmo que de superfície, e pleno de
curiosidade, ainda que determinada pelo interesse próprio, para dar
notícia ampla de sua experiência, entre 1824 e 1826, no novíssimo país
americano que despontava no horizonte das nações modernas. Carl
Schlichthorst foi um dos mercenários, vagabundos e ex"-presidiários
europeus seduzidos e embarcados rumo ao Brasil pelo duvidoso agente
imperial von Schaffer -- amigo íntimo de nossa Imperatriz austríaca -
cujas mentiras visando atrair a escória alemã para formar as tropas
estrangeiras do novo Império lhe valeram a alcunha de \emph{carniceiro
de gente.}

E, de fato, numa das muitas surpresas do relato do viajante sobre a sua
experiência com o novo país -- publicado em 1829 em Hannover através da
subscrição de amigos -- ele nos conta a história de como o navio que o
trouxera, bem como a outros futuros militares do Império recém fundado,
se aproximava constantemente, por vezes, de se converter em \emph{uma}
\emph{espécie} \emph{de} \emph{navio de escravos --} uma fantasmagoria
dos mares mundiais que moveu toda a expansão do capitalismo mercantil
global desde o século \versal{XVI} até o final do \versal{XIX} -- pelos modos
contraditórios totalitários dos contratantes e para a revolta de muitos
dos \emph{contratados} para a desconhecida vida nova no Brasil\emph{\ldots{}}
Mary Karasch lembra, exatamente, que a experiência da importação de
homens livres, mercenários, irlandeses e alemães, para o exército
brasileiro nos anos das origens nacionais foi encerrada com importantes
sublevações das tropas estrangeiras, uma revolta armada de três dias no
Rio de Janeiro\ldots{}, que se sentiam ao final da história, no fim daqueles
mesmos anos de 1820, exatamente \emph{tratadas como escravos pelo
oficialato do Império do Brasil}\ldots{}\footnote{Mary C. Karasch, \emph{A
  vida dos escravos no Rio de Janeiro}, São Paulo: Companhia das Letras,
  2000, p. 126.} Afinal, como em 1854 ainda escrevia Manuel Antônio de
Almeida, em seu \emph{Memórias de um sargento de milícias}, ``ser
soldado era naquele tempo, e ainda é hoje talvez, a pior coisa que pode
suceder a um homem.'' A não ser, no caso brasileiro, \emph{ser escravo.}

Schlichthorst também foi aquele interessante escritor que, conhecendo o
país desde as ruas de sua Corte, comentou as primeiras impressões e
hipóteses sobre a natureza e o futuro da literatura brasileira que ao
seu tempo estava toda por ser inventada. Seu pequeno tratado de teoria
literária do Brasil, uma parte das suas memórias cariocas, era um
trabalho retrospectivo de história da literatura colonial local e um
estudo especulativo sobre a ordem simbólica possível do novo país,
também, em parte, por ser criada, em um conjunto de questões vivas a
respeito do qual Antonio Candido escreveu, com um mínimo grão de
admiração e um tanto de respeito: ``alguns viajantes estrangeiros se
referem ao passado literário do Brasil ou auguram o desenvolvimento de
uma literatura original, quando não fazem as duas coisas ao mesmo
tempo.// Ninguém no primeiro sentido foi mais minucioso, interessado e
simpático do que o alemão Schlichthorst, oficial nos corpos estrangeiros
no Exército Imperial de 1824 a 1826, que publicou em 1829 um dos livros
mais interessantes sobre o país (\ldots{})\emph{//} Mesmo que os brasileiros
não tenham tomado conhecimento da sua obra, escrita em alemão e a que
não há referências no tempo, ela representa bem claramente o que nos
interessa verificar: a noção da existência de uma continuidade literária
e a formulação de princípios que deveriam caracterizar as novas
tentativas literárias.''\footnote{\emph{Formação da literatura
  brasileira}, 1º. Volume, São Paulo: Martins Fontes, 1959, pág.302.}
Diferentemente das hipóteses ideológicas abstratas então em voga, não
por acaso, sobre como narrar e inventar o Brasil e seu mundo, havia por
parte do viajante e soldado alemão um interesse verdadeiro sobre as
origens e o destino da literatura possível no novo país americano, com
real intuição de um movimento histórico próprio, de um escritor que
também se dedicou a escrever o Brasil ao seu próprio modo.

E, de fato, nas interessantíssimas memórias brasileiras de Schlichthorst
lemos e ficamos sabendo, em traço límpido e preciso, que, para a honra
do novo país, na sua grande Corte, a cidade do Rio de Janeiro:

``Os traficantes de escravos são considerados os negociantes mais ricos
da cidade. Habitam quase exclusivamente as ruas do Valongo, do Aljube e
algumas outras nas proximidades do porto. Muitas de suas casas, que
podem ser considerados verdadeiros palácios, tem a mesma disposição no
andar térreo: largo vestíbulo dando para pequenos pátios, onde nada mais
se vê além de bancos baixinhos. São o chamado armazém de depósitos de
escravos, geralmente muito limpo, de chão lavado e varrido várias vezes
por dia. A fresca brisa do mar sopra por toda parte, de maneira que,
mesmo quando cheio de negros, pouco se sente o mau cheiro que
caracteriza as cadeias e casas de detenção da Europa.''\footnote{A boa
  tradução de Schlichthorst é de Emmy Dodt e Gustavo Barroso, da edição
  Editora Getulio Costa, sem data, mas da década de 1940.}

É significativo o movimento de tolerância e entendimento que move o
mercenário prussiano, que parecia, com grande ambivalência, buscar se
abrasileirar na medida do que lhe era possível, quando ele descreve o
catálogo da vida civil sob o regime da escravidão brasileira dos anos 20
do século \versal{XIX} -- de fato o momento literário mais próximo da real
civilização portuguesa já instalada por aqui, herdada como mentalidade
original, modo de ver e de agir que -- na radical, inventiva e
psicanalítica hipótese glauberiana de \emph{Terra em transe} --
\emph{tem longa duração e costuma retornar}, particularmente em tempos
de crise e de ruptura, como forma específica de resistência nacional.

O texto do alemão une, de modo bastante original, o armazém de escravos,
a casa senhoril ampla, luxuosa e asseada ao modo tropical e a brisa
fresca da grande cidade portuária, que beija e balança uma vida que
quase chegamos a sentir, uma vida que deseja ser estável e digna, de
modo contínuo e contíguo à sua própria ignomínia. Barbárie e civilização
tropical, violência consentida generalizada e privilégios fantásticos
dos senhores, luxo e escravidão, já andavam bem unidos, e o estilo
límpido e realista do estrangeiro acentua a integridade, já quase
estética, de tal mundo fortemente duplicado, mas de fato uno. Um mundo
violento ao extremo frente à sua mão de obra produtiva, mas civilizado.

Curiosamente já parece haver muita bossa brasileira nesta literatura de
nossos primeiríssimos tempos, e o esforço em civilizar a vida
escravocrata brasileira, e ser \emph{civilizado por ela,} faz balanço,
leve, rápido, mas preciso, como o vento, com a crítica atualizada aos
momentos bárbaros de uma Europa que, surpreendentemente, do ponto de
vista do projeto ideológico periférico, e de seu imigrante pobre, já não
se apresentava exatamente como modelo. O escritor não se aterroriza
diante da escravidão brasileira, e não a descreve em registro de
inominável e irrepresentável barbárie. As cadeias da Europa --
exatamente as mesmas estudadas por Foucault com enormes consequências
para o entendimento do desenvolvimento do poder na modernidade -
\emph{cheiravam pior} do que nossos armazéns de escravos, dizia um pobre
diabo em trânsito global, que provavelmente as conheceu. Será de fato em
geral leve, quase descomprometida, a visão de Schlichthorst da cultura
brasileira dos primeiros tempos, para ele um mundo especialmente
gracioso e erótico de modo particular, vital e jovem, desde que se
aceite como parte de tal encantamento, como vemos na passagem do sobrado
armazém dos comerciantes cariocas, a naturalidade da escravidão
onipresente. Ele fala de um mundo, e de um modo, que lembra muito bem os
mexericos e a excitação superficial das jovens personagens de José de
Alencar, de trinta ou quarenta anos depois, em seus romances urbanos ou
até mesmo naqueles passados em venerandas fazendas imperiais, onde, de
fato, diferentemente do alemão, apenas os escravos que se aproximavam da
cozinha \emph{existiam}¸ minimamente, para aquela escritura.

Assim, Schlichthorst, pode ainda ir mais longe, e ensinar que o Brasil e
sua ordem escravista era um caso concreto e objetivo de civilização que,
se parecia desvio e discutível, e efetivamente o era, tinha dimensão
positiva e prática própria, vida que se impunha com coerência e que
podia ser rapidamente reconhecida e \emph{utilizada} ou, melhor ainda,
nos termos de uma psicanálise contemporânea ainda com algum sotaque
francês, \emph{gozada}:

``Entre os muitos milhares de escravas jovens que perambulam pelas ruas
do Rio de Janeiro vendendo flores e frutas, talvez não se encontre uma
só que recuse um convite para isso {[}\emph{relações sexuais}{]}. Muitos
europeus sentem aversão natural pelas raparigas pretas, mas, desde que a
vençam, passam a gostar delas. Demais, há umas de feições tão lindas, de
estrutura de membros tão esplêndida e de tanta frescura que se torna
difícil resistir à tentação de possuir todos estes encantos por alguns
vinténs.''

``Para dar ao bondoso leitor uma ideia aproximada da confortável vida
meridional, tentarei fazer o resumo de um dia como, sem grandes
despesas, os estrangeiros desocupados podem usufruir no Rio de Janeiro.
Muitos acharão esta vida fastidiosa, mas são os que nunca experimentaram
a deliciosa plenitude do \emph{dolce far"-niente.} Os raios do sol
nascente entram no quarto pelos batentes abertos do balcão. Eu, como
verdadeiro filho do Norte, que nunca fui amigo de acordar cedo, viro"-me
mais de uma vez na cama. Mas a fantasia pinta à minha inata preguiça os
encantos de uma bela manhã, com as mais sedutoras cores, o bom senso a
auxilia com algumas razões higiênicas e, assim, ela é vencida mais
depressa e mais facilmente do que supunha.''

De modo significativo, os traços \emph{brascubianos} deste discurso
original sobre o Brasil já são bem acentuados -- e aqui, como também li
no jovem José Alencar de \emph{Ao correr da pena}, de 1854, o volúvel
personagem Brás Cubas de Machado de Assis é o próprio sujeito da
enunciação literária que, além de \emph{narrador}, responde ao que fala
\emph{como um} \emph{eu}, instâncias que gozam o direito contraventor
real \emph{de nada fazer}, \emph{da preguiça senhoril}, ao mesmo tempo
que de algum modo a notam como picante desvio moderno local, próprio de
uma economia na qual quem trabalha são os escravos reais -- e cada vez
mais podemos compreender o quanto Machado de Assis apreendeu \emph{em
profundidade histórica}, em conjunto com a poderosa e extensa novidade
da sua linguagem, o espírito e a formação de nossa ambígua elite
imperial, e da nação.\footnote{Ver a respeito o estudo da relação entre
  Alencar e Machado, que apresentei em \emph{Ensaio, fragmento}, Editora
  34, 2014.} De fato, todos que se aproximavam do Império politicamente
insólito, mas socialmente radical, do Brasil do século \versal{XIX}, e que
tentavam enunciá"-lo com correção realista, se aproximavam
espontaneamente do vértice da volubilidade, social, política e simbólica
brasileira, em um fenômeno de força de imposição de uma realidade
histórica sobre seus sujeitos e enunciadores muito notável. Todos que
sentiam compromisso real com o problema Brasil, e o memorialista alemão
talvez seja o primeiro deles, tendiam a realizar a dissolução e o
balanço ambivalente das formas culturais e do sujeito, adentrando o
estatuto da \emph{forma difícil}, brasileira. Se aproximavam assim, cada
um ao seu modo, de nossos vértices simbólicos \emph{bráscubiano},
\emph{macunaímico}, \emph{tropicalista, etc}\ldots{}

Muito rapidamente, Schlichthorst aprende que a vida da elite nacional é
a sustentação ideológica, graciosa ou cínica, tanto faz, de um estado de
coisas em que os benefícios e privilégios são expressivos e singulares,
incluindo aí vantagens e práticas simplesmente vedadas ao processo
civilizatório do século do Capital e do mercado europeu, liberal,
puritano já há séculos e, gradualmente, burguês, vitoriano e, por fim,
de mercado de massas. E também de luta de classes. Baseado na
escravaria, e esquecendo o seu destino real -- que segundo o próprio
Schlichthorst, morria então, na cidade, por volta dos trinta e cinco
anos\ldots{} -- um homem livre, senhor brasileiro urbano, podia dedicar"-se
tranquilamente à preguiça, à vida levada na flauta, como passeio, ao uso
erótico do corpo da escrava, se acaso o desejasse, e à graça distraída
de uma civilização irresponsável, mas feliz. \emph{À deliciosa plenitude
de nada fazer}, em pleno século \versal{XIX}. De fato, tudo indica, o Brasil já
constituía fortemente então um estatuto singular de biopolítica, único
no processo da modernidade ocidental: eram os corpos, os tempos e os
usos da vida, tanto de senhores quanto de escravos, que se contrapunham
e que se constituíam, como prática social de violência, em tal processo
histórico que acumulava, de modo não integro, mas plenamente realizado
como civilização, a sua alegria com o seu próprio terror.

Por outro lado, o imigrante soldado autor afirma também, de modo aberto
e claro único no século \versal{XIX} brasileiro, a presença de vida sexual
negociada e amplamente disponível entre os homens livres e as escravas
de ganho que circulavam com certa liberdade pelas ruas do Rio de
Janeiro. Segundo ele, \emph{possivelmente não havia uma mulher negra
escrava de ganho} que, vendendo frutas ou flores, doces ou água,
recusasse trocar alguns vinténs por algum sexo. Assim, a cultura popular
da rua do Rio de Janeiro seria também uma imensa negociação sexual,
certamente de pouco ganho e movimento social para as mulheres negras
envolvidas, mas um campo possível de formas de vida e de sobrevivência.
Como lembra bem Luis Carlos Soares, sobre essa dimensão da cultura
econômica da escravidão urbana brasileira, ``apesar de condenações, até
as portas da abolição da escravatura no Brasil, a prostituição e a
mendicância se constituíram, dentro do leque de possibilidades
fornecidas pela sociedade escravista, nas \emph{formas limites de
obtenção de renda pela classe senhorial}''.\footnote{Luis Carlos Soares,
  \emph{O ``povo de Cam'' na capital do Brasil: a escravidão urbana no
  Rio de Janeiro do Século \versal{XIX},} Rio de Janeiro: 7 letras, 2007, p. 176.}

De fato, Soares nos lembra a história ambivalente, anotada originalmente
pelo viajante Charles Expilly, de uma senhora católica e muito devota
brasileira que, encostando o chicote, desiste de castigar sua cativa
pessoal que fazia sucesso na negociação sexual da cidade do Rio de
Janeiro e que desaparecera por um dia inteiro, já que havia tratado com
a senhora, por bons motivos econômicos, de ter as noites livres para o
trabalho e não os dias\ldots{} quando a moça retorna, ela lhe paga em dobro
por sua falta, com o dinheiro do comércio sexual: 2 mil reis pela noite
e 2 mil reis pela falta do dia\ldots{}\footnote{Idem, p. 181.} Ainda em 1883
Joaquim Nabuco escrevia, em sua crítica radical à escravidão brasileira:
``Os senhores podem empregar escravas na prostituição recebendo os
lucros desse negócio, sem que isso lhes faça perder a propriedade, assim
como o pai pode ser senhor do filho.'' O fenômeno social da prostituição
escrava, e sua corrosão moral ainda aprofundada da consciência de
senhores e de senhoras, era generalizado, e, no entanto, só passa a
representar problema ``ético'' social a partir de década de 1870, com a
expansão gradual da crítica abolicionista. Mesmo na historiografia
moderna ele demorou muito para ser estudado e no clássico trabalho de
recenseamento e configuração da escravaria urbana de Leila Mezan
Algranti, \emph{O feitor ausente}, encontramos apenas esta única
referência à prostituição escrava no Rio de Janeiro, apoiada em Gilberto
Freyre: ``A prostituição, que segundo Gilberto Freyre se desenvolveu no
Rio após a chegada da família real, era também uma fonte de renda para
os senhores que embelezavam suas escravas mais jovens e forçavam"-nas à
prostituição. Já em 1845 elas podiam ser classificadas em
`aristocráticas ou de sobrados, as de sobradinhos ou rótulas, e a
escória refugiadas até em casebres', em geral libertas e
escravas.''\footnote{Leila Mezan Algranti, \emph{O feitor ausente},
  \emph{estudos sobre a escravidão urbana no Rio de Janeiro -1808-1822},
  Petrópolis: Vozes, 1988.}

De resto, o mesmo período histórico viu a prostituição pobre e branca
explodir também nas grandes cidades da experiência burguesa europeia,
onde mulheres sem condição de sustento, muitas vezes envelhecidas,
solitárias e com maridos sem condição de trabalho no mercado, ou
liquidados por guerras, fossem as napoleônicas ou as nacionais, passavam
à prostituição moderna como forma limite de sobrevivência em plena
civilização da nova liberdade liberal, e da realização da riqueza
industrial capitalista. Uma situação muito evidenciada na literatura
romanesca de contato social em ascensão, da \emph{Dama das Camélias}, de
Dumas Filho e \emph{Os miseráveis}, de Victor Hugo, e das cortesãs de
Balzac ao mundo cão da poesia alta baudelairiana, ou à Olympia de Manet,
até às lindas atrizes proustianas\ldots{}\footnote{Ver, entre tantas
  referências possíveis, ``Conhecimento carnal'', em Peter Gay, \emph{A
  experiência burguesa, a educação dos sentidos}, São Paulo: Companhia
  das Letras, 1989, pág. 242, e \emph{Les filles de noce}, \emph{misére
  sexuel e prostituition au \versal{XIX}} siécle, de Alain Corbin, Paris:
  Flamarion, 2015.}

Schlichthorst, que aparentemente não tinha os impedimentos católicos em
nomear coisas sexuais com o próprio nome, também retomava uma questão
real que a lírica seiscentista de Gregório de Matos, de seu ponto de
vista concreto interessado na vida mundana e citadina na Bahia, já havia
assinalado, duzentos anos antes, a respeito da escravidão urbana da
então colônia e sua vida, em um discurso isolado em relação à máquina
ideológica colonial hegemônica -- católica e burocrática --:

\begin{verse}
Mulatinhas da Bahia\\
que toda a noite em bolandas\\
correis ruas, e quitandas\\
sempre em perpétua folia\\
porque andais nesta porfia,\\
com quem de vosso amor zomba?\\
eu logo vos faço tromba,\\
vós não vos dais por achado,\\
eu encruzo o meu rapado,\\
vós dizíeis, arromba, arromba
\end{verse}

Luiz Felipe de Alencastro, comentando estes versos, relembra que no
Brasil ocorreu um fenômeno de formação de vida social, definido pelo
sistema de interesses dos senhores e pela opressão do sistema do
escravismo colonial, que transformou a miscigenação típica das
violências coloniais em \emph{mestiçagem}, configurando um processo
histórico particular, complexo, que daria origem à nossa sociedade
multirracial. E conclui com o que importa: ``o fato deste processo ter
se estratificado e, eventualmente, ter sido ideologizado, e até
sensualizado, não se resolve na ocultação de sua violência intrínseca,
parte consubstancial da sociedade brasileira.''\footnote{Luiz Felipe de
  Alencastro, \emph{O trato dos viventes}, São Paulo: Companhia das
  Letras, 2000, p. 352.}

Também Jean"-Baptiste Debret, o pintor iluminista francês trazido ao
Brasil por Dom João \versal{VI} que viveu por aqui de 1816 a 1831 para ajudar na
construção da civilidade europeia contemporânea tão rarefeita no local,
em sua pequena mas profunda sociologia da escravidão brasileira
suavemente ilustrada nos precisos comentários às suas aquarelas e
desenhos, aparentemente superficiais -- em um conjunto notável em estilo
de catálogo, de enciclopédia social iluminista -- deixa ver com clareza
o tipo de ócio e de corpo, a vida adocicada e preguiçosa dos senhores,
que não carregavam por si mesmos suas próprias tralhas na cidade do Rio
de Janeiro, quando não eram eles próprios, ou, ainda mais elas,
carregados em redes ou liteiras pelas ruas. E, ao mesmo tempo, em
contato com estes corpos senhoris em estado de preguiça e de torpor
tropical, \emph{sensuais} \emph{ao} \emph{seu} \emph{modo}, dizia
Schlichthorst, encontramos sempre o seu radical \emph{outro}, do
trabalho, e do ``erotismo'', os escravos e as escravas urbanas que
negociavam frutas, flores, doces e água, entre outros serviços e
favores, pela cidade.

E ficamos sabendo, de modo bem cifrado pela moral racional do artista
francês, ao final da célebre prancha em que ele recuperou o semblante de
cada uma das nacionalidades africanas das mulheres que aqui se
encontravam como escravas, na sua antropologia visual da
particularidade, que: ``as negras monjolas são mais particularmente
geniosas, mas compartilham da alegria, da vaidade \emph{e sobretudo da
sensualidade} que caracterizam as congos, rebolos e benguelas'' {[}o
grifo é meu{]}. Novamente os corpos em oposição no trabalho e na
violência da hierarquia social brasileira opõem texturas biopolíticas
diferentes: de um lado preguiça e desleixo com o corpo amolecido de
senhores e de senhoras brancos, ex"-portugueses novos brasileiros, de
outro a disposição necessária do trabalho e o erotismo presentificado,
afirmativo, das escravas urbanas, prontas para sobreviver, e também, de
algum modo, viver em tal ordem de poder que lhes era amplamente
desfavorável.

E até mesmo Darwin, o verdadeiro cientista representante da cultura
burguesa inglesa, voz encarnada da ideia de dignidade humana então em
voga que, por um lado, em seu diário da viagem do Beagle, em 1833,
apontou com veemência a violência incrustrada nas práticas e nos corpos
dos recém brasileiros, o estranho do sadismo e da dor real da escravidão
que tomava as ruas e as cidades, afirmando os valores humanitários que o
liberalismo europeu inventava como baliza para o julgamento do outro,
também ele, por outro lado, não deixou de ser sensível às graças únicas,
exclusivas daquela nova ordem de civilização: ``deixando de lado a ideia
de escravidão, tem alguma coisa de deliciosa nesta vida patriarcal, tão
absolutamente separada e independente do resto do mundo.'' Deixando de
lado a ideia de escravidão\ldots{}, o Brasil era um país muito interessante.

\section{VII}

Para completar o quadro com bastante precisão frente às mazelas de
longuíssima duração de nossa vida civil incivil, Schlichthorst também
teve um vislumbre, em um discurso que já é uma boa performance
brasileira, de algo que pode ser visto tanto como uma \emph{antropologia
urbana} \emph{da escravaria} do século \versal{XIX} local quanto como a
emergência de um sistema ideológico mais poderoso, orientado, ainda uma
vez, para dispensar o senhorio da responsabilidade por sua própria
civilização, uma ordem muito especial e profunda de violências, que
começava a ser festejada.

O soldado alemão, agregado às tropas brasileiras, um possível
proto"-brasileiro que fazia cálculos com juros para saber quanto
enriqueceria em vinte anos \emph{se investisse em escravos no Brasil},
concluindo ser um negócio razoável, tem também o seu verdadeiro
vislumbre do paraíso: daquilo que um dia seria o lugar simbólico
privilegiado da \emph{cultura brasileira}, e sua imagem original
realizada de \emph{eros} e \emph{civilização}, bem fundada, como não
poderia deixar de ser, na vida local do ex"-africano, negro:

``Por mais dura que seja a um ouvido europeu a palavra escravidão, esse
estado é, na América do Sul, em geral suportável. O português e o
Espanhol tratam bem seus escravos, sem dúvida melhor do que o plantador
das Índias Ocidentais, seja inglês, holandês ou francês. Na cidade, o
escravo é senhor de seu nariz e cuida de sua vida sem sujeição de
qualquer violência. Não são grandes suas necessidades, e seus gozos nada
lhes custam. Qualquer terreno baldio vale por uma sala de baile,
qualquer rapariga torna"-se a dama do seu coração e, se de quando em vez
necessita dum incitamento à alegria, encontra em todas as vendas, como
bebida predileta e baratíssima, a cachaça.''

A longa duração desta noção, tão ideológica quanto concreta, a respeito
do que viria a ser o povo brasileiro, impressiona. Schlichthorst fazia
parte do grupo de viajantes estrangeiros que via com simpatia o modo de
ser da escravidão brasileira -- a ideia da ``fácil adaptação dos
escravos africanos ao Brasil'', refutada por Luiz Carlos Soares -- em
oposição a um outro, em que talvez Darwin seja a voz mais forte, que
reconhecia imediatamente a sua dimensão de irredimível violência,
\emph{dívida impagável}. Segundo o relato, o escravo urbano brasileiro,
já na sua origem, era uma espécie de \emph{meio sujeito}, de uma vida
\emph{meio cativa}, produzindo experiências urbanas de um certo
erotismo, singular e concreto. Sua vida também era marcada por próprias
vantagens, produto de uma civilização ocidental particular, não
constituída de forma \emph{excessivamente rígida}. Deste modo, \emph{um
terreno baldio} \emph{é uma sala de baile}, realizando uma imagem
erótica e onírica pré"-moderna brasileira, uma imagem dialética forte
\emph{de nossa civilização da precariedade} que vai
atravessar grande parte de nossas ilusões e pesquisas estéticas e
civilizatórias muito posteriores, já de cunho modernista.

É do espaço indeterminado de uma civilização incivil nos termos dos
direitos universais, mas também incorrespondente às específicas
violências super estruturadas modernas europeias, aos olhos do alemão,
que surge a utopia de nossa expressão erótica concreta, junto à natureza
e ao próprio gesto e corpo, \emph{como civilização}. E o negro escravo,
semi"-autônomo da cidade, mas escravo na apropriação real do valor de seu
trabalho e vida, é o corpo social que expressa essa civilidade erótica
moderna, utópica de um modo próprio, e radical.

Qualquer terreno baldio, uma rapariga e um pouco de cachaça, na
insolidez geral do país, produz a vida do espírito nova de um povo cujo
real estatuto jurídico é o de \emph{escravo}, o que, segundo o soldado
alemão, não deveria soar muito duro aos ouvidos europeizados de então,
parciais, formalizados. Neste mundo, para este povo, realizava"-se uma
modalidade precoce de \emph{onirismo} \emph{moderno}, baseado nas
chibatas reais, em que \emph{um terreno baldio é um salão de baile}. Ou,
de outra perspectiva, tratava"-se de uma cultura lúdica, musical
corpórea, compensatória do terror, existente nos interstíscios da vida
dominada, que operava com as mesmas potências criativas \emph{na ilusão}
-- que não equivale à \emph{mentira}, nos lembra Donald Winnicott -
próprias de quando as crianças brincam. Aqui brilha a famosa aquarela de
Debret, ``O primeiro ímpeto da virtude guerreira'', entendida por
Rodrigo Naves como exatamente a configuração plástica da \emph{intuição
deste espaço social rarefeito}, de \emph{formalização difícil} pelos
padrões centrais, e quando lido fora de seus próprios termos
históricos.\footnote{Rodrigo Naves, ``Debret, o neoclassicismo e a
  escravidão'', em \emph{A forma difícil}, São Paulo: Ática, 1996.} E o
que mais este tipo original de deslocamento simbólico era capaz de
produzir?

Esta mesma concepção das coisas do Brasil, da cultura espontânea popular
e erótica, estará presente, por exemplo, do samba urbano e de seu
carnaval até a \emph{poesia pau Brasil} e o \emph{Macunaíma}, dos nossos
grandes modernistas, dos bólides e dos parangolés de Hélio Oiticica --
uma palavra que ele recolheu mesmo junto a um mendigo, um negro velho\ldots{}
e uma \emph{obra corpo} \emph{que fazia exatamente com que qualquer
terreno baldio fosse um salão de baile}\ldots{} -- ao cinema marginal de um
\emph{Bandido da luz vermelha} e do Brasil popular radical, alucinado e
outro da vida ocidental, espontâneo informado, da Belair, de Rogério
Sganzerla e de Julio Bressane. Todas estas são obras e mundos avançados
que pesquisaram e se constituíram diante e na \emph{civilização da
precariedade} brasileira, uma experiência popular original que se tornou
pedra de toque da vanguarda futura possível.

E, enfim, no estudo social incorporado de Schlichthorst, surge a fala
original da \emph{cultura brasileira} pela primeira vez, o primeiro
resultado daquela mesma \emph{África que um dia civilizaria o Brasil},
segundo o político imperial conservador, escravista, Bernardo Pereira de
Vasconcelos:

``O canto, a dança e os folguedos enchem as horas de folga dos escravos.
Quando se quer ver gente alegre, basta procurá"-los. De natureza é o
brasileiro melancólico, muito sensual, cerimonioso, e desconfiado,
qualidades essas que não produzem a verdadeira alegria. A inconsciência
do negro deixa"-lhe gozar do que o momento lhe propicia, sem cuidados
sobre o futuro. Sua dança predileta chamava"-se fado, e consiste num
movimento trêmulo do corpo que, suavemente embalado, exprime os
sentimentos mais sensuais de um modo tão natural como indecente. São tão
encantadoras as posições desta dança que muitas vezes os dançarinos
europeus às imitam no Teatro de S. Pedro de Alcântara, recebendo
aplausos entusiásticos.// Encontram"-se entre os negros, excelentes
improvisadores. Todos os seus trabalhos, folguedos e danças são
acompanhados de cantigas. Todas as impressões que recebem tomam uma
forma poética. O pensamento gera a rima, e a rima gera outro pensamento.
Estão sempre cantando suas felicidades e suas dores em estâncias mais
curtas ou mais longas.''

O escritor estrangeiro nota a diferença cultural entre o espírito e os
corpos dos senhores e o de seus escravos e observa que, embora
desprovidos de direitos civis e de acesso ao espaço público da política
e da economia que os implicava, eram os escravos que, com sua dança e
sua música, percebidas e sentidas como ricas, ocupavam o espaço do mundo
da vida e da rua, com elementos simbólicos fortes e novos. Eram eles que
produziam o afeto da \emph{alegria}, que seria ideologicamente tão
característico do novo espaço social e nacional. Com coerência e
inteligência quase conceitual, Schlichthorst concebe uma verossímil
teoria materialista para o importante problema da felicidade dos pobres
no Brasil, entendida como a \emph{inconsciência}, ligada a uma
hipersensibilidade para o presente, advinda de uma \emph{plena
consciência} \emph{sobre} \emph{não ter futuro}. Assim o alemão nos
oferece uma primeira imagem das práticas de vida criativa, da música
misteriosa e mágica, mesmo que feitas à luz do dia e em plena rua, na
capital do Brasil, dos reais antepassados do que um dia seriam os nossos
Cartolas, Caymmis e Geraldo Pereiras -- e Joãos Gilbertos, Gilbertos
Gils e Caetanos Velosos, porque não dizer -- nos dando intuição para uma
tradição profunda, de longa duração, de uma experiência popular
brasileira originária que explica muita coisa. A \emph{performance
negra}, uma dialética de vida com a escravidão que não podia limitar a
potência dos corpos em absoluto, ganhava vida literária e conceitual no
olhar bastante livre do escritor alemão.

O caráter repressivo de uma sociedade racista e religiosamente moral não
podia se expressar então, na totalidade do mundo da vida e da cidade,
como ação prática e política, policial, contra a vida cultural de maior
erotismo dos pobres, escravos, negros, isto por que, muito
provavelmente, em um país que ainda precisava importar até mesmo seu
próprio exército por ausência demográfica significativa de um povo livre
local, não havia força institucional, burocrática e nem riqueza
disponível para a repressão geral de uma ampla expressão popular que se
generalizava. Além disso, os escravos e escravas eram mandados à viração
todos os dias e sua renda possível dependia de algo de seu próprio
movimento livre, como bem se pode observar no trabalho de Debret. Nessa
dimensão da vida, eles eram, por um segundo, quase \emph{donos dos seus
narizes}, nas palavras do soldado alemão. Assim se produzia uma múltipla
inserção e produção do escravo da rua, que se representava em várias
esferas, inclusive na criação estética, a contrapelo do poder, da nova
ordem nacional. Dito nas palavras do historiador: ``as relações
escravistas de trabalho no ambiente urbano implicavam, obviamente, uma
exploração direta dos senhores sobre os cativos, mas este tipo de
exploração também se combinava com outras formas indiretas de exploração
(que envolviam terceiros) e outras formas de trabalho e obtenção de
renda que não significavam o controle total, pelos senhores, do que era
produzido ou arrecadado pelos cativos (\ldots{} \emph{no caso da escravidão
de ganho}) uma parcela bastante variável da renda obtida ficava nas mãos
dos cativos, sem que isso viesse a afetar a natureza das relações
escravistas, no sentido de sua transmutação em relações de trabalho
assalariadas.''\footnote{Luiz Carlos Soares, \emph{O ``povo de Cam'' na
  capital do Brasil: a escravidão urbana no Rio de Janeiro do Século
  \versal{XIX},} \emph{op. cit.}, pág. 139.}

E ainda, sobre a alegria dos escravos, segundo Debret, desde os navios
negreiros até os grandes armazéns de negócios do Valongo -- os
\emph{palácios} dos negociantes de escravos locais\ldots{}­ -- \emph{o
antidoto para a depressão e a desmobilização dos escravos era fazê"-los
dançar e festejar}, em um paradoxo de inversão do sentido das coisas
humanas verdadeiramente notável, \emph{real} brasileiro, um festejo em
falso, a vida nua como única celebração, radical biopolítica perversa
brasileira, que deu origem exatamente à cena patética limite famosa que
Heine figurou em seu poema sobre o navio negreiro internacional: do
traficante que obrigava os africanos a festejarem, no mesmo movimento
que os coisifica e escraviza, no navio e no mar que os
desterritorializava de tudo. Menos disso: ``Impressionados com as perdas
de homens, que aumentava demais o preço dos escravos, os traficantes
sentiram necessidade de trazer menos negros da cada vez e de tratá"-los
mais humanamente; de fato, desde então proporcionam"-lhes uma distração
consoladora, deixando"-os subir todo dia ao convés, cujo ar puro os
predispõe mais facilmente a dançar de vez em quando ao som de uma música
que, apesar da mediocridade, ainda os encanta, e bem mais quando a eles
se juntam as negras dançarinas.''\footnote{Jean"-Baptiste Debret, Org.
  Straumann, Patrick. \emph{Rio de Janeiro cidade mestiça}, São Paulo:
  Companhia das Letras, p. 19.}

Além disso, segundo o soldado alemão brasileiro, toda a experiência da
vida negra era concebida, simbolizada, na forma poético musical, de
qualidade e de inteligência formal perceptíveis. Aquele registro
estético poético, lúdico erótico, a música e o canto, era o modo de uma
experiência radical de violência em plena modernidade dizer de si
própria, o \emph{vértice} simbólico, como dizia Bion, que falava a
história, o nome e a vida, simultaneamente, daqueles que não conheciam
nenhum acesso à cultura letrada, própria do poder mundial que os
submetia, ocidental, colonial. No Brasil o pensamento popular, e negro,
sua metafísica, ontologia e filosofia -- como nos Estados Unidos,
segundo o pensador de \emph{teoria negra} Fred Moten, mas com resultados
históricos diferentes\ldots{} -- sempre foi musical, e também sempre dançou,
como um dionisíaco europeu moderno, no limite das contradições de seu
próprio espaço histórico, desejou mesmo que Deus o fizesse.

Schlichthorst via de fato tudo isso com grande benevolência erótica e
intelectual, própria de um homem inteligente e livre. Ele entendia a
expressão negra muito significativa nas ruas do Brasil dos anos de 1820
como matriz real de cultura legítima, produtora de formas de vida e de
arte nova, na dança, na música e no corpo dos brasileiros que de fato
trabalhavam. Dava notícia e se deixava levar de algum modo pela força do
\emph{improviso ontológico}, musical, corpóreo e erótico, da real
performance negra brasileira, que não era absolutamente reconhecida pelo
quadro do poder, mas que, para ele, tinha especificidade e força de
existência, futuro significativo, como o próprio país. Sua liberdade de
homem exterior à forma social do poder local, duplamente
desterritorializado, naquele momento nem brasileiro nem europeu, lhe
permitia conceber com erotismo e pensamento em movimento o objeto
erótico e ideológico que nenhum brasileiro ao seu tempo podia sequer
nomear. Que todos, de fato, recusavam o acesso ao símbolo.\footnote{É
  possível também um pensamento ampliado da recusa ocidental em nomear a
  sua pratica econômica secular da escravidão colonial, e sua vida, em
  dar nome à coisa e sua experiência: ``(\ldots{}) Embora as histórias sobre
  a escravidão nos Estados Unidos tenham mudado muitas vezes desde o fim
  do século \versal{XIX}, elas nunca foram realmente sobre a escravidão. Essas
  histórias tem sido determinadas pela natureza específica do
  engajamento de negros americanos na economia política do país no
  momento em que são contadas. Em outras palavras, a historiografia da
  escravidão é uma grande sinédoque. Uma sinédoque é uma pequena
  história que deve representar uma maior, uma que é mais importante
  para quem está contando. E assim, a história da escravidão foi
  utilizada repetidamente como uma sinédoque do status
  econômico"-político das pessoas afro"-americanas -- e de como o contador
  daquela história em particular acha que seu futuro status deve ser.''
  Edward E. Baptist, ``Seres humanos escravizados como sinédoque
  histórica: imaginando o futuro dos Estados Unidos a partir de seu
  passado'', em \emph{Escravidão e capitalismo histórico no século \versal{XIX},}
  Rafael Marquese e Ricardo Salles, \emph{op.cit.}, p. 265.}

Menos o estrangeiro, que em algum ponto, desde sua viagem no navio que
por segundos se confundia com um negreiro, um devir negro brasileiro, em
algum ponto se identificava e recebia aquela experiência popular como
também dele. Assim ela ganhava forma como nunca teve para os próprios
senhores brasileiros, que formavam o país em grande cisão com sua
própria vida, se recusando abertamente a pensar aquele mundo.

Embora todos o \emph{usassem}.

\section{VIII}

Aquela espécie de pobre diabo e aventureiro europeu, figura liberada
pelos novos fluxos sociais do capitalismo em ascensão do século \versal{XIX}, já
lançado nos movimentos titânicos da modernidade, de fato via as coisas
dos primeiros tempos brasileiros de modo muito diferente da real
sugestão de repressão da vida popular que, com forte mentalidade ao
contrário, o colono português e \emph{professor régio de língua grega}
na cidade da Bahia -- ou seja, Salvador -- Luiz dos Santos Vilhena,
propunha, ainda em 1802, em carta endereçada ao seu \emph{Soberano e
Augustíssimo Príncipe Regente N. Sr. o Muito Alto e Muito Poderoso
Senhor} Dom João, missiva áulica e de interesse na qual expunha, de modo
embrulhado, uma opinião radicalmente oposta frente à mesma experiência
do \emph{proto"-povo brasileiro, africano}, que tomava a cidade:

``Por outro princípio não parece ser muito acerto em política, o tolerar
que pelas ruas e terreiros da cidade façam multidões de negros de um e
outro sexo, os seus batuques bárbaros a toques de muitos horrorosos
atabaques, dançando desonestamente canções gentílicas, falando línguas
diversas e isto com alaridos tão horrendos e dissonantes que causam medo
e estranheza, ainda aos mais afeitos, na ponderação de consequências que
dali podem porvir, atendendo ao já referido número de escravos que há na
Bahia, corporação temível e digna de bastante atenção a não intervir a
rivalidade que há entre os crioulos e os que não o são; assim como entre
as diversas nações de que se compõe a escravatura vinda das costas da
África. // Seria muito para desejar que estes se pusessem num estado de
subordinação tal que julgassem quanto ao respeito que qualquer branco
era seu senhor, e não em uma altivez que geralmente se veem todos os que
são de pessoas que figuram por suas qualidades empregos e haveres que
não duvidam tratar todos os demais brancos com aquela displicência e
pouco apreço com que observam serem tratados por seus senhores; muito
curta serão as luzes de quem não conhecer a suma importância de um tal
rasgo de política em uma cidade povoada de Escravos, cafres e tão bravos
como feras.''

Para esta postura fundamental -- escrita meio em português, meio em
grego -- de política paranoica e repressiva, compreensível como controle
social interessado, nada havia a admirar, viver, fruir ou aprender com a
intensidade real da cultura da vida, da rua negra e mestiça, que emergia
com força nas cidades daquele mundo, que seria, em breve, o Reino Unido
de Portugal, Brasil e Algarves, e, alguns anos depois, o Império
Constitucional do Brasil. O verdadeiro resultado final daquela postura
biopolítica colonial portuguesa explícita seria um racismo bem
transfigurado em desprezo pela vida popular no Brasil, para o qual, ao
longo do tempo, as vítimas do poder seriam sempre as verdadeiras
culpadas pelo vexaminoso do nosso atraso. Além do grande amor
generalizado pela repressão social antipopular. Aqui se confirma a
\emph{divida impagável} da vítima, a torção instrumental do racismo
sobre a razão ocidental, duplicando o gozo da exploração capitalista
colonial, já ela própria ilimitada, nos termos contemporâneos de Denise
Ferreira da Silva.\footnote{Denise Ferreira da Silva, ``Unpayable debt:
  reading scenes of value against the arrow of time'', em \emph{The
  Documenta 14 reader}, editora Munique: Quinn Latimer, 2017.}

Vilhena reeditava exatamente, no início do século \versal{XIX}, a posição ainda
inquisitorial que Nuno Marques Pereira -- um escritor reformista situado
entre a piedade e mística católica popular e a afirmação pesada da
soberania portuguesa colonial -- expressara no início do século \versal{XVIII},
nas páginas de seu discurso híbrido, entre ficção e memória sobre o
Brasil, o \emph{Compendio narrativo do peregrino da América}, no qual,
entre outras peripécias de pequeno mandarim absolutista da cultura,
\emph{mandava queimar os instrumentos de batuque dos negros} das
fazendas rurais brasileira\emph{s}, diante de senhores locais que já
julgavam melhor \emph{conviver} com os \emph{calundus} dos negros\ldots{} e
ameaçava com \emph{o espancamento exemplar e purificador a populares
brasileiros} que executassem alguns tipos de \emph{música nova}, que ele
não define precisamente as qualidades, mas que não eram estruturadas
pelos códigos das práticas do \emph{trivium} e de corporações de ofício
de origem antiga, medieval, em que ele parecia ter sido educado na
Bahia. Deste modo o \emph{peregrino} católico ciente do núcleo simbólico
do poder colonial que enunciava julgava assim o canto popular que ouvira
na rua:

``E foi o caso: que estando eu uma noite na Cidade da Bahia, ouvi ir
cantando pela rua uma voz: e tanto que punha fim à copla, dizia, como
por apoio da cantiga: Oh! Diabo! E fazendo eu reparo em palavra tão
indecente de se proferir, me disseram que não havia negra, nem mulata,
nem mulher dama, que o não cantasse, por ser moda nova que se usava.
(\ldots{}) Outras muitas música desonestas tenho ouvido cantar; como é uma
moda, que se usou, e ainda hoje se canta, e acaba dizendo: \emph{Berra
tua alma}. Parece que quem tal canta, e folga de ouvir cantar, já estão
anunciando o como lhes há de vir a suceder quando forem ao inferno,
chorando e berrando, pelas profanas músicas com que nessa vida pecaram e
foram causa de fazerem pecar muitos. (\ldots{}) Se eu fora ministro da
Justiça, ou tivera poder sobre eles, eu os fizera cantar ou berrar ao
som de golpes de um verdugo pelas ruas públicas, para seu castigo e
emenda dos mais, que tais modas usam. E veriam então se lhes valia o
Demônio, por quem chamam.''

O moralista religioso de mentalidade inquisitorial contrareformista do
início do século da explosão da riqueza colonial do Brasil via na música
que se hibridizava e se tornava expressão social própria que existia --
uma \emph{modernização} desde o mundo antropologicamente complexo da
vida colonial escravocrata -- um risco ao espírito de dogma e
referencias prévias, tanto religiosas quanto da plena soberania cultural
portuguesa sobre o espaço colonial americano.

José Ramos Tinhorão, que estudou o \emph{Peregrino da América}, conclui
a respeito da postura diretamente orientada contra a música portuguesa
em contato com a vida negra, desta espécie real de \emph{sensor civil da
colônia}: ``Para o vigilante ideólogo político"-moral"-religioso do
absolutismo o perigo das novas formas de música de dança e de canções
profanas (quer dizer, fora do alcance do controle da Igreja e do Estado)
estava em afastar"-se dos seus \emph{predicados} oficiais, o que vinha
acontecendo com frequência, principalmente no Brasil colônia, onde a
diversificação social começava a efetivar"-se longe dos modelos, da
vigilância e das sansões do poder real central.''\footnote{José Ramos
  Tinhorão, \emph{A música popular no romance brasileiro}, 2ª edição
  revista e ampliada, São Paulo: Editora 34, 2000, p. 26.}

De fato, aquele universo animado que tanto preocupava Vilhena e Nuno
Marques Pereira existia já em profundidade histórica, e era presença
forte, o próprio espaço público, o mundo da vida, e de modo diferente da
sociedade de mão de ferro das colônias de povoamento inglesas da América
do Norte ele parece não ter sido excessivamente reprimido em sua
expressão nas cidades brasileiras da origem. Foi exatamente este grande
problema de vida social original que deu um nervo estrutural de forma ao
\emph{Memórias de um sargento de milícias}, de 1854-1855, de Manuel
Antonio de Almeida, a primeira pequena grande obra da literatura
nacional, que -- como sabemos desde o esclarecimento de Antonio Candido
-- tem sua organização interna \emph{dialética} muito viva dada entre os
polos temáticos, sociais e formais da \emph{ordem}, o mundo do Estado e
da lei quando se aproximava do rei, e da \emph{desordem}, do mundo da
vida popular e da rua, dos capoeiras, dos terreiros de feitiçaria e das
festas, mundo bem perseguido, farsescamente, mas que também, como
ocorria na vida, não podia ser inteiramente reprimido, no proto"-Brasil
das origens.

Assim, com antecedência, o, sob certo ponto de vista, muito moderno
Schlichthorst, um homem já liberado para as possibilidades do mundo meio
livre e em grande renovação europeu -- quando se escapava de ser
proletário -- em alguma medida ciente, pela própria condição, das
violências da própria modernidade central, podia ver nas manifestações
originais de algo do Brasil escravo um traço de criação de vida e de
cultura forte e desejável, algo que a própria cultura estética moderna
europeia, provavelmente já impactada pelo valor legítimo do popular e da
ideia do povo, pós Revolução Francesa, também podia perceber: ``são tão
encantadoras as posições desta dança {[}\emph{o fado, herdeiro do lundu,
e pai do samba}{]} que muitas vezes os dançarinos europeus às imitam no
Teatro de S. Pedro de Alcântara, recebendo aplausos entusiásticos.''

No entanto, noutra direção, oposta, Vilhena, o erudito colono português
de Salvador, um proto"-brasileiro livre com aspiração à elite, ao modo do
inquisidor moralista do século \versal{XVIII}, só podia recusar abertamente esta
dimensão da vida local, do que afinal não estava, naquele momento, sob
nenhum signo de constituição de país. Não havia sensibilidade alguma do
português passadista a respeito de uma possível nova ordem humana que se
formasse por aqui, e sua expressão cultural. Existia apenas a rigidez da
própria oposição, novamente embrulhada em cultura clássica, e disponível
e feliz com a submissão prática ao seu \emph{muito alto e muito poderoso
senhor}.

De um lado, representado pelo professor português de grego no começo do
século \versal{XIX} na colônia do Brasil, temos o convite à repressão e ao
desprezo aberto pela vida popular, antierótico por natureza, afirmativo
do poder em que ideologia das ruas e Estado se encontram, que, além do
controle político do espaço e das relações na cidade, provavelmente era
orientado para as ideias culturais tradicionais de nobreza, honradez e
superioridade, de tipo antigo e moderno; enquanto, na outra direção,
oposta, temos a curiosidade, a tolerância cultural e o erotismo
produtivo, mesmo que volúvel, expressos pelo mercenário alemão, um homem
que claramente estudava a possibilidade de \emph{se tornar um
brasileiro}.

São estas as duas posturas originais, antitéticas, mas também
provavelmente em dialética, diante do espaço de criação e de experiência
que era legado à escravaria urbana brasileira no início do século \versal{XIX}. A
repressiva, de Vilhena, que sustentaria a violência da exclusão e o
racismo de então, e a culturalmente inclusiva, mas economicamente
exclusiva, de Schlichthorst, que, volúvel deste modo preciso, permitia à
sua própria perspectiva \emph{moderna} um excedente de uso e de gozo
daquela ordem tão especial, da vida do trabalho e da cultura da vida em
uma nova nação americana, além de poder \emph{concebê"-la}.

Esta segunda postura -- em que um popular alemão \emph{devora} e
\emph{digere} o Brasil que se fundava -- formação primeira muito
original que terá vínculos com a nossa perspectiva modernista e com
nosso amplo, genérico e festivo \emph{tropicalismo} cultural, também se
desdobraria ainda em um terceiro movimento no campo dos costumes, da
música e da dança no novo espaço histórico social que se criava. Um
movimento iniciado nos primeiros sinais da emergência de um espirito
nacional possível, no século \versal{XVIII}, quando da explosão do território
precioso da Capitania das Minas Gerais da colônia portuguesa do Brasil.

Trata"-se do processo de incorporação pelo alto, da fusão cultural entre
portuguesa e africana na própria vida da elite gestora da exploração
colonial, que passava a ganhar alguma consciência do fenômeno. Assim ele
foi expresso por Tomás Antônio Gonzaga -- lembrado por José Ramos
Tinhorão -- nas passagens das \emph{Cartas} \emph{Chilenas} em que se
referiu \emph{ao batuque que aconteceu nos bastidores do palácio do
Governador das Minas Gerais}, onde uma dança de salão que envolvia
umbigadas teve lugar.\footnote{José Ramos Tinhorão, ``A modinha'', em
  \emph{Pequena história da música popular}, Petrópolis: Vozes, 1978.}
Naquela festa e naquela dança, os homens iam ``seguindo da viola o
compasso'' algo que pouco tempo antes, segundo o poeta, ``só era visto
em humildes choupanas, onde negras e mulatas batiam no chão o pé
descalço''.

O poeta inconfidente reconhecia de fato, mesmo que em tom satírico, a
presença da vida popular na dança e na música dos renovados costumes
portugueses, na própria vida da casta dirigente, dentro do palácio do
Governador Cunha Menezes, quase meio século antes de Schlichthorst
admirar muito aquela mesma dança e potência humana, que ocupava
plenamente as ruas do Rio de Janeiro.

Ficava assinalada a fusão cultural, entre a vida portuguesa e a nova
vida popular, com seus vínculos africanos. A mesma que permitiria que,
ao final daquele século \versal{XVIII} brasileiro, o mulato carioca Domingos
Caldas Barbosa -- também admitido com o nome de Lereno Selinuntino na
Arcádia de Roma\ldots{} -- registrasse, no segundo volume da sua coletânea de
versos musicados, a \emph{Viola de Lereno}, o \emph{lundu canção} de
título \emph{Nhanhazinha}, uma forma da modinha portuguesa antiga já
atravessada por ritmos novos de algum batuque local, e que ele parece
ter apresentado quando foi trovador palaciano em Lisboa, a partir de
1775, e cujo total colorido e bossa brasileira, com a presença da vida
popular, dispensa comentário:

\begin{verse}
Se não tem mais quem te sirva\\
O teu moleque sou eu.\\
Chegadinho do Brasil\\
Aqui stá que é todo teu.
\end{verse}

\section{IX}

Apesar da presença geral, maciça e culturalmente forte da mão de obra
escrava na vida brasileira dos primeiros anos do século \versal{XIX}, discursos
literários claros, destemidos e abertos, e, sendo assim, biopolíticos a
respeito daquela ordem de coisas e civilidade, como foi o de
Schlichthorst, sempre foram raros na vida intelectual brasileira de
então. Se é que é possível, por isso, falarmos em vida intelectual
brasileira na primeira metade do século \versal{XIX}. Havia, certamente, ativa
vida ideológica nacional àquele tempo, e o episódio importante, uma
estratégia cultural quase de Estado, da busca e da afirmação da temática
indígena como o campo simbólico que deveria ordenar e representar o país
irrepresentável, ``sem história'' e \emph{sem povo} -- em um circuito de
produção de ideias que envolveu os incipientes escritores brasileiros
bem como pensadores e escritores de Portugal e França -- é expressivo do
tipo de trabalho programático, edificante conservador dos ideais
nacionais superiores elitistas que se projetavam, e que, sobre outros
critérios modernos não configuraria uma vida livre, e de implicação, das
ideias no Brasil dos primeiros anos.

De fato, discursos sobre a vida do trabalho no Brasil praticamente
inexistiram nas origens do Brasil. A matéria e a vida do trabalho, o
real mundo da escravidão brasileira, em geral não configurou vida
simbólica suficiente para os próprios brasileiros que a viviam. Tudo o
que sabemos a respeito da escravaria brasileira no século da origem
nacional, principalmente nos primeiros anos do país, sempre pareceu vir
das impressões esparsas e \emph{estrangeiras} dos viajantes europeus ou
americanos, que, de fato, criaram uma quase tradição a nosso respeito,
que alimenta não por acaso a maior parte da historiografia sobre o
período, como Debret, Rugendas, Luccock, Graham, Expilly, Ewbank,
Schlichthorst, entre outros.

Nossa alienação específica para a situação, que nos levava a cometer
coisas estranhas como o \emph{Nicteroy} de Januário da Cunha Barbosa, em
detrimento da luz, dos corpos e dos jogos de linguagem que de fato
estavam nas ruas do Brasil, nosso comprometimento edificante com a ordem
do poder originalmente colonial restante por aqui, nosso gozo concreto
aprofundado e prévio da situação econômica e social dada, nossa
instabilidade política constante e de raiz e nossa estabilidade social
real da vida bem afirmada entre senhores e cativos, nossa proverbial e
prévia falta de imaginação para a fantasia e para a crítica e submissão
normativa à hierarquia e ao poder, lembradas e estudadas por Sérgio
Buarque de Holanda em \emph{Visão do paraíso} e \emph{Raízes do Brasil},
nos impediam de modo mais ou menos definitivo de simbolizar, até mesmo
positivamente, a contraditória civilização escravocrata moderna que
ensaiávamos e que sustentávamos nos trópicos.

Com a exceção do discurso original matizado e pragmático de José
Bonifácio, que via a si próprio como ``filantropo, cristão e liberal'' e
que em proposta pública à constituinte de 1823 já concebia a extinção da
escravidão, e a sua reparação, como um projeto positivo de nação, mesmo
que gradual, ``por ser da maior necessidade ir acabando tanta
heterogeneidade física e civil'', aquele silêncio intelectual e
literário foi aparentemente a norma mais geral sobre a coisa real do
Brasil dos primeiros tempos. De um ponto de vista da consciência do país
para si próprio, até os anos \emph{das estreias} de nossa literatura
mais forte, como dizia Machado de Assis, na década de 1850, praticamente
não houve representação simbólica positiva da estrutura e da vida
escravocrata brasileira entre os próprios brasileiros.

Anunciava"-se assim, em profundidade histórica, precocemente, um
embotamento muito próprio de nosso mundo conservador, relativamente
pouco imaginativo ou inquieto, um universo que tradicionalmente se
expressa por generalidades de interesses basais e, principalmente,
\emph{essencialmente em ato}, \emph{fundamentalmente} \emph{como}
\emph{coisa dada}, mais do que por alguma ordem de pensamento
trabalhado. Nosso real negacionismo, que é ataque ao próprio pensamento
e também de toda dimensão crítica da própria vida simbólica ocidental, é
valor político e subjetivante forte dos modos de ser e vir a ser
brasileiros. E, com a dispensa prática do mundo do pensamento, se
dispensava também o mundo da vida, ou das vidas, a ele possivelmente
ligadas. Noutras palavras, \emph{nada de modernidade na vida estável do
escravismo brasileiro}. Não havia modernidade ativa, experiência
histórica da instabilidade do tempo do capital, como conceito e como
escritura, estrutura da enunciação, no mundo das práticas comuns da
escravidão brasileira.

O pensamento conservador, super estruturado como vida social e vazio de
vida imaginativa para si próprio, chegava mesmo a reconhecer"-se
reflexivamente quando contrastado com a famosa falsa lei da abolição do
tráfico brasileira, que nunca gerou prática social alguma e que no
entanto existia:

``Senhores, houve um erro grave quando se promulgou a lei de 7 de
Novembro de 1831. Não se muda o estado de um país de um momento para o
outro. Os nossos hábitos de três séculos, durante os quais foi mantida a
escravidão entre nós, as necessidades do país, tudo enfim pedia que com
um rasgo de pena não se dissesse -- fica abolido o tráfico -- sem que
para isso nada se tivesse preparado. Uma abolição tão repentina não era
possível. Entretanto foi o que se fez, e, quaisquer que fossem as causas
que obrigaram os nossos homens de estado a fazer a lei de 7 de Setembro
de 1831, houve erro grave em decretá"-la, sem que primeiramente se
preparasse o país, sem que se tratasse da introdução de Braços livres,
sem se fazer cousa nenhuma, nem antes nem depois da lei, sendo o nosso
país puramente agrícola e dependendo de braços para viver.''\footnote{Em
  \emph{A força da escravidão,} de Sidney Chalhoub, São Paulo: Companhia
  das Letras, 2012, pág. 116.}

Assim, apenas no combate aos efeitos e à própria lei que proibiu o
tráfico no Brasil ficamos sabendo que \emph{os hábitos de três séculos
se mantinham}, que \emph{nada foi feito}, \emph{nem antes nem depois} de
uma lei nacional que não marcou nada e, portanto, \emph{nada foi de fato
pensado ou imaginado social ou politicamente} naquela ordem efetiva,
sobre a qual, para destacarmos a expressão a respeito deste real mais
forte, do próprio discurso, ``tudo pedia que com um rasgo de pena
\emph{não se dissesse}''\emph{\ldots{}} Esta fala, feita nos debates de 1848
sobre o fracasso das ações imperiais frente ao ato social total do
tráfico ilegal de escravos brasileiro na primeira metade do século \versal{XIX},
era a resposta do então ministro da justiça conservador Campos Mello a
questionamentos do deputado Moraes Sarmento, exatamente o mesmo que
pedira a instauração de \emph{sessão secreta} da Câmara para a discussão
da reforma da lei falida, justificando"-se: ``que na discussão desse
projeto talvez pudessem aparecer ideias que não conviria muito que
fossem ouvidas, atenta a consideração de ser o Brasil um país onde há
grande número de escravos.'' Sidney Chalhoub comenta a fala: ``portanto,
se preocupava com o que os escravos viessem a saber sobre o que se
falava no Parlamento''.\footnote{Idem, pág. 111.}

Assim se produzia o \emph{debate secreto} das elites brasileiras sobre o
assunto central do país, ocultando"-o da própria vida popular e social,
em um segredo de Polichinelo histórico, que era o verdadeiro modo do
Brasil falar sem falar de sua escravidão: ``não posso aceitar agora a
discussão relativa à lei de 7 de Novembro de 1831, porque essa discussão
é sobre matéria melindrosa; tanto que a câmara {[}(\ldots{}) \emph{da
oposição}{]} julgou conveniente discuti"-la em sessão secreta, e o mesmo
fez a câmara atual no ano passado'',\footnote{Idem, pág. 123.} dizia
Eusébio de Queiróz, ministro da justiça, em 1851\ldots{}

Deste modo, a verdadeira \emph{literatura} \emph{em ato} da escravidão
nas primeiras décadas do país foram, acredito, as reais determinações de
conduta da vida popular negra das \emph{posturas municipais} que, a
partir de 1830, complementaram o vazio simbólico da Constituição do
Império a respeito da instituição do cativeiro, das práticas de
possessão do homem e do trabalho forçado e sua vida na cidade existente
entre nós. Em 1838, \emph{dezesseis anos após a independência}, o
segundo Código de Posturas Municipais do Rio de Janeiro estabelecia por
fim o texto real, a vida imaginativa possível desde o concreto como
norma e como ordem para a escravaria urbana, \emph{duplicando em alguma
linguagem a própria coisa em ato da escravidão} que estava por tudo e
definindo em lei aquela mesma postura de hipercontrole que era, não por
acaso, a do erudito Luiz dos Santos Vilhena, de trinta anos antes.

Obrigava"-se então, em texto, aos escravos transitarem ``com vestes
decentes'' que ocultassem ``qualquer parte do corpo que ofendesse a
honestidade e moral pública''. Eles não deveriam proferir ``palavras
indecentes'', praticar ``gestos ou atitudes da mesma natureza'' ou
apresentarem ``quadros ou figuras ofensivas da moral pública''. Ficavam
proibidos os ``alaridos'' e ``gritos nas ruas'', mas, eram permitidos,
por interesse franco, ``cantos e pregões para facilitar o trabalho'', ou
seja, a voz do comércio cotidiano. Ficavam igualmente proibidos os
``batuques, cantorias e danças de pretos'' que incomodassem vizinhos
``dentro das casas e chácaras'', bem como, em lugares públicos, nas
``casas de bebidas e tavernas'', o agrupamento de escravos com
``tocatas, danças ou vozerias''. Não poderiam mais funcionar as ``casas
de zungu e batuque'', onde também, por vezes, ocorriam os
\emph{candomblés} dos negros. Também se proibia escravos de circularem
depois das 7 horas da noite ``sem escrito do seu senhor datado do mesmo
dia, no qual {[}declarasse{]} o fim a que ia'', e estabelecia"-se por fim
que, diz"-nos Luiz Carlos Soares: ``As `desordens' (brigas, discussões,
barulhos, etc.) eram punidas com 8 dias de prisão, multa paga pelos
senhores e 100 açoites para os escravos considerados `motores'
(causadores), recebendo os `não"-motores' a metade da pena.''\footnote{\emph{O
  ``Povo de Cam'' na capital do Brasil: a escravidão urbana no Rio de
  Janeiro do Século \versal{XIX}, op. cit.}, pág. 218.}

O desenho forte das ações, dos limites e dos corpos, biopolítica
escravocrata pós"-colonial, e o punitivismo pesado, entre lei e sadismo
público, ganhavam por fim representação jurídica, de modo a duplicar e
espelhar simbolicamente as práticas reais da nação escravista. E, entre
outras coisas, esta ação normativa da violência produtiva local visava
nitidamente à desagregação ideal do gesto cultural próprio ao escravo,
reduzindo aquele espaço de manifestação e de experiência simbólica e
corpórea que chamou a atenção sensível de Schlichthorst. Estado e
sociedade escravocratas brasileiras já tinham organicidade e estrutura
repressiva suficientes para tentar calar a vida popular mais ativamente,
mantendo"-a sob risco e terror. Era também, na forma de norma pública, a
culminação da retomada da legitimidade do poder social escravista, após
alguma instabilidade de proposição política para a restrição do tráfico
que se deu ao redor da lei de 1831, sempre na esfera da discussão de
Estado e da representação institucional, e não no mundo da vida, da
imaginação e da representação realista do que se passava no próprio
país.\footnote{Ver a respeito dos debates políticos do início dos anos
  1830, em algum jornalismo, mas principalmente na Câmara dos Deputados,
  pela aceleração e extinção do tráfico, e gradual abolição, e a forte
  reação político ideológica saquarema que se seguiu em resposta, a
  sessão ``Economia mundial, abolicionismo britânico, ação escrava e
  política parlamentar: momentos decisivos, 1831-1835'', em Tâmis
  Parron, \emph{A política da escravidão no império do Brasil,
  1826,1865}, \emph{op. cit.}.}

É a lei municipal do mundo mesmo ao qual Machado de Assis, 70 anos
depois, já no início do novo século, recolheria os restos grotescos e as
relíquias da violência, o \emph{ferro ao pescoço}, o \emph{ferro ao pé},
as \emph{máscaras de folhas"-de"-flandres}, os instrumentos de pancada, no
conto de recenseamento da mentalidade material e popular da escravidão
Imperial, ``Pai contra mãe'' -- de fato um conto \emph{tardio} na força
de sua nomeação da coisa, e também de claro \emph{estilo tardio}, do
maior escritor do tempo e do local. O mundo que emergia da negação e da
recusa do outro mundo, popular, representado no desenho ``Jogo de
capoeira'', de outro alemão, Johan Moritz Rugendas, também dos anos
1820: da dança, da música, do corpo erótico da mulher e forte e ágil do
homem negros -- desenho em que uma escrava sentada vendendo algum
alimento, em frente aos capoeiras, exibia os seios quase inteiros
através do decote aberto do vestido, e sua mão colocava \emph{o milho a
altura do órgão sexual do comprador,} em outra imagem cifrada da
negociação sexual erótica da escravidão urbana brasileira, consciente ou
inconsciente do desenhista, tanto faz -- a performance negra brasileira
tentando viver, que precisaria, a partir dos anos 1830, desaparecer do
espaço escravocrata nacional.

Daí, deste panorama da vida toda comprometida apenas com fazer operar a
escravidão sem nome, provavelmente, também se produziu a real, concreta
e evidente pobreza da literatura brasileira nos primeiros trinta anos do
pais. Sobre este tempo, vivo em ato, mas irrepresentado como conceito,
forma ou dinâmica, Antonio Candido falou de uma espécie de literatura e
de uma \emph{poesia a reboque}, mecânica e pequena, instrumentalizada
pelos valores conservadores pesados mais gerais do processo nacional.
José Veríssimo anotou a ``paralização literária'' da época, frente ao
movimento político mais amplo existente que parecia prenunciar alguma
autonomia e expressão cultural por vir.

Se houve forte agitação política nos anos de 1820 e de 1830, a
verdadeira esfera de sua representação era a da elite branca imperial
disputando posições e modos de seu próprio poder, em um teatro de
\emph{ricos entre si}, como dizia bem Machado de Assis, entre
representações oligárquicas locais, legislativo e executivo, com
figurações abstratas das ideias em disputa nas crises duradouras do
Primeiro Reinado, da Regência e do golpe do Segundo Reinado, em um
processo concreto da construção institucional nacional escravocrata e
desviante a um tempo -- na medida em que se fugia sempre da abolição do
tráfico, acertada em 1826, e novamente em 1831, com os ingleses. Como
ocorreu com o republicanismo radical de um Cipriano Barata, por exemplo,
que com seus jornais ajudou a desestabilizar o Primeiro Reinado, em um
processo que deu origem à verdadeira reação, própria do que seria o
Segundo.\footnote{Marco Morel, \emph{Cipriano Barata na Sentinela da
  Liberdade}, Salvador: Academia de Letras da Bahia, 2001.} Em seu livro
sobre os conflitos à favor e contra o tráfico na imprensa brasileira que
se seguiram à lei de extinção de 1831, Alan El Youssef tenta demonstrar
a existência de um debate e de uma presença abolicionista na cultura
política brasileira de então, mesmo que muito moderada, em um jogo de
conflito de interesses de Estado e de elites, mas diante do qual ficamos
com a impressão constante de ser sempre mais forte e bem presente o
ataque e o enquadramento saquarema do destino do tráfico, na intenção de
mantê"-lo legítimo e funcionando, do que a defesa moderada de sua
extinção, de Feijó, Evaristo da Veiga, Oliveira Coutinho e Aureliano de
Souza, baseada naquilo que, afinal, era a lei vigente, de fato tornada
ineficaz.\footnote{Alain El Youssef, \emph{Imprensa e escravidão:
  política e tráfico negreiro no Império do Brasil}, São Paulo:
  Intermeio, \versal{FAPESP}, 2016.} Tâmis Parron, estudando estas tenções
políticas, instauradas no mundo branco pela lei antitráfico, e pelos
efeitos das revoltas escravas do início dos anos 1830, particularmente a
revolta dos Malês na Bahia, embora reconheça o debate no espaço político
do poder não tem como recusar a afirmação superiora, no processo, do
sistema econômico social escravocrata nacional:

``A discussão política sobre tráfico negreiro e escravidão despertada no
início da Regência mostra bem que o problema não fora resolvido nos
debates de 1827, ocupados antes com a constituição dos poderes de
Estado. Alçada à ordem do dia, a soberania e a ordem social, deslocaram,
por assim dizer, o futuro do contrabando para o fim da agenda.(\ldots{}) Em
todo caso, se as ações populares e escravas no país podem ser ligadas ao
mundo da economia e da política, talvez devessem sê"-lo de modo pontuado,
indireto e em mão dupla. É que a conjuntura aberta pela Abdicação, pela
emancipação inglesa e pelos efeitos parlamentares dos malês acabou por
suscitar também iniciativas pró"-escravistas que estreitaram o campo
discursivo adversário e pavimentaram, nas entranhas do poder público, o
caminho para a expansão acelerada da escravidão negra. Percebe"-se então,
pela primeira vez no Brasil independente, uma estreita conexão de grupos
sociais e políticos em torno da reabertura do tráfico sob a forma de
contrabando em nível sistêmico.''\footnote{Tâmis Parron, \emph{op.cit.}.
  p. 103.}

Era o estranho mundo brasileiro que acontecia nas páginas dos jornais, e
nas esferas de Estado, em plena modernidade da \emph{era das
revoluções}: da política sem povo, e da vida sem representação.

\section{X}

Podemos lembrar também que apenas em alguns poemas muito bem \emph{não
publicados em vida}, e no diálogo de forma difícil \emph{Meditação} --
estudados hoje por Priscila Figueiredo -- expressivo da natureza da
violência da vida social brasileira e sua angústia para uma consciência
mais ampla dos horizontes nacionais, um escritor como Gonçalves Dias
deparou"-se com o problema da nossa realidade escravocrata mais comum.

Não por acaso, aquelas pequenas obras \emph{não foram publicadas} em
nenhuma das três coleções de poemas do primeiro artista significativo do
país: elas não apareceram nem nos \emph{Cantos}, do mesmo ano de 1846 do
diálogo antiescravista, nem nos \emph{Segundos cantos}, de 1848, e muito
menos nos \emph{Últimos cantos}, de 1851. Nas obras completas,
\emph{Cantos, coleção de poesias}, publicadas em Leipzig em 1857, o
poeta pode mesmo acrescentar 16 novos poemas ao conjunto, mas evitou dar
notícia do seu diálogo antiescravista mais nítido. \emph{Meditação}
apareceu apenas na revista Guanabara, em partes, em 1849, e restou
inconcluso, não publicado em livros durante a vida do autor.

Assim, embora existisse em um lugar subjetivo de indefinição, parecia
não haver um lugar real possível para tal literatura em seu próprio
mundo e cultura, já adentrando a década de 1850. Embora existisse, vinte
e cinco anos após a independência, tal ordem de pensamento não
encontrava continente simbólico no todo da cultura, que não existia para
ela, para que de fato chegasse a existir. Um problema muito tradicional
e bastante comum da trama da cultura insólita, de \emph{segunda água},
ou de \emph{ar rarefeito}, brasileira. O pacto do silêncio estrutural a
respeito do tema, do \emph{continente recusado} da trama simbólica, que
era o próprio país, não devia ser quebrado. Antonio Candido anotou
exatamente isso a respeito da obra antiescravista mais radical surgido
até aquele momento no país: ``não havia condições, na literatura, para
uma atitude rasgadamente liberal, que incorporasse o tema político à
própria inspiração, incorporando"-o à economia íntima das obras mais
significativas de um escritor, como ocorreria, vinte anos depois, na
última geração romântica.''\footnote{\emph{Formação da literatura
  brasileira,} 2º. Volume, \emph{op. cit.}, p. 52.}

O próprio Gonçalves Dias, no texto de 1846, parece ter chegado bem perto
de nomear esta sua própria situação, em que escravidão satisfeita e
trabalho do pensamento eram de fato mundos excludentes, antitéticos, no
Brasil dos primeiros tempos. Surgia a imagem do real mundo prático
brasileiro, apenas \emph{avesso ao pensamento: }

``É porque o belo e o grande é filho do pensamento -- e o pensamento do
belo e do grande é incompatível com o sentir do escravo.

E o escravo -- é o pão, de que vos alimentais -- as telas, que vestis --
o vosso pensamento cotidiano -- e o vosso braço incansável!''

Assim, o nosso primeiro artista importante, e não por acaso, aproximava
e fundia em sua escritura a rudeza e a brutalidade da vida e da
subjetivação escrava com a estupidez e a brutalidade da vida e da
subjetivação dos próprios senhores no Brasil. Pela primeira vez, em um
discurso que teria longo e importante futuro crítico, a escravidão
satisfeita brasileira era percebida como a fonte de nossa real regressão
e degradação, social, material e subjetiva.

Este mundo de pouca expressão a respeito de si próprio revelava também
um momento nacional de \emph{má} \emph{fé} singular, aquela
impossibilidade simbólica real, que Roberto Schwarz anotou, daquela
cultura produzir qualquer coisa própria a seu respeito, como o que fosse
\emph{um Kant do favor} e da escravidão. E, para termos uma medida
global da situação local, olhando para o rápido ponteiro do relógio
simbólico do tic tac mundial em que o país estava situado, e apenas do
ponto de vista da literatura, estamos falando do preciso mesmo tempo
moderno em que, na Europa, Balzac já escrevera \emph{Ilusões perdidas},
Flaubert apresentou \emph{Madame Bovary}, Baudelaire escrevia \emph{As
flores do mal}, de modo a expor com todas as letras a estrutura de
violências, inclusive subjetivas e já inconscientes, da sociedade
burguesa francesa, sempre instável na dinâmica radical da modernidade
que lhe era própria.

E, no mesmo momento, nos Estudos Unidos, Hermann Melville publicava
\emph{Moby Dick}, enquanto Edgar Allan Poe morria de tanto beber.

\section{XI}

Também havia signo de contradição, pressão real de um \emph{outro
simbólico} existente, que era algum poder, quando os debates da Câmera
dos Deputados do Império, que mantiveram vigente o pacto pela ação
brasileira de ilegalidade frente à lei da proibição do tráfico de 1831,
foram realizados como \emph{debates} \emph{secretos} de forma que
tratava"-se, amis uma vez, de coisa política que não podia ser enunciada
\emph{como pensamento} ou \emph{realidade pública} por uma real
instância de Estado. ``(\ldots{}) O assunto de quantas sessões secretas houve
sobre o tráfico, em 1837, 1848, 1850, foi como lidar com as
consequências do descumprimento da lei de 7 de novembro de 1831
{[}\emph{da proibição do tráfico para o Brasil}{]}''.\footnote{Sidney
  Chalhoub, \emph{A força da escravidão}, São Paulo: Companhia das
  Letras, 2012, pág. 122.} O país era de fato um país cindido,
tensionado em uma contradição que, antes de ser trabalhada na cultura,
precisava e era constantemente ocultada. O que aqui se realizava de fato
não podia ser dito pelo próprio Estado. O que, por seu lado, também era
política de Estado. E essa cisão da cultura, estatuto de dupla gestão da
verdade, que é, mas que não pode ser enunciada, ou, que não é, mas de
fato é, é uma das chaves históricas centrais da insolidez da vida
simbólica, subjetiva e política no Brasil.

De fato, os políticos conservadores que defenderam o instituto da
escravidão no sistema geral do Estado brasileiro, na Câmara e no Senado,
e contra as investidas inglesas, desenvolveram, nas palavras de Alfredo
Bosi, ``um discurso dominante, uma variante pragmática'' de posições de
poder originais pré"-estabelecidas desde a origem da nação a respeito do
trabalho escravo e do destino da riqueza nacional. Assim foi
estabelecido o caráter ``funcional e tópico'', segundo o crítico, do seu
auto denominado \emph{liberalismo} para si próprios, de modo que, para
aquela elite, ``mantendo, sob controle terras, café e escravos,
bastava"-lhes o registro seco, prosaico, às vezes duro, da linguagem
administrativa. É o estilo da eficiência saquarema.''\footnote{``A
  escravidão entre dois liberalismos'', Alfredo Bosi, Revista Estudos
  Avançados, \versal{USP}, no. , pág. 7.} O estilo mínimo da linguagem da
eficiência política, alheia à imaginação e ao compromisso, fundado na
eficiência máxima \emph{das práticas} de poder escravocratas.

O Brasil, que além de escravista se tornava uma nação pirata por
desrespeitar originalmente a lei que aceitara sobre o assunto, e cujas
ações de traficância seriam posteriormente perseguidas nos mares a
partir de 1845 pela Inglaterra da lei intervencionista mundial Bill
Aberdeen, era também um país que evitava amplamente falar no assunto,
\emph{simbolizar} o próprio uso que fazia, em todos os níveis, do corpo
do escravo. Configurava"-se assim uma das mais impressionantes
experiências de uma real civilização \emph{perversa} em plena
modernidade, um espaço social cuja ação prática de gozo sobre o seu
próprio \emph{objeto fetiche}, de poder e de riqueza, era mais forte
\emph{do que qualquer} \emph{vida simbólica que a regulasse}, ainda mais
de modo afirmativo, publicamente percuciente, como literatura, como
política, ou como qualquer natureza de crítica.

Este é um dos fundamentos mais profundos de nosso real
anti"-intelectualismo brasileiro mais geral. Ele significou, na raiz, a
recusa de um engajamento social produtivo no próprio espírito mais amplo
do iluminismo, exatamente quando ele emergia historicamente no horizonte
da modernidade, \emph{porque a escravidão nacional não podia ser
pensada}, com a consequente conversão da vida pública em ato, gozo
patriarcal autoritário e preguiça para os senhores, ação sádica liberal
sobre o corpo do escravo e da escrava, versus uma cultura popular
espontânea compensatória, e experimental, base da vida, também não
inscrita na consciência pública letrada. Como já foi dito, as
\emph{ideias fora do lugar} aqui, na origem de tudo, eram de fato todas
as ideias iluministas, estéticas, científicas, humanistas ou críticas
mundiais, que não correspondiam à força da ação muda e total da concreta
vida escravocrata nacional.\footnote{Ver a respeito, Roberto Schwarz,
  \emph{Um mestre na periferia do capitalismo}, São Paulo: Duas Cidades,
  Editora 34, 2000.} Trata"-se do real mundo anti"-simbólico brasileiro, o
mesmo que, ainda hoje, por exemplo, pretende calar o direito à
inquietação pública de professores, artistas e museus, em nome de que as
coisas sigam simplesmente como são, más e antissociais -- burras\ldots{} E
que calou, por trinta anos, o nome oficial da barbárie ditatorial de
1964, de fato elidindo a sua consciência da vida da democracia
danificada restaurada em 1984, exatamente por isso.

Como também o filósofo português contemporâneo José Gil escreveu, a
desrealização simbólica paralisante e o infantilismo regressivo de
mercado epidêmicos no Portugal de hoje é a superfície de afetação de um
não dito social muito mais amplo e profundo, historicamente constituído:
o não pensado, o não criticado e o não integrado do salazarismo do
século \versal{XX} -- que foi mesmo a configuração final do \emph{colonialismo}
português --; de modo semelhante, e simétrico, nossa própria vida
prática disponível para qualquer violência, de grande descompromisso e
imediatismo -- da supressão de direitos ao assassinato de pobres, da
repressão política policial à derrubada de governos eleitos, da festa
absoluta sem vida do espírito da vida da mercadoria por aqui -- se
remete à força primeira de uma nação que, existindo em ato, não podia
existir em nenhum trabalho do conceito, pacto simbólico universal, a
respeito daquilo que de fato fugia à representação \emph{dos que}
\emph{praticavam}: a pragmática escravista brasileira de quatrocentos
anos, alheia ao processo das modernas classes sociais, duplo periférico
do mesmo fundo colonial que também deforma a vida do espírito em
Portugal hoje, segundo José Gil.\footnote{José Gil, \emph{Portugal, hoje,
  o medo de existir}, Lisboa: Relógio D´Água, 2004.}

Mais do que pensamento, de modo que o pensamento fosse menos do que ela,
a escravidão brasileira era \emph{verdadeira ideologia em ato}, ou,
melhor dizendo, era \emph{puro} \emph{ato}. E por isso ela sempre
apareceu tão pouco, ou nada, em nossa literatura dos primeiros tempos. O
Brasil, no tópico, agia fortemente, mas também, de um modo histérico
muito particular, se ocultava francamente de si próprio. Realizado
enquanto ação, o gesto perverso de dispor diretamente do corpo do outro
para tudo e qualquer coisa implicava na recusa e na falência de todo um
plano, continente, espaço, realidade psíquica, produtividade e
compromisso daquilo que podemos chamar, grosso modo, de
\emph{pensamento}.

Foi apenas após a emergência da consciência modernizante da geração de
intelectuais e literatos dos anos de 1870, e da rápida renovação dos
temas e das consciências de então, pressionados pela aceleração da
cultura produtiva liberal capitalista mundial, e o absurdo histórico
concentrado da situação brasileira de grande atraso generalizado, que
sucederam"-se visões e textos que trataram e situaram o problema da
escravidão no Brasil, e sua vida social, em uma escalada significativa
da produção ideológica, articulada à presença renovadora na vida pública
cada vez mais forte do movimento abolicionista. Deste modo, em 1869
Joaquim Manuel de Macedo explicita o problema nacional da vida do
escravo em três pequenas novelinhas panfletárias, \emph{As vítimas
algozes}, e, na introdução do livro, chegava a comentar a paralisia
simbólica geral diante do fato escravocrata, insistindo na necessidade
de pensá"-lo, de dispor dele como problemática vital, em um movimento ao
contrário do auto"-recalque político e da recusa mais forte, próprios das
elites nacionais até então: ``Não queremos ter segredos, nem reservas
mentais convosco. É nosso empenho e nosso fim levar ao vosso espírito e
demorar nas reflexões e no estudo da vossa razão fatos que tendes
observados, verdades que não precisam mais demonstração, obrigando"-vos
desse modo a encarar de face, a medir, a sondar em toda a sua profundeza
um mal enorme que afeia, infecciona, avilta, deturpa e corrói a nossa
sociedade.''

O crítico Edu Teruki Otsuka, em seu renovador estudo a respeito de
\emph{Memórias de um sargento de milícias} (1854-1855), situou a
representação da escravidão na obra, na verdade uma ausência
significativa\ldots{}, diante do fato político bastante relevante de Manuel
Antônio de Almeida ter escrito abertamente contra a instituição do
trabalho cativo no Brasil em 1851, ao atacar publicamente a proposta de
Francisco Adolfo de Varnhagen de, frente ao risco iminente do
encerramento do tráfico africano, reeditar as \emph{bandeiras} e trazer
os índios do interior para o mundo do trabalho escravo imperial -- o que
se daria sob o nome fictício e falso, ainda uma vez, de \emph{tutela} --
: ``Seja como for, no momento que Manuel Antonio escrevia {[}\emph{as
Memórias\ldots{}}{]}, o tema da escravidão era tratado pelos escritores de
maneira apenas ocasional (um exemplo é a \emph{Meditação}, de Gonçalves
Dias); naquele momento, ainda predominava o nativismo indianista, e o
escravo negro não era visto senão como um elemento natural da paisagem.
Num artigo de 1854, intitulado ``Fisiologia da voz'', Manuel Antônio
enumera alguns absurdos existentes no mundo, entre os quais inclui esse:
`no Brasil há escravos', acrescentando que, embora se saiba disso,
ninguém se admira dessas coisas. Com efeito, o tratamento literário do
negro só despontaria no final dos anos 1850 (\emph{O demônio familiar}
de Alencar é de 1857; as \emph{Primeiras trovas burlescas} de Luis Gama,
de 1859), ganhando força apenas nos decênios seguintes -- e mesmo assim
imbricado por vezes num complexo ideológico duvidoso, como é o caso da
\emph{As vítimas"-algozes} (1869) de Macedo.''\footnote{Edu Teruki
  Otsuka, \emph{Era no tempo do rei, atualidade das Memórias de um
  sargente de milícias}, Cotia: Ateliê, 2016, p. 187.}

Este movimento de tomada de posse do interdito nacional -- que teria a
expressividade e o \emph{topos} constante do terror ignominioso, que
pressupõe a emergência ou fundação de \emph{um outro lugar de medida} do
mal local -- teria seu auge cultural, de fato, na poesia pública de
grandes tons \emph{vitoruguianos} de Castro Alves e, principalmente, no
meu entender, na obra prima original do pensamento social brasileiro, de
viés crítico muito forte para o local, verdadeiramente incomum no país,
o ensaio histórico de intervenção política \emph{O abolicionismo,} de
Joaquim Nabuco, já no muito avançado da hora do ano de 1883. Um livro
crítico brasileiro de alta escritura que, se apenas lido hoje em nosso
ensino médio, representaria verdadeira revolução simbólica em toda linha
-- mas as coisas da cultura desde o Estado no Brasil, sabemos, estão
indo em direção contrária a isso. Joaquim Nabuco voltou a ser excessivo
para o Brasil\ldots{}

E, todavia, por outro lado, este mesmo movimento também teria o seu auge
\emph{negativo}, especial, como não poderia deixar de ser, no conjunto
de textos panegíricos bastante raros da instituição da escravidão
brasileira, \emph{Ao Imperador: novas cartas políticas de Erasmo}, do
patriarca das nossas letras, político conservador e ideólogo do Brasil
Imperial, José de Alencar, de 1867.

\section{XII}

Este notável conjunto de textos, a despeito de humanamente indignos,
parece ter se tornado historicamente necessário ao seu tempo como reação
e resposta à profunda mudança de ares que ocorria no céu do Brasil. Dado
o avanço externo contínuo do constrangimento internacional ao país, a
ação de polícia marítima dos ingleses que enquadrou o tráfico e,
principalmente, o elemento novo da fundação e do avanço de uma opinião
pública modernizante interna, expressão de homens livres não
proprietários, que passava -- olhando para o progresso da Europa e dos
\versal{EUA}, que abolira a escravidão em 1863 -- a ligar o atraso econômico,
científico e humano brasileiro à instituição do trabalho cativo. O
conservadorismo brasileiro de tipo antigo, de fato caudatário de
estruturas simbólicas de origem colonial portuguesa pré"-nacionais, teve
finalmente necessidade de se tornar \emph{simbólico}, literário,
manifesto.

Naquelas cartas políticas públicas, o romancista José de Alencar buscou
efetivamente atualizar de modo definitivo a ideologia conservadora do
senhorio escravocrata brasileiro existente, mas que até então em geral
amplamente se calara, evitando assumir o próprio nome na esfera
permanente da vida letrada e da cultura, por redundante e desnecessário.
Mesmo que de intervenção política mais imediata, como era o caso
daquelas \emph{cartas}. Por isto, este texto entre o cinismo e a astúcia
que deu voz pública ao extremo conservadorismo \emph{saquarema}
brasileiro -- um conservadorismo atrasado e outro em relação ao
conservadorismo liberal industrial central, efetivamente burguês --
normalmente esquecido do nosso pensamento social e crítico sobre o
Brasil, é de fato tão significativo, historicamente importante.

O fato histórico marcante de um escritor escravista e escravocrata, que
era também um político, um intelectual edificante da nação e o principal
romancista do país até então, ser obrigado a se posicionar publicamente
a favor da escravidão nacional, revelava, em meados da década de
sessenta, que a verdadeira pura ideologia em ato, aquela mais poderosa
porque silenciosa e inteiramente prática, \emph{real}, que queria evitar
a todo custo a própria discussão, já estava abalada se via obrigada a
falar, em um campo que já a deslocara para uma razão pública em
movimento. Um processo simbólico profundo, que era também uma crise
material, estava em curso.

A vida ideológica brasileira se descongelava, o trabalho público do
pensamento era forçado à consciência histórica e a vida intelectual
local ganhava autonomia e relevância.

E, dentre as várias ações públicas e intelectuais do \emph{modernismo}
brasileiro da década de 1870, será também este o momento histórico que,
nada por caso, vai anteceder e vai preparar, tornar possível, o salto
mortal crítico de Machado de Assis, que vai exatamente radiografar com
muita precisão a biopolítica das vantagens escravocratas da própria
classe e elite à qual Alencar pertencia, suas formas subjetivas e de
linguagem, geradas e sustentadas na ordem escravocrata secular, ao
apresentar ao mundo \emph{Memórias póstumas de Brás Cubas}, em 1881.

As cartas de elogio à escravidão de Alencar foram escritas no momento
intermediário da expansão histórica local da tendência abolicionista, e
dão provas expressivas da grande recrudescência do pensamento
conservador nacional à respeito das possibilidades efetivas de
liquidação da estrutura jurídico, econômica, cultural e ``ontológica''
da escravidão brasileira. E, tudo parece indicar, elas estão claramente
posicionadas frente e contra os debates políticos que levariam a
instauração da lei do ventre livre, no início da década de 1870, no
processo que deu início ao ciclo final e mais forte dos debates e das
ações abolicionistas no Brasil. De fato, Joaquim Nabuco comenta, em
\emph{O abolicionismo}, as ondas de conservadorismo que desmobilizavam
de tempos em tempos o trabalho social crítico frente à escravidão, que
tiveram por volta de 1871, ano do estabelecimento da Lei do Ventre
Livre, um novo patamar de paralização das ações políticas
antiescravistas.

O livro epistolar de Alencar, de gênero antigo de tipo \emph{espelho de
príncipe}, se inscreve exatamente nesta última tentativa da cultura
conservadora de homens do poder de barrar a transformação social
econômica do tempo no Brasil, um movimento de ideologia travada,
reacionária, muito profundo de nossa vida política. Grande parte do
trabalho é uma ressentida resposta pragmática e com requintes de
retórica de um importante escritor e político ao \emph{filantropismo
mundial} da época, politicamente inglês, teoricamente francês, que
parecia isolar o campo social e simbólico do Brasil do espaço projetado
da \emph{humanidade}. O que acarretava grandes problemas, em muitos
níveis, para uma elite, e um artista, que tinham pretensão de plena
participação no concerto das nações contemporâneas relevantes. Vemos
nele com clareza o resultado melancólico e cindido, de fato precocemente
ressentido, um tipo especial de \emph{ressentimento dos senhores}, bem
brasileiro, daquilo que a exceção nacional produzia no próprio mundo
mais amplo, a medida cosmopolita a que se aspirava, e que se sabia poder
incontornável. Este mal estar da elite ligada à conservação do mal do
Brasil tradicionalmente costuma encontrar limite e resistência em uma
ordem meio iluminista democrática externa, \emph{a medida mundial das
ideias fora do lugar}, que revela a constância de nossos imensos
déficits -- o que exatamente acabamos de ver ocorrer um real exemplo
tardio com o questionamento internacional forte das certezas nacionais à
direita que moveram o golpe jurídico parlamentar dos anos 2010, e seu
resultado \emph{descivilizador} final.

Pelo mal estar geral do discurso, desenvolvido em todos os pontos
ideológicos que cercavam a questão, que era o próprio Brasil, podemos
compreender bem a violência interna, subjetiva, que calou os próprios
senhores sobre a sua prática recusada no grande \emph{significante que
vinha do todo}, que vinha de fato da reorganização liberal que movia a
história do tempo, industrial e de mercados nacionais mundial, com
aumento exponencial de produtividade, com a integração do trabalho aos
mercados internos, europeia e norte americana.

Assim, naquelas \emph{cartas} envenenadas ao Imperador, de elogio à
escravidão, ficamos sabendo, por exemplo, que:

``Nas possessões ultramarinas, e especialmente na América, o tráfico de
africanos se desenvolveu em vasta e crescente escala. Não só Espanha e
Portugal, já acostumados com os escravos mouros, como as outras
potências marítimas, Inglaterra, França e Holanda, se foram prover, no
grande mercado da Nigrícia, dos braços necessários às suas colônias.
Como explicar esta anomalia de povos, repelindo na metrópole uma
instituição que adotam e protegem no regime colonial? (\ldots{}) Esta causa
era a necessidade, a suprema lei diante da qual cedem todas as outras; a
necessidade, força impulsora do gênero humano. (\ldots{}) Não houve remédio
senão vencer a repugnância do contato com a raça bruta e decaída. Um
escritor notável, Cochin, estrênuo abolicionista, não pode, apesar de
suas tendências filantrópicas, esquivar"-se da verdade da história. Deu
testemunho da missão civilizadora da escravidão moderna, em sua obra
recente, quando escreveu estas palavras; `Foi ela, foi a raça africana
que realmente colonizou a América.' (\ldots{}) A população da Europa, longe
de transbordar, como agora, era pouco intensa naquele tempo: seu
território, embora pequeno, sobejava"-lhe. Minguados subsídios, portanto,
devia prestar às novas descobertas; e estes mesmos estorvados pela
dificuldade e risco das comunicações. Eram raras as viagens então; a
emigração nula. // Foi esta uma causa; outra a degradação do trabalho
agrícola em toda a sociedade mal organizada, que vive dos despojos do
inimigo ou dos recursos naturais do solo. A colônia era uma aglomeração
de aventureiros à busca de minas e tesouros. Sonhando riquezas
fabulosas, qualquer europeu, ainda mesmo o degredado, repelia o cabo do
alvião como um instrumento aviltante. A lavoura na América parecia uma
nova gleba ao homem livre. // Eis a necessidade implacável que suscitou
neste continente o tráfico africano. Vinha muito a propósito parodiar a
palavra célebre de Aristóteles: `Se a enxada se movesse por si mesma,
era possível dispensar o escravo.'"

Evidentemente incapaz de criticar qualquer ordem de poder, ou de
movimento mundial do Capital ao seu tempo, já de plena constituição
competitiva de Impérios comerciais coloniais e fornecedores de matérias
primas, concebendo a realidade como fixada por interesses fora de toda
questão, e dando notícia de uma elite que não conhecia a própria crítica
interna que as sociedades liberais burguesas sofriam no centro de seu
mundo, Alencar só podia afirmar o processo da nova escravidão mundial
como uma espécie de \emph{puro real}, um princípio de realidade fixado,
em que a \emph{necessidade} é o valor máximo e único, dado civilizatório
natural, orgânico, a"-histórico.

Simplesmente não se pode perguntar, deste ponto de vista, mais
exatamente \emph{necessidade do que?} que este discurso expressa. O
enunciador está fixado, de modo total, na própria posição da expansão do
capital global mercantil, desde uma perspectiva menor e periférica,
\emph{pós"-colonial} atrasada, e para ele, em sua sociedade de poderes
totais e escravos movidos como coisas através de oceanos, não deve haver
dúvida ou movimento do pensamento sobre o ponto. Se nos fiarmos por esta
proposição, era mesmo \emph{a necessidade do capital mundial},
constiuído em espaço de acumulação primitiva colonial, a razão prática
que fundava a consciência da elite escravocrata mínima brasileira. E
também, tudo o que ela não podia pensar a respeito, o seu real
inconsciente recusado, de fato o próprio mundo que produzia.

Mesmo assim, invertendo seu argumento de modo calculado, Alencar se
permitiria, do ponto de vista cultural e civilizatório, dar um lugar
específico de \emph{sujeito} à massa global de homens mercadorias que
foram os escravizados importados da África, adiantando um discurso sobre
a positividade formativa do novo mundo que teria longa história no
Brasil:

``Sem a escravidão africana e o tráfico que a realizou, a América seria
ainda hoje um vasto deserto. A maior revolução do universo, depois do
dilúvio, fora apenas uma descoberta geográfica, sem imediata
importância. Decerto não existiriam as duas potências do novo mundo, os
Estados Unidos e o Brasil. A brilhante civilização americana, sucessora
da velha civilização europeia, estaria por nascer. (\ldots{}) É, pois, uma
grande inexatidão avançar que a raça Africana nem ao menos prestou para
povoar a América {[}\emph{referindo"-se à miscigenação e ao branqueamento
secular dos negros, que avançava}{]}. Quem abriu o curso à emigração
europeia, quem fundou a agricultura nestas regiões, senão aquela casta
humilde e laboriosa, que se prestava com docilidade ao serviço como aos
prazeres a ralé, vomitada pelos cárceres e alcouces da metrópole? //
Longe de enxergar a diminuição da gente africana pelo odioso prisma de
um precoce desaparecimento, cumpre ser justo e considerar este fato como
a consequência de uma lei universal da humanidade, o cruzamento das
raças, que lhe restitui parte do primitivo vigor. Bem dizia o ilustre
Humboldt fazendo o inventário das várias línguas e famílias
transportadas à América e confundidas com a indígena: `Aí está inscrito
o futuro do novo mundo!'// Verdade profética! A próxima civilização do
universo será americana como a atual é europeia. Essa transfusão de
todas as famílias nesse solo virgem deste continente ficaria incompleta
se faltasse o sangue africano, que no século \versal{VIII} afervorou o progresso
da Europa.''

Nova civilização, e novo povo, com base na liberdade de miscigenação
brasileira, que foi o nosso verdadeiro avanço civilizatório -- de fato
um direito original exclusivo de uso do corpo da escrava pelos
senhores\ldots{} -- nova ordem humana já evocada pela autoridade de Humboldt,
seria o resultado futuro civilizatório desta imensa hecatombe global,
que foi a diáspora africana do tráfico e da escravização americana dos
séculos \versal{XVI}, \versal{XVII}, \versal{XVIII} e \versal{XIX}, em um processo econômico que realizava a
sua acumulação na Europa, e que financiou o moderno desenvolvimento
capitalista. Da riqueza, do holocausto colonial e do terror, ficaríamos
finalmente \emph{com a cultura}, de origem na miscigenação, também ela
uma ação prática e de poder, sexual.

A elite local, que representava a \emph{necessidade} mundial do
movimento da escravização em massa para sustentar a produção do
capitalismo mercantil dos séculos \versal{XVII} e \versal{XVIII}, precisava integrar em
algum ponto simbólico o valor da mercadoria \emph{escravo}, que o mundo
moderno do século \versal{XIX}, movido pelos seus próprios interesses, insiste
Alencar, transformava simbolicamente em \emph{homem}, de fato um virtual
consumidor das massas do novo mercado liberal emergente, inglês.
Consentia"-se, assim, que o escravo criava a América tanto quanto o seu
senhor. Mais uma vez, ele era uma espécie especial de \emph{meio
sujeito}; e o horizonte político e social preconizado por Alencar para o
processo inevitável da extinção da escravidão era o de que o escravo
fosse transformado gradualmente, pelo avanço dos costumes e das
tendências humanitárias na vida social brasileira -- mas não pela
legislação, pelos direitos efetivos, cujos avanços ele simplesmente
combatia -- em \emph{sujeito} de uma amistosa e benévola \emph{relação
servil}, uma \emph{tutela benéfica}, com os seus senhores de origem
branca europeia.

Uma transfiguração em servidão, este seria o destino da escravidão
nacional, no próprio mundo da vida e não no da lei pública, situação
social definida por Alencar em uma passagem do ensaio: ``metade livre e
metade cativo: uma propriedade vinculada a uma liberdade; eis a imagem
perfeita do servo.'' Interessantemente podemos lembrar que Schlichthorst
viu na situação concreta do escravo urbano brasileiro algo de algum modo
aproximado: livre para se mover como quisesse pela cidade e ``dono do
seu nariz'' para fazer suas virações, ele estava preso ao senhor a quem
devia o fruto de seu trabalho como obrigação, e seu corpo como
propriedade. Eram os \emph{meio sujeitos} do processo de criação do
Brasil, nação que também, em geral, \emph{meio os enunciava}.

Assim, Alencar antecede em tudo o sentido real do que se tornaria o
paternalismo moderno e atrasado brasileiro já no século \versal{XX} -- e aqui é
preciso lembrar apenas, para não nos enganarmos sobre a longa duração de
todos estes problemas, que os direitos trabalhistas das empregadas
domésticas no Brasil só foram regulamentados no ano de 2015, ou dizendo
melhor, no ano passado ao qual escrevo, 127 anos após o término formal
da escravidão novecentista brasileira\ldots{}

Pulando para uma imagem compensatória do sentido de toda a violência e
alienação do trabalho no Brasil, que se tornava uma própria cultura de
poder e submissão, utilizando"-se de noções cristãs católicas arcaizantes
e insistindo no valor do atraso social local em relação à modernidade
ocidental, acentuando o fundamento subjetivo, de algum modo
\emph{hiperliberal}, do destino social a ser fundado \emph{no coração do
senhor} e não na lei estruturada, o escritor de \emph{O guarani}
completava o seu discurso com imagens dignas do argumento de \emph{Que
horas ela volta?} (2015), de Anna Muylaert, já que apenas o Brasil de
hoje pôs o \emph{meio objeto} da elite de José de Alencar sob outra
perspectiva:

``A única transição possível entre a escravidão e a liberdade é aquela
que se opera nos costumes e na índole da sociedade. Esta produz efeitos
salutares: adoça o cativeiro; vai lentamente transformando"-o em mera
servidão, até que chega a uma espécie de orfandade. O domínio do senhor
se reduz então a uma tutela benéfica. // Esta transição fora preciso
cegueira para não observá"-la em nosso país. Viesse ao Brasil algum
estrangeiro, desses que devaneiam em sonhos filantrópicos nas poltronas
estufadas dos salões parisienses, e entrasse no seio de uma família
brasileira. Vendo a dona de casa, senhora de primeira classe,
desvelar"-se na cabeceira do escravo enfermo; ele pensaria que a
filantropia já não tinha que fazer onde morava desde muito a caridade.
// Estudando, depois, a existência do escravo, a satisfação de sua alma,
a liberdade que lhe concede a benevolência do senhor; se convenceria que
esta revolução dos costumes trabalha mais poderosamente para a extinção
da escravatura do que uma lei porventura votada no parlamento. // Todas
as concessões que a civilização vai obtendo do coração do senhor limam a
escravidão sem a desmoralizar. O escravo não as erige em direito para
revoltar"-se, como sucede com os mínimos favores de uma lei; ao
contrário, tornam"-se para ele benefícios preciosos que o prendem ainda
mais à casa pela gratidão. Esse cativo, se for libertado, permanecerá em
companhia do Senhor; e se tornará criado.''

É impressionante a famosa desfaçatez senhoril brasileira quando
enunciada abertamente, não há dúvida. O coração do senhor escravocrata
como a medida, fundamento político do romantismo brasileiro, do direito
à vida e ao reconhecimento do escravo é uma perola da teoria
conservadora profunda, que se funda no direito subjetivo do poder e
elide o espaço da crítica moderna do direito, da dinâmica da razão em
sua imagem universal. Esta prerrogativa da subjetivação para a
violência, que aparece para si mesma invertida, como \emph{tutela
benéfica}, \emph{caridade} e \emph{gratidão}, é consubstancial à
formação do Brasil. Os termos subjetivos do poder, imaginários, são a
verdade das coisas, não importando nenhuma mediação civilizada ao
contrário. Assim surge a benevolência patriarcal e o convite à
dependência perpétua, junto com atenuantes ideológicos do próprio
racismo embutido na proposição política e social, como o fundamento e
medida da civilização não europeia.

O resultado é aquele espírito de altiva benevolência irresponsável e
desejante de minoridade e dependência, que se pode ler bem nos romances
de Alencar, e em sua peça que representa um menino escravo urbano,
\emph{O demônio familiar.} Distanciamento familiar que não esconde o
desprezo elevado comum, a nobreza do senhor diante da negação civil do
estatuto do outro, disfarçado em interesse humano, para qual nada há a
compreender da vida inferiorizada do escravo. E, absolutamente, muito
menos a modificar. Uma posição de convivência sem dúvida, avessa ao
culturalismo excitado de Schlichsthorst, por exemplo. Bem visível, a
respeito da produção de outridade da vida escrava, na abertura de
\emph{O tronco do Ipê}, já em 1871:

``(\ldots{}) A gente pobre inclinava"-se mais à explicação de umas três ou
quatro beatas do lugar. Segundo a lição das veneráveis matronas, a causa
do desmantelo e ruina da rica propriedade fora o feitiço.// A fazenda do
\emph{Boqueirão} era mal assombrada; e em prova do que afirmavam, além
de umas histórias de alma de outro mundo, como vozes remoneadas entre os
costumados blocos; mostravam de longe a cabana do pai Benedito.// (\ldots{})
É natural que já não exista a cabana do pai Benedito, ultimo vestígio da
importante fazenda. Há seis anos ainda eu a vi, encostada em um alcantil
da rocha que avança como um promentório pela margem do Paraíba.// Saia
dela um negro velho.//(\ldots{}) Era este, segundo as beatas, o bruxo preto,
que fizera pacto com o \emph{Tinhoso}; e todas as noites convidava as
almas do Ipê um \emph{samba} infernal que durava até o primeiro clarão
da madrugada. //(\ldots{}) Ignorante das relações íntimas que entretinha o
habitante da cabana com o príncipe da trevas; tomei"-o por um preto
velho, curvado ao peso dos anos e consumido pelo trabalho na lavoura; um
desses veteranos da enxada, que adquiriram pela existência laboriosa o
direito à uma velhice repousada, e costumam inspirar até a seus próprios
senhores um sentimento de pia deferência.// O pai Benedito descera a
rocha pelo trilho, que seus passos durante trinta anos haviam cavado, e
chegou ao tronco decepado de um ipê gigante que outrora se erguera
frondoso na margem do Paraíba. Pareceu"-me que abraçava e beijava o
esqueleto da árvore // (\ldots{}) Curioso de ver de perto o tronco do Ipê,
que o preto velho tratara com tanta veneração, descobri junto às raízes
pequenas cruzes toscas, enegrecidas pelo tempo e pelo fogo. Do lado do
nascente, numa funda caverna do tronco, havia uma imagem de Nossa
Senhora em barro, um registro de S. Bernardo, figas de pau, feitiços de
várias espécies, ramos secos de arruda e mentruz, ossos humanos,
cascaveis e dentes de cobras.// -- Que quer dizer isto, pai?
Perguntei"-lhe eu apontando para as cruzes.// O velho abriu os olhos,
toscanejnado, e murmurou com a voz cava: //-- Boqueirão!\ldots{}// (\ldots{})
Acomodei"-me à sombra sobre a relva para esperar que o sol descambasse. O
preto de seu lado, como um instrumento perro a que houvessem dado corda,
começou a cantilena soturna e monótona, que é o eterno solilóquio do
Africano. Essas almas rudes não se compreendem a si mesmas sem falar
para ouvirem o que pensam. // A brisa trazia"-me por lufadas trechos da
cantilena, a que eu procurei, mas em vão, ligar um sentido.''

O interesse antropológico senhoril, que liga a vida do senhor ao destino
do escravo, é visível na passagem. O ponto de vista é do alto, casmurro
ou irônico; o que vai desaguar um dia no nosso modernismo teórico
conservador sobre a formação social do país, tão rico quanto
desengajado, de um Gilberto Freyre ou um Câmara Cascudo. Uma espécie de
olhar científico e condescendente, que tem por garantida todas as
distâncias. A partir deste lugar distanciado, de um \emph{complexo de
superioridade}, do narrador em primeira pessoa, se desdobram figuras de
crendices raciais populares, \emph{a culpa da decadência econômica de
uma rica fazenda imperial é da feitiçaria, sempre ligada ao mundo negro,
encabeçada por um velho preto do lugar}, que o narrador, modernamente, e
superior à própria cultura, lamenta com decoro, descarta e desfaz. A
modernidade patriarcal não podia pensar nestes termos. Há uma teleologia
da superioridade senhoril, que não se confunde com a cultura popular do
lugar, e também não integra simbolicamente a experiência negra. Assim,
ele também vai desfazer, sem expressar dúvida, da própria experiência
antropológica e social com o velho negro, que sabe bem ser um
sobrevivente do trabalho escravo, \emph{um veterano da enxada, desses
que adquiriram pela existência laboriosa o direito à uma velhice
repousada}, e que na política da exploração com fundamentos cristãos da
escravidão brasileira, do coração depois do uso do corpo do escravo,
\emph{costumam inspirar até a seus próprios senhores um sentimento de
pia deferência}. Diante do sentido complexo e certamente rico dos atos e
cantos tardios deste homem -- que mais tarde, com a mudança da
consciência nacional, interessariam imensamente a um Mário de Andrade\ldots{}
-- , um resto do sistema escravocrata, uma ruína humana realizada pelo
Brasil, mas também um sobrevivente simbólico poderoso do próprio
massacre -- um sobrevivente e um testemunho da dinâmica de todo
capitalismo colonial escravista dos últimos 400 anos\ldots{} -- surge a
negatividade política radical do patriarcado brasileiro, que vê na ação
poética da existência limite da história do negro apenas \emph{tudo o
que ela não é} para os próprios valores de fundo do narrador. O canto
escravo melancólico tardio, de mística popular, simplesmente não importa
em seus termos verdadeiros. Ele é assinalado somente como a decadência
de quem, deste ponto de vista, é decadente desde sempre, para esta
posição do sentido da relação social\emph{: a cantilena soturna e
monótona, que é o eterno solilóquio do Africano}. A hipótese psicológica
enunciada neste momento não oculta a superioridade ideológica
pressuposta de uma subjetividade fora de questionamento diante do
ex"-escravo, rasteiramente não dialética: \emph{essa almas rudes não se
compreendem a si mesmas sem falar para ouvirem o que pensam, procurei,
mas em vão, ligar um sentido à cantiga. }

A experiência negra é inconsistente e desautorizada. Como se pode ver,
estamos em terreno psíquico, simbólico e político oposto ao interesse
sensível e participante da da vida escrava brasileira, e seu pensamento,
do oficial dos corpos estrangeiros do Império Constitucional do Brasil
Carl Schlichsthorst, para quem, ``todas as impressões que os negros
recebem tomam forma poética, de modo que, o seu pensamento gera a rima,
e a rima gera outro pensamento''. Mesmo argumentando"-se que o restante
do romance é o desenvolvimento e a contextualização daquele quadro, que
poria e velho negro em uma situação de valor, mesmo que paternalista e
romantizada, a expressividade do \emph{desprezo benévolo} senhoril que
se percebe na passagem significa algo importante, tomando a enunciação a
contrapelo, ou, na melhor das hipóteses, como denuncia sutil de um
quadro de pouco respeito, para dizer o mínimo, pelo campo popular do
trabalho, para dizer o mínimo.

Todavia, muitos destes argumentos a respeito do poder branco e do
direito à escravidão europeia do negro, transposta à América, eram
tradicionais e tinham origem na produção do \emph{racismo cosmopolita}
europeu, para a autojustificativa do capitalismo colonial. Eles
convergiam para o núcleo mundial do poder colonial e sua superioridade
pressuposta, base do direito ao comércio mundial de homens, bem expressa
por aqui por Varnhagen em sua campanha para a escravização dos índios
locais: ``não há direito de conquista mais justo do que o da civilização
sobre a barbárie.''\footnote{Em Ana Priscila de Souza Sá, ``Dos
  cannibaes, ou Varnhagen contra os `philo"-tapuyas'", Revista
  Contraponto, Universidade Federal do Piaui, v. 6, n. 1, 2017.} Eles
podem ser verificados, com traços ainda mais explicitamente racistas do
que era próprio à mentalidade portuguesa brasileira que tendia sempre a
recusar o nome da coisa, mas também com mais convicção abolicionista,
por exemplo, no interessante estudo de Maurice Rugendas sobre o Brasil
negro, do seu famoso \emph{Viagem pitoresca ao Brasil,} de 1827. Este
era apenas um dos muitos textos europeus que dava o enquadre da
ideologia do domínio colonial racista do tempo, aqui diretamente
vinculado à experiência brasileira, provavelmente lido por José de
Alencar:

``De todo modo, a superioridade dos brancos sobre os negros não se dá
apenas nas coisas exteriores. (\ldots{}) Se trata mais de uma superioridade
intrínseca e orgânica; ela cria entre o Negro e o branco as mesmas
relações, de algum modo, que existem de parte da mulher e da criança
para com o homem. (\ldots{}) todos os dias se dão coisas que, abstraindo as
vantagens da civilização, provam uma superioridade real e física do
branco sobre o negro, e ninguém é mais disposto a reconhecer isso do que
o Negro ele mesmo; de modo que, onde esta regra age, se estabelecem
entre os negros e os brancos relações que tem muito daquelas do filho
com o pai, e nada é mais fácil a um bom mestre do que converter a
escravidão em algo bom para as duas partes. (\ldots{}) Que a emancipação dos
escravos negros na América seja um direito natural ou não, que ela lese
ou não os direitos assegurados pela lei aos proprietários, ela é de todo
modo a consequência de forças de fato existentes, e os proprietários só
poderão conservar suas vantagens se renunciarem voluntariamente a uma
parte dos seus direitos.''

A partir deste grande quadro ocidental de violências econômicas e
iniquidades, o do colonialismo mercantil e o do racismo moderno -- que
duplicava o sentido da escravidão e não existia sem ela -- que tanto
definiram o real lugar do Brasil no mundo, ao que tudo indica, a
tradicional sociedade escravocrata brasileira também se esforçara, por
fim, em meados da década de 1860, em produzir o seu próprio e improvável
\emph{Kant da escravidão}.

Porém, o movimento principal da vida social e da simbólica do poder
diante do privilégio sádico único da escravidão moderna brasileira, foi
\emph{a grande recusa} em alçá"-la à ordem própria do pensamento\emph{,}
verdadeira recusa, cujas raízes vão alcançar a própria natureza daquilo
mesmo que é, e o próprio modo que funciona, o pensamento entre nós.

\section{XIII}

O congelamento do espírito crítico e imaginativo, que correspondia ao
congelamento da vida social movida por escravos que não podia ser
pensada, fez com que na primeira metade do século \versal{XIX} nenhum brasileiro
pudesse produzir algum discurso verdadeiramente moderno sobre o assunto,
central à vida do país. Podemos dizer que realmente habitávamos uma
espécie de forte \emph{pré"-modernismo de comprometimento colonial} de uma cultura ainda sem caráter ou, mais precisamente,
um anti"-modernismo, cujo sentido era a suspensão da própria ordem
simbólica sobre o real da experiência brasileira, como mentalidade de
fundo e predominante. Nenhum discurso livre, que operasse com as
contradições amplas presentes e desse uma medida minimante arejada e
dialética da situação, da nova nação, de vida social antiga, católica e
escravista, foi, em geral, de fato produzido. O silencio sobre a coisa
era o silencio expressivo da própria cultura, que, sendo ocidental,
também buscava ser \emph{um outro ocidente}, liberal escravocrata, um
modo de ser que estou chamando de anti"-moderno. E de fato, de modo ainda
mais radical do que esta falência do tema e do objeto, os brasileiros
mal produziram alguma literatura no período. E não por acaso. É provável
também que essa ordem cultural sem caráter, vulgar, de \emph{não
pensamento}, de fato vazia de imaginação ou desejo, seja um dos vértices
reais de produção de vida política brasileira de longa duração. Ela
estaria ainda hoje presente, e ainda mais fortemente \emph{hoje},
fazendo das suas na vida brasileira.

Não há registro nos inícios do país de ninguém que tenha escrito com a
liberdade de investigação e inteligência, de modo espontâneo -- com uma
espécie de traço baudelairiano \emph{avant la lettre} -- algo como o
seguinte \emph{poema em prosa} sobre a experiência de nossa escravidão,
alcançando um resultado que, de modo muito mediado e concebendo as
instâncias sociais em jogo através da forma mais ampla do romance,
apenas Machado de Assis chegaria na escritura do século Imperial
brasileiro, mais de cinquenta anos depois:

``À frente da igreja, um telheiro sustentado por quatro colunas cobre
bancos de pedra, que permitem contemplar comodamente a vasta superfície
do Atlântico. Em face, boiam no espelho azul do mar ilhas emplumadas de
palmeiras e vestidas de vegetação tropical sob o céu límpido e arqueado
até o horizonte. Estirado num daqueles bancos, ouvindo o marulhar das
ondas, sonhei que estava novamente à bordo e naveguei, com a velocidade
do pensamento, de volta à minha terra distante. Então, ouvi pertinho de
mim um som de marimba tocada por uma negrinha mimosa, que se aproximara
e me oferecera doces. Tinha uma companheira, deitada perto, à sombra da
igrejinha, naquela cômoda atitude que caracteriza os africanos. Para não
desapontar a menina, comprei um pedaço de marmelada, bebi da sua bilha e
pedi"-lhe que dançasse. Não se fez de rogada muito tempo, chamou a outra,
entregou"-lhe a marimba e, à sua música, começou o fado, dança que na
Europa seria julgada indecente e que aqui é inteiramente popular entre
velhos e moços, brancos e pretos.

Imagine"-se uma mocinha na flor da idade, com um corpo soberbamente
formado, negra como a noite, o leve vestido de musselina branca caindo
negligentemente de um ombro, a carapinha oculta num turbante vermelho,
olhos brilhantes como estrelas, a boca fresca como um botão de rosa
desabrochando e dentes que ultrapassam as perolas em brilho e alvura;
imagine"-se esta mocinha em movimento suavemente embalante, mãos e pés
batendo o compasso da dança maravilhosa, ao lado de uma mulher bem
nutrida, verdadeira beldade africana, assentada no chão e tocando a
marimba com os dedos carnudos; ouçam"-se os sons do instrumento e o canto
que o acompanha; depois olhe"-se para mim, comodamente deitado no banco,
com o desenfado de um fazendeiro das Índias Ocidentais, tragando e
exalando o fumo aromático de um charuto, e se terá visto a cena que
pretendo descrever.

A canção que a bela filha da África cantou, enquanto dançava, deveria
ser mais ou menos esta:

\begin{verse}
\emph{Na terra não existe céu,\\
Mas se nas areias piso,\\
Desta praia carioca,\\
Penso estar no paraíso!\\[5pt]
Na terra não existe céu,\\
Mas se numa loja piso,\\
E compro metros de fita,\\
Penso estar no paraíso!}
\end{verse}

Algumas das dimensões mais profundas da civilização da escravidão dos
primeiros tempos do Brasil cabem nestas palavras. E elas nos surpreendem
bastante, por exemplo, ao indicar o amor da jovem escrava, realista, sem
recusa da dureza do mundo, pelo espaço erótico e ideológico do novo
país, que já a formara e se formava. E o ainda mais surpreendente amor
pela muito incipiente civilização da mercadoria local, que já parecia
então, no início de tudo, convencer a moça plenamente. A mercadoria, a
fita, a vaidade da beleza, o corpo erótico que dança e a sedução vital
da moça se comunicam e se integram em um eu nítido, nesta imagem que
condensa muitos mundos, de quem recém pisava numa praia que era bem
essa.

Anunciava"-se aqui com precisão fenomenológica a percepção aguda da
produção popular brasileira, que abria o ciclo de nossa vida nacional na
esfera do tempo Imperial de nossa escravidão. A mesma produção que, na
análise de Roberto Schwarz de ``Dança de parâmetros'', também Machado de
Assis observou a presença e expressão, já então como que fechando aquele
ciclo do ponto de vista dos pobres, na cantiga popular recolhida ao
final da excepcional cena de abertura de \emph{Esaú e Jacó}, da subida
do Morro do Castelo pelas madames elegantes do fim do Império:

\begin{verse}
Menina da saia branca\\
Saltadeira de riacho\\
Trepa‑me neste coqueiro\\
Bota‑me os cocos abaixo.\\[5pt]
Quebra coco sinhá, Lá no cocá\\
Se te dá na cabeça,\\
Há de rachá\\[5pt]
Muito hei de me ri\\
Muito hei de gostá,\\
Lelê, cocô, naiá.
\end{verse}

A respeito da qual o crítico comentou: ``na primeira quadra, em
português culto, a sinhá manda a menina --- presumivelmente uma negrinha
--- subir no coqueiro para botar abaixo os cocos. Na sequência, em
língua de preto e com sadismo alegre, a menina diz que, se acaso um coco
rachar a cabeça de sinhá, muito há de rir, muito há de gostar. ``Lelê,
cocô, naiá''. Completa‑se o desfile dos assuntos fortuitos, que
entretanto dimensionam o quadro (\emph{de toda a cena de abertura do
livro}). Aí está, como um comentário oblíquo sob forma de cantiga, um
ponto de vista saído da escravidão, recém‑abolida no momento em que se
escrevia o romance.''\footnote{Roberto Schwarz, ``Dança de parâmetros'',
  Novos Estudos \versal{CEBRAP}, no. 100, 2014.}

Machado -- como faria Mário de Andrade -- anota o canto popular, com
sutil graça romântica, e leve tonalidade erótica que envolve a menina de
saia branca, saltadeira de riacho, com o ambíguo e sutil verso
``trepe"-me nesse coqueiro'', entoado pelo pai violeiro da moça, a jovem
advinha do morro do Castelo. Roberto Schwarz entendeu a cena toda, entre
outras coisas, como a figuração de uma homenagem e de uma aproximação
possível, pela via do misticismo mágico religioso pactado entre as
classes, que a elite imperial branca, quase republicana ao modo
brasileiro, prestava deste modo aos pobres, ex"-escravos, agora negros e
negras livres da cidade.

Exatamente como Machado, Schlichsthorst nos dá uma imagem ligeira, mas
cheia de implicações, da vida popular e sua expressão estética no início
do processo nacional. E se ele não estiver distorcendo demais as coisas,
temos em sua verdadeira crônica um relato \emph{da ação biopolítica de
uma jovem mulher negra escrava de ganho}, já brasileira pela própria
experiência e expressão, em uma cena que, embora rápida, como um
instantâneo, tem grandes implicações.

A cena literária descrevendo a relação comercial de mutua sedução, mas
enquadrada por todos os lados e desequilibrada pelos fundamentos
escravistas patriarcais daquela realidade desfavorável à mulher negra,
pode ser lida junto com as imagens semelhantes das aquarelas de Debret
do mesmo momento histórico, principalmente com aquelas em que mulheres
escravas trabalham no comércio das ruas do Rio. Como a imagem da
aquarela ``Loja de barbeiros'', em que se vê à direita do quadro uma
jovem negra vendendo doces para um homem branco recostado e relaxado,
com seu leque contra o calor da cidade, em uma situação análoga à
conversação descrita em palavras e imagens pensamento, sobre a
experiência da coisa mesma de Schlichsthorst. Sobre esta imagem de
caráter documental, como era o princípio geral do trabalho do artista da
missão francesa no Brasil, ele anotou, em um outro momento de realismo
estrangeiro, mesmo que estritamente descritivo:

``Um vizinho do barbeiro, displicentemente recostado perto da janela,
com o leque chinês numa das mãos, deixa a outra para fora, na agradável
sensação do ar que refresca.// Recém acordado e com o estômago cheio de
água fresca, olha com indiferença o tabuleiro repleto de doces que lhe
apresenta uma jovem negra, a quem ele faz, na falta do que fazer,
algumas perguntas sobre seus senhores. Mas logo, aborrecido com esta
distração inútil, despacha"-a com esta frase de pouco"-caso: `Vai"-te
embora', expressão grosseira, empregada em todos os tons, desde o mais
amistoso até o mais injurioso; essa separação destrói as esperanças da
vendedora, bem como do cãozinho que aguarda humildemente pedaços da
doces.''\footnote{Jean"-Batiste Debret, \emph{Rio de Janeiro},
  \emph{cidade mestiça}, Org. Patrick Straumann, São Paulo: Companhia
  das Letras: 2012, p. 16.}

Com a diferença do desfecho autoritário que indica algo dos jogos de
linguagem locais, antieróticos, da cena, a experiência social
apresentada é a mesma do memorialista alemão, e há relação nítida nos
dois relatos das posturas físicas dos agentes envolvidos, uma verdadeira
sociologia dos corpos de nossa primeira cultura escravocrata. Mas o
texto do escritor é muito mais rico e faz pensar e imaginar mais, pelo
agenciamento do narrador na própria cena, qualidade literária notável,
do que o conjunto realista e o ponto de vista que se esforça por ser
neutro, de aquarela e de vinheta do pintor.

No nosso caso, as quadrinhas populares cantadas pela linda garota
escrava, na origem desse grande processo de aproximação e interesse
mútuo, recolhidas pelo escritor/narrador estrangeiro dono de
inteligência moderna, tem correspondência e de algum modo se aproximam
\emph{em complexidade} ao poema fundador da sensibilidade romântica, do
ponto de vista da elite, que foi a famosa pequena obra prima
\emph{Canção do exílio}, de Gonçalves Dias. Aquela primeira obra
significativa do campo simbólico e imaginário romântico brasileiro,
melancólico, da elite que buscava saber de si e pensar, e que
gradualmente se aproximava da terra, tem alguma trama dialética interna,
ligação subterrânea, com aquela pequena visão luminosa da utopia
brasileira possível à vida popular da garota escrava descrita por
Schlichsthorst. De fato, em ambas as poesias se sonha com uma terra
maravilhosa, que tem algum vínculo com o paraíso, e que é bem esta. A
moça inteligente que precisa ganhar a vida, e celebra seu poder de
sedução encantado com arte, que lhe permite ganhar a vida, canta mesmo
um primeiro \emph{pais tropical, abençoado por Deus,} paraíso no qual
ela já \emph{pisa as areias}, mas que, com inteligente e realista
ironia, sabe muito bem que \emph{não é o céu.} Do mesmo modo que a
tristeza profunda, embutida no poema emblema do pai de nosso romantismo
reflexivo, também o sabe bem, embora o negue relativamente a favor do
momento nacionalista que enfeixa a composição. Porém, o desterro da moça
é, surpreendentemente, afirmativo, pragmático e positivo, ele está no
fundo de um princípio de luta necessária pela vida, que erotiza com
clareza o mundo ao redor, a praia, a loja e a fita, em conjunto com o
próprio corpo da garota que performa a sua dança, viva\ldots{} Se nesse mundo
não há céu, ela também sabe fazer a \emph{ficção do paraíso}, que
durante muito tempo seria a fantasia integradora de uma nação jamais
integrada de fato, e que aproxima instâncias opostas em tudo deste modo,
\emph{estético erótico}: o homem, branco, europeu, colonial, desterrado
que sonha com um país, e com fugir dele ao mesmo tempo, e a mulher,
negra, escrava, desterrada, que ecoa esse sonho em seu próprio corpo,
dança e música, realizando um território possível como seu, como
necessidade criada. Este território sonhado pacificava as violências e
projetava emancipação no próprio ato, ao mesmo tempo que as congelava
onde estavam. Enquanto a tonalidade elegíaca do \emph{exílio} da elite
brasileira de Gonçalves Dias acentua o valor de \emph{tristeza do
Império} de sua experiência, e suas cisões subjetivas, melancólicas, com
a própria realidade nacional, uma presença que é também, desde sempre,
uma ausência. Tudo isto configurado, na poesia da menina, com elegância
e precisão minimalista e concreta, digna de um possível poeta modernista
futuro, talvez mais bandeiriana do que oswaldiana, talvez o contrário, e
contado pelo escritor alemão em uma situação humana análoga à dos
artistas e intelectuais plenamente modernos que, 150 anos depois,
puderam anotar sobre um fenômeno semelhante da vida erótica nacional,
\emph{olha que coisa mais linda, mais cheia de graça, é ela menina, que
vem e que passa, num doce balanço, caminho do mar}. Uma imagem musical,
aquela do fim dos anos 1950, início dos 60, cheia de jogos miméticos com
o corpo da mulher brasileira e a paisagem que, aproximando o samba da
grande canção americana e do jazz, seduziria o mundo, mas cuja utopia o
próprio país não sustentaria mais a partir daí. Algo deste mundo
brasileiro, estético, conceitual e \emph{ético ao seu modo}, já estava
lá, no canto da menina que dança dentro da menina, do Rio de Janeiro de
Schlichsthorst.

Nunca saberemos se a poesia da canção, a sua letra, foi feita pela moça,
por alguém de seu círculo de existência, o que é provável, ou se era
forma de presença mais ampla, também ligada, conhecida ou produzida no
mundo branco da escravidão brasileiro, o que duvido. A canção me parece
ser a expressão de uma experiência estrangeira, de aculturação e de
tomada de posse imaginária de um lugar. Acredito que ela deva ter origem
no campo da integração escrava ao Brasil, esse nacionalismo
antropológico pouco reconhecido em seus próprios termos. Porém, não
sabemos nada sobre a sua origem, tradição ou contexto. Também nada
sabemos do ritmo, da estrutura melódica e da entonação da sua
performance, se mais próxima do lundu, da modinha em transformação ou já
de algum novo canto falado possível brasileiro, que configuraria aquele
chamado fado que a menina oferecia ao senhor extasiado, que compraria as
suas bugigangas, \emph{quentinhas gostosas}. Mas sabemos que a jovem
garota, descrita como muito linda aos olhos do narrador, escolheu estes
versos e não outros para falar com o estrangeiro que jogava com ela, e
lhe pedia um agrado em troca de um trocado e algum afeto. Percebemos
nestes versos claramente os ecos estruturados de uma poesia neoclássica
portuguesa modernizada, os traços límpidos e afirmativos de uma peça de
auto"-afirmação, com o eu bastante bem desenhado, autoconsciente e
positivo, embora leve, mais ou menos como ocorria em uma ode de do
século \versal{XVIII} colonial português de algum Dirceu apresentando os seus
dotes e valores para alguma Marília, sinhá, local, em forma de sedução e
de autoafirmação.

A poesia da garota na verdade parece ser uma espécie de elo perdido,
muito moderno, auto"-responsável pela própria leitura da terra, entre a
nitidez sintética de algo de nossa poesia de ideal de arcádia,
transposta ao novo espaço histórico, e o elemento subjetivo de baixo
tom, pessoal, mas menor, inscrito no plano imaginário do eu, típico de
grande parte de nosso romantismo subjetivista e confessional. Daí a
aproximação, certamente abusiva, de plena responsabilidade deste
escritor, com Gonçalves Dias. De todo modo, uma pequena poesia
significativa, melhor do que tudo que temos notícia no período. Sendo
negra, sendo mestiça, sendo branca, a incorporação nítida do país, da
paisagem, do clima, da rua e da vida, se dá nela de forma exemplar, pela
concisão e pela leveza que surpreendem, e por fazer da mulher, do
próprio feminino, parte do sonho histórico, sem que nada pese
excessivamente sobre o eu que canta, e que, surpreendentemente, \emph{é}
plenamente. Essa existência restaurada elegante é que comove. Uma joia
viva, um trabalho real e fino de arte, acredito.

De todo modo fica claro no caso como nossa produção musical e
experiência social popular local, negra, brasileira, aquela que pode ser
``a letra"-poema, típica da modinha, ou a letra cômico"-maliciosa, dos
lundus, mas {[}\emph{principalmente}{]} a letra do falante nativo,
aquela que já nasce acompanhada da entoação correspondente'',\footnote{Luiz
  Tatit, ``O século \versal{XX} em foco'', em \emph{O século da canção}, Ateliê,
  Cotia:2004.} ou ``a forma do falar sublimada''\footnote{Na expressão
  de José Miguel Wisnik em ``A gaia ciência: literatura e música popular
  no Brasil'', em \emph{Ensaios brasileiros contemporâneos: música},
  Org. Marcos Lacerda, Rio de Janeiro: Funarte, 2016.} em uma primeira
canção brasileira, tem uma lógica formal própria, bem
estruturada, ritmo existencial, para nossa surpresa, que parece mais com
a imagem evocada por Roberto Schwarz da estrutura da dança de um
mestre"-sala, com gestos amplos e nobres com seu leque da cintura para
cima, e passos rápidos e que lidam com a instabilidade, e com o risco, a
síncope, da cintura para baixo. Nossa poderosa cultura musical popular
tem vínculos especiais com a própria cultura poética consciente, ou
inconsciente, branca e elevada, o nome que o poder daria a si mesmo,
então em desenvolvimento.

O canto negro aqui, a julgar pelo fado representado da mocinha de
Schlichthorst, parece ter sido poeticamente estruturado, desenhado e
aproximado estrategicamente de algo da experiência estética literária da
cultura dominante que enquadrava o seu reconhecimento -- como faz efeito
mesmo sobre mim até hoje -- cultura literária em evolução que,
invertendo os termos, e com alguma esperança modernista anacrônica,
talvez tenha aprendido com o próprio canto popular, organizado e em
contato com as coisas, o que ela mesma viria a ser um dia\ldots{} Isso tudo é
muito diferente do gesto de corte profundo, de quebra e de encantamento
poderoso reunificado em outra natureza de performance que um crítico da
chamada \emph{jazz} \emph{theory}, como Fred Moten, viu na origem da
vanguarda musical negra norte"-americana.

Não por acaso, Moten inicia a sua recuperação teórica da consciência
agonística da experiência estética da performance negra de vanguarda
americana, no choque e na alienação do corpo e na dor, com a cena
primária, para ele central, como o negativo que emana sua diferença,
\emph{do espancamento impiedoso de uma escrava por um senhor} nas
memórias negras \emph{Narrative of the Life of Frederick Douglas --} o
relato de vida do ex"-escravo que se tornou um importante abolicionista
americano, parceiro político de Abraham Lincon --. Aquela cena é
entendida pelo crítico como marco traumático e prática que estaria no
fundo da recomposição da experiência popular pelo grito/som, pelo canto
-- já destinado ao fonógrafo -- que integraria a vida em outro lugar, o
canto que é grito, \emph{da real mercadoria que fala}, que é ruptura,
que é gozo estrangeiro e exotérico mas marcado, que é amor, utopia
formada e outra, em um mundo em que tudo, inclusive o vínculo amoroso
entre os membros da própria comunidade, e entre os sexos, se perdeu e
estava sobre o regime do traumático.\footnote{Fred Moten, ``Resistence of
  the object: aunt Hesters Scream'', \emph{In the break: the aesthetics
  of the black radical tradition,} Minneapolis: University of Minnesota
  Press, 2003.}

Ao contrário, na dança sutil, elegante, clássica moderna de nossa menina
escravizada para o seu senhor interessado, a forma brasileira de
fundação do erotismo no corpo de performances singulares negras é de
outra natureza: é a dança contida e expressiva, ``\emph{natural e
indecente}'', de uma mútua sedução possível como espaço social,
\emph{relação sexual cultural}. É o espaço ambíguo na rua, na economia e
na cultura sexual do novo país, que o Brasil determinou para negros e
brancos de sua origem. Nos dois casos nacionais, com a poderosa e vital
marca negra, o \emph{não realizado na vida social, no país, se realiza
na canção}, mas de modos efetivamente diferentes, e essa diferença é o
grão de areia de uma historicidade específica que diferencia e acelera
as horas históricas. Assim, por isso, o mundo americano vai levar a
performance negra à imagem final verdadeira de um Hendrix destroçando a
sua guitarra, junto com o hino americano, diante de seu país sempre em
guerra contra o mundo, e contra o seu resto colonial que se integra
assim, como força e como choque, técnico, de vanguarda, em um gesto
maior da arte do século \versal{XX}, enquanto no mesmo momento histórico, nosso
Hélio Oiticica ainda traz a sinuosidade e o improviso erotizante da
Mangueira, adocicando e arredondando as distâncias entre os mundos, em
um devir o seu corpo feminino, de passista, e seu lirismo, de Cartola ou
intensidade cordial de Elza Soares, para os seus Parangolés, como o
último gesto encantado de uma civilização eroticamente interessada, do
improviso e que positiva o inacabado, que convida à dança e à alegria,
mas que se tornava, desde então, definitivamente impossível naqueles
termos de invenção e liberdade. A situação de nossa performance popular
na origem da menina negra que dança e canta e encanta, em uma cena
enquadrada pelos limites do desejo e do olhar do senhor, em 1824, já
tinha um elemento preciso daquela condição de nossa música popular,
definida por José Miguel Wisnik, como algo cujo ``saber alegre'',
poético musical, depende ``de sustentar a provisoriedade na satisfação,
e a satisfação na provisoriedade, vale dizer, o atributo que dá o grão
de sal a voz que canta (o encontro feliz da interinidade com a
perpetuação, diria Tatit).''\footnote{José Miguel Wisnik, \emph{op.
  cit.}, p. 32.} E aqui, satisfação e provisoriedade não eram apenas a
chave estética profunda de uma prática produtiva, mas eram mesmo \emph{a
própria situação real de possibilidade da existência}.

E também, noutra direção, pelo lado implícito da violência, na passagem
do escritor alemão estão cifrados os jogos sexuais culturais, o sexismo
e o machismo tradicionais, próprios à cultura popular e ao modernismo
populista brasileiro. A ligação possível das diferenças no primeiro
nível de emanação mimética de um próprio corpo, sexual, que o samba
sempre produziu e seduziu: ``a vitória dos nossos ancestrais'' no samba,
porque ``fizemos do corpo a coisa mais bela que se tem na vida, pois ele
é a sua única razão de ser'', dito por Paulo Lins em \emph{Desde que o
samba é samba}. Um mundo que tem origem nesta ampla negociação prática,
que faz cruzar a visão do paraíso do corpo erótico da mulher, da negra
ou da mulata brasileira -- em detrimento dos corpos mal cuidados, da
preguiça e do fausto, das senhoras portuguesas/brasileiras -- com o
elogio erotizante do seu senhor, em busca de vantagens humanas possíveis
frente à existência limite do desamparo social e da ordem autoritária ao
redor, que enquadra tudo. \emph{Que nega é essa}? É o espaço erótico
fundamental, sexualizado e negociado, que criou a cultura brasileira
exatamente aí, entre o olhar desejante do senhor e o corpo funcionando
como arma estética e erótica da escrava, no nível da sobrevivência, mas
também da pulsão, para um virtual, e socialmente impossível, mútuo gozo.
Ou, inversamente, o aberto ao novo, ou impossível, olhar desejante da
mulher negra, para uma civilização cultura de reparação e
reconhecimento. A origem sexual real, fonte de cultura \emph{desde os
corpos} e também de violências sublimadas sem fim na expansão do espaço
moderno brasileiro.

Estas práticas já matizadas tinham um enquadramento real e forte muito
anterior, desde a vida da violência direta da ordem colonial
escravocrata, até a nova ordem de país de população escravizada.
``Obviamente se trata de relações radicalmente hierarquizadas: o padrão
do intercurso inter"-racial rola num só sentido: o grupo dominante branco
fornece sempre o genitor e mais raramente o marido, enquanto a
comunidade negra dominada cede sempre a mãe e mais raramente a esposa.
Desde logo, a miscigenação se combina com a aculturação para dar lugar
ao processo social de mestiçagem.''\footnote{Luiz Felipe de Alencastro,
  \emph{O trato dos viventes}, \emph{op.cit.}, p. 346.} E desta mútua,
mas desiquilibrada no poder, sedução e relação sexual na linguagem, e na
vida, no nosso caso literário, nasceria a \emph{máquina mulata}
brasileira, o complexo cultural imaginário de longa duração, apoiado no
corpo, da figura feminina popular de origem negra, figura aberta à
miscigenação desde a origem nacional, por violência ou por jogo de
linguagem, que seria emblema e objeto da busca de mediação com o popular
própria de nosso modernismo, crítico ou não, e também dos próprios
sexismo e machismo populares brasileiros, com suas formas de longa
duração que vão de Schlichsthorst a Di Cavalcante, de Vinícius de Moraes
a Jorge Ben.

Era o mundo concreto que daria origem à figura social da
\emph{superexcitação sexual} da mulata, e por lógica metonímica à ideia
da \emph{mulher brasileira}, cantada em prosa e verso como
\emph{civilização erótica baseada no balanço sensual do que era um corpo
real} -- um duplo popular, feminino, do \emph{homem} \emph{cordial},
masculino, e dependente --, no qual ``\emph{quando você se requebrar
caia por cima de mim, caia por cima de mim, caia por cima de mim}'',
sendo a mulata ainda mais exposta ao erotismo ``pela indeterminação de
seu lugar de classe pela indeterminação do seu lugar de raça'', como
disse Gilberto Freyre, produto mestiçado desta relação social sexual,
origem de povo, de cultura e de biopolítica no Brasil. Porque a ação
sexual de senhores e escravas na rua e na vida das cidades diminuía a
dor violenta do cativeiro, como gesto micro"-político utópico de
encontro\ldots{} ou, a aumentava, como produção de cultura do reconhecimento
somente desde aí, sublimando forçosamente os terrores e na medida em que
se tornou nova ordem de cativeiro, biopolítico?

Uma prática erótica política, que tentava inverter os sinais claros
sadomasoquistas da relação social em jogo, a favor minimamente das
mulheres negras, e que já aparece sob o signo protetor do amor romântico
possível entre as raças na poesia popular semi"-paródica de um Luiz Gama;
e que a mulata Joana, já em outro contexto histórico, no adiantado da
hora de fevereiro de 1873, em uma noite de carnaval, já podia recusar
abertamente, desejando outra posição de direito e de sustentação
biopolítica, cidadã e de valor feminista negro \emph{avant la letre} ,
como Wlamyra R. de Albuquerque nos fez ver em sua pesquisa:

``Já era pouco mais de duas horas da manhã, e Idelfonso Raimundo da
Silva ainda estava disposto a `entregar"-se aos divertimentos', enquanto
seus amigos se despediram dele e de Momo. Sozinho, resolveu que uma
investida amorosa completaria a felicidade daquela noite e passou a
galantear Joana, uma mulata forra, que também preferiu prolongar a festa
consumindo mais algumas doses de `espíritos fortes' na mesa ao lado. E
como tudo continuava a correr em bom carnaval, Idelfonso insistia nos
galanteios a Joana, cada vez com mais entusiasmo e menos pudor. Aquele
teria sido o desfecho perfeito para a noite, julgou ele, se não fosse a
relutância de Joana diante de suas `gentilezas'.// Ao convidá"-la para
tomar um copo da sua bebida, `o que {[}lhe{]} parecia muito natural em
tais dias, qualquer que seja a posição da pessoa a quem se oferece', foi
agredido com as `maiores grosserias'. Além de não se mostrar disposta a
compartilhar as últimas horas da madrugada com Idelfonso, Joana também
não demonstrou nenhuma gentileza para desvencilhar"-se do pretendente.
Diante da recusa da liberta, ele `observou"-lhe que não era essa a
maneira de corresponder a uma oferecimento tão delicado, e que ela não
conhecia o seu lugar'. Não tardou para que, `tomada da maior fúria',
Joana lhe atirasse pedra, insultos e ameaças.// Idelfonso era
comerciante; Joana, mulata e liberta. O encontro no botequim resultou em
processo"-crime por lesões corporais, no qual ele era a vítima e ela era
a ré.''\footnote{Wlamyra R. de Albuquerque, \emph{O jogo da
  dissimulação, abolição e cidadania negra no Brasil}, São Paulo:
  Companhia das Letras, 2009, p. 31.}

Sobre \emph{a fita}, que tem lugar muito especial de desejo e de
subjetivação na poesia, o amor vaidoso pelo próprio corpo e seus adornos
da jovem negra brasileira, modo de expressão da unidade do si mesmo,
narcisismo de vida e força social de algum poder no plano imaginário,
mas que, também, vai ligar com alguma facilidade o desejo político de
emancipação popular com o poder sedutor da própria vida das mercadorias
modernas, a ordem futura de um capitalismo de consumo e de massas --
pois, como sabemos, \emph{com aquela gravata florida qualquer homem feio
vira príncipe, até eu\ldots{},} e também, \emph{eu vou torcer pela paz, pela
alegria, pelo amor, pelas moças bonitas e pelas coisas uteis que se pode
comprar com dez cruzeiros\ldots{}} -- para o grande embaraço de pensadores
críticos marxistas, como eu, que propõem o trabalho perdido e
melancólico da negatividade revolucionária diante dos poderes
subjetivantes de dominação da mercadoria, ordenados previamente pela
forma Capital\ldots{}, é preciso lembrar que esta opção profunda de desejo e
de vida popular pela força estética integradora de um sujeito vem de
longe, vem de fato de muito antes do seu redesenho pela experiência
social brasileira:

``Nesta área {[}\emph{do mercado Zobeme, de Abomé, capital do reino de
Daomé, grande centro ``produtor'' e exportador de escravos no século
\versal{XVIII} e \versal{XIX}, grande parte para o Brasil português e Império}{]}
expunha"-se cerâmica; naquela, peixe seco; naquela outra, azeite de
dendê; ali, panaria; acolá, frutas; mais adiante, amuletos e objetos
religiosos. Predominava a venda de comida feita. E quase todas as
barracas tinham atrás delas mulheres, que dominavam, como em outras
partes da África, o comércio a retalho.// Com um pano colorido enrolado
à cintura e a descer até embaixo dos joelhos, de cabelos entrelaçados e
usando muitos colares, braceletes e tornozeleiras, elas amamentavam os
bebês, vigiavam as crianças, fumavam seus longos cachimbos e comentavam
as notícias do dia, enquanto esperavam os fregueses. Estes, se homens,
também trariam o torso e os braços nus, e como elas estariam descalços,
pois só o rei e o \emph{agasunon} (o sacerdote de Agasu, o filho do
leopardo que fundara a linhagem real) podiam por sandálias. Muitos
usavam por debaixo do pano enrolado à cintura uma espécie de ceroulas
frouxas e tinham a cabeça penteada cuidadosamente, com trancinhas ou
múltiplas repartições, quando não as cobriam com um barrete, gorro ou
chapéu. Os bem situados na vida jogavam sobre o ombro esquerdo um pano
longo e largo (com cerca de 2,50m de comprimento e 1,50m de largura) e o
enrolavam por baixo dessa espécie de toga romana um camisão de mangas
largas, que ia até a cintura. Os grandes senhores não dispensavam os
colares e as pulseiras.''\footnote{Alberto da Costa e Silva,
  \emph{Francisco Félix de Souza}, \emph{mercador de escravos}, 4ª.
  edição, Rio de Janeiro, Nova Fronteira, 2012.}

Por fim, o fato escandaloso de que um aventureiro alemão, de passagem
pelo país, negociando a sua própria identidade e sonho com o daquela
nova terra, pudesse enunciá"-la com tanta precisão e graça, mas que,
efetivamente, só no século \versal{XX} brasileiro discursos modernistas pudessem
chegar a se aproximar do país naqueles termos, revela, pela própria
ausência histórica \emph{da nossa escritura nacional} sobre a coisa ao
seu tempo, uma outra dimensão, vetusta e \emph{casmurra}, de nossa
formação simbólica na escravidão. Para não dizermos, como efeitos
duradouros de cultura, plenamente \emph{burra}, de valorização da vida
não intelectual ou consequente do que é o nosso próprio caso histórico.
A única inteligência produzida ao longo do tempo da elite brasileira
para a condição social que ela de fato produzia era a inteligência da
sofismática volúvel, infernal, expressa como coisa em si por Alencar, e
como objeto crítico radical por Machado de Assis. É também uma
\emph{tristeza do Império} -- \emph{a dos conselheiros que esqueciam a
dor dos escravos, sonhando com os telefones futuros\ldots{}} -- a
impossibilidade de poder simplesmente contar a sua realidade, o que, em
literatura, também sempre tem correspondência com sonhá"-la, e, assim
chegar mesmo a desejá"-la em alguma dimensão.

Por que jamais um novo brasileiro daqueles tempos da origem, com seus
códigos portugueses e católicos de conduta, com seu entulho simbólico
clássico dissociado do senso da história, poderia expor a verdade
erótica ambígua daquele seu próprio mundo, todavia um mundo de grandes,
imensos, privilégios, desde o poder do dinheiro global até o acesso
direto aos corpos, para o trabalho e para o gozo, violentados,
produzidos pela presença da vida escrava. Jamais ele poderia se ver como
o novo nababo tropical, \emph{fazendeiro do ar das Índias Ocidentais},
que de fato era, cultivando uma cultura erótica de acesso ao corpo
feminino negro, de preguiça, de prazeres e de estética tropical, ou
\emph{tropicalista}, que se duplicava na vida ambígua, também convite
interessado ao erotismo e à oscilação entre trabalho e preguiça, da
própria casta trabalhadora, a grande escravaria urbana brasileira.

Assim nenhum brasileiro do tempo poderia definir bem o que era o
\emph{fado} por exemplo -- palavra que ecoa o sentido do trabalho
estafante, e o destino -- a \emph{dança} \emph{popular} que deu origem
ao samba, nos termos entre reais e conceituais que Schlichthorst
realizou -- por ter mobilidade geográfica, mobilidade de classes,
mobilidade de corpo e de desejo, e mobilidade de pensamento -- livre o
suficiente para de fato ser \emph{simbolicamente livre} no país tropical
dos senhores e dos escravos, hiperdeterminado na mentalidade, mas quase
experimental no mundo da vida -- : ``o fado consiste num movimento
trêmulo do corpo que, suavemente embalado, exprime os sentimentos mais
sensuais de um modo tão natural como indecente''. E, no entanto, os
brasileiros que não podiam dizer deste próprio movimento cultural,
adoravam e usavam a dança negra de expressão e erotismo, como, em uma
prévia de \emph{Sobrados e mucambos}, ficamos sabendo no relato do
soldado alemão.

Uma definição crítica sintética e moderna, a daquele fado, que nos
permite mesmo intuir o teor erótico do encontro com a jovem escrava,
dançarina e poeta diante do mar e do céu do Rio de Janeiro \emph{como
ele era}. Definição elegante, precisa e suficiente, que deixa entrever
os dois polos do encontro sexual e cultural das raças e dos poderes no
Império tropical escravocrata: algo do \emph{corpo natural}, talvez da
dança original africana que certamente devia ter outro caráter, e algo
do \emph{corpo indecente}, desta dança já referida às tramas sociais, de
sedução, objetificação, enfrentamento e desacato controlado da ordem de
poder colonial/nacional brasileira, ibérica católica, em que as
escravizadas jogavam abertamente, na dança, com o poder de acesso dos
senhores ao seu corpo e de simultânea negação pela cultura deste próprio
corpo. Um corpo simultaneamente \emph{natural} e \emph{indecente}, seja
lá o que isso queria dizer, assim africano e brasileiro.

Nenhum brasileiro pode nomear ao seu tempo este mesmo mundo brasileiro,
que também, em meio ao próprio terror, era composto de joias humanas,
\emph{princesas à venda}, como aquela menina descrita por Schlichthorst,
um homem que, atravessando oceanos, continentes e sonhos para vir
trabalhar no Brasil de 1820, meio como servo, meio como senhor,
atravessou todas as estruturas simbólicas nacionais congeladas da época
e tocou, pela primeira vez, a cultura brasileira moderna, já formada
\emph{nesta visão}, nos dando notícia fresca de uma de nossas primeiras
artistas.